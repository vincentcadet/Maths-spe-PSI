  
\documentclass[12pt,a4paper]{article}
% Engine-specific settings
% Detect pdftex/xetex/luatex, and load appropriate font packages.
% This is inspired by the approach in the iftex package.
% pdftex:

\usepackage[T1]{fontenc}
\usepackage[utf8]{inputenc}
\usepackage[french]{babel}
\frenchbsetup{StandardLists=true}
\usepackage{enumitem}
\usepackage{systeme}
\usepackage{amsmath,amssymb}
\usepackage[thmmarks,amsmath]{ntheorem}
\usepackage[colorlinks=true]{hyperref}
\usepackage{fullpage}
%\usepackage{eulervm}
\usepackage{graphicx}
\usepackage{array}
\usepackage{multicol}
\usepackage{geometry}
\geometry{tmargin=1.5cm,bmargin=1.5cm,lmargin=1.5cm,rmargin=1.5cm,headheight=12cm}
\usepackage{array}
\usepackage[svgnames,table]{xcolor}
\usepackage[tikz]{ bclogo}
\usepackage{pifont}
\usepackage{url}
\urlstyle{same}
\usepackage{pifont}
\usepackage{multicol}
\usepackage{diagbox} %oblique dans les tableaux
%\usepackage[framemethod=TikZ]{mdframed}
\usepackage{fancyhdr}
\pagestyle{fancyplain}
\lhead{\textit{CSI2B-PSI TD10}}
\chead{\textsc{Intégrales généralisées}}
\rhead{\textit{2021-2022}} 

\everymath{\displaystyle}
\renewcommand*{\thefootnote}{\fnsymbol{footnote}}
\newcommand{\un}{(u_n)_n}
\newcommand{\R}{\mathbb{R}}
\newcommand{\C}{\mathbb{C}}
\newcommand{\Q}{\mathbb{Q}}
\newcommand{\Z}{\mathbb{Z}}
\newcommand{\N}{\mathbb{N}}
\newcommand{\K}{\mathbb{K} }
\newcommand{\I}{\mathbf{i}}
\renewcommand{\Re}{\mathcal{R}e}
\renewcommand{\Im}{\mathcal{I}m}
\DeclareMathOperator{\Ima }{Im}
\newcommand{\E}{\mathrm{e}}
\newcommand{\conj}[1]{\overline{#1}}
\newcommand{\diff}{\mathop{}\mathopen{}\mathrm{d}}%element differentiel
{%
\theoremstyle{break}
\theoremprework{%
\rule{0.5\linewidth}{0.3pt}}
\theorempostwork{\hfill%
\rule{0.5\linewidth}{0.3pt}}
\theoremheaderfont{\scshape}
\theoremseparator{ ---}
\newtheorem{Prop}{%
\textcolor{blue}{Proposition}}[section]
}

{%
\theoremstyle{break}
\theoremprework{%
\rule{0.6\linewidth}{0.5pt}}
\theorempostwork{\hfill%
\rule{0.6\linewidth}{0.5pt}}
\theoremheaderfont{\scshape}
\newtheorem{Theo}{%
\textcolor{red}{Théorème}}[section]
}


{%
\theoremheaderfont{\sffamily\bfseries}
\theorembodyfont{\sffamily}
\newtheorem{Def}{%
\textcolor{green}{Définition}}[section]
}

{%
\theorembodyfont{\small}
\theoremsymbol{$\square$}
\newtheorem*{Dem}{Démonstration}
}

{%
\theorembodyfont{\small}
\newtheorem*{Exemple}{Exemple}
}


{%
\theorembodyfont{\small}
\newtheorem{Exo}{Exercice}
}

{%
\theoremnumbering{Roman}
\theorembodyfont{\normalfont}
\newtheorem{Rem}{Remarque}
}




\begin{document}

 \begin{Exo}
	Primitiver les fonctions $f$ telles que $f(x)=...$ (préciser l'intervalle où évolue $x$)
	\begin{multicols}{4}
		\begin{enumerate}
			\item 
			$\ln x$
			\item 
			$\frac{1}{x^2+a^2}$ avec $a\neq 0$
			\item
			$\left( \frac{x}{e}\right) ^{x}\ln x$
			\item
			$\frac{xe^{x}}{\sqrt{1+e^{x}}}$
			\item 
			$\frac{x}{\sin ^{2}x}$
			\item 
			$\frac{1}{\left( 1+x+x^{2}\right) ^{2}}$
			\item 
			$\frac{\arctan x}{\sqrt{x}}$
			\item 
			$\frac{\arctan x}{x^{2}+1}$
			\item 
			$\frac{1}{\sin x+\cos x}$
			\item 
			$\frac{\cos x}{4+\sin ^{3}x}$
			\item 
			$\sin \left( \ln x\right) $
			\item 
			$\ln \left(
			1+x^{2}\right) $
		\end{enumerate}
	\end{multicols}
	%$$\left( \frac{x}{e}\right) ^{x}\ln x,\frac{xe^{x}}{\sqrt{1+e^{x}}},\frac{x}{\sin ^{2}x},\frac{1}{\left( 1+x+x^{2}\right) ^{2}},\frac{\arctan x}{\sqrt{x}},\frac{\arctan x}{x^{2}+1},\sin \left( \ln x\right) ,\ln \left(
	%1+x^{2}\right) ,\frac{1}{\sin x+\cos x},\frac{\cos x}{4+\sin ^{3}x}$$
\end{Exo}


\begin{Exo}(merci L.Garcin...)
	CV et calcul éventuel de:
	\begin{multicols}{4}
		\begin{enumerate}
			\item
			$\int_{0}^{+\infty} \frac{1}{4+t^{2}} \mathrm{~d} t$	
			\item
			$\int_{0}^{2} \frac{1}{4-t^{2}} \mathrm{~d} t$
			\item
			$\int_{0}^{\frac{1}{3}} \frac{1}{\sqrt{1-9 t^{2}}} \mathrm{~d} t$
			\item
			$\int_{0}^{+\infty} \sin (t) \mathrm{d} t$
			\item
			$\int_{3}^{+\infty} \frac{\mathrm{d} t}{t^{2}-3 t+2}$
			\item
			$\int_{0}^{1} \ln (t) \mathrm{d} t$
			\item
			$\int_{2}^{+\infty} \frac{\mathrm{d} t}{t \ln t}$
			\item
			$\int_{0}^{+\infty} e^{-a t} \mathrm{~d} t$
		\end{enumerate}
	\end{multicols}
\end{Exo}

\begin{Exo}
	Redémontrer rapidement $\cosh^2-\sinh^2=1$ et $\tanh'=\frac{1}{\cosh^2}$. Etablir l'existence et calculer $$I=\int_2^{+\infty}\frac{\diff x}{x^2\sqrt{x^2-4}}$$
\end{Exo}

\begin{Exo}
	Calculer $\displaystyle\int_{0}^{1}\frac{\diff x}{\sqrt{1-x^2}}$ après avoir établi son existence. En déduire pour $a<b:\displaystyle\int_{a}^{b}\frac{\diff x}{\sqrt{(x-a)(b-x)}}$.
\end{Exo}

%\begin{Exo}
%	Montrer que $\displaystyle\int_{0}^{\infty}\frac{\sin^2t}{t^2}\diff t$ converge, en déduire via une IPP que $\displaystyle\int_{0}^{\infty}\frac{\sin t}{t}\diff t$ converge et que les intégrales généralisées ont même valeur.
%\end{Exo}

\begin{Exo}
	Existence et calcul de $\displaystyle\int_0^1\sqrt{\frac{t}{1-t}}\diff t$, $\displaystyle\int_0^1\frac{\diff x}{(1+x)\sqrt{1-x^2}}$ (poser $x=\cos (2\theta)$).
\end{Exo}

\begin{Exo}
	Notons pour $n\in \N:I_n=\displaystyle\int_0^{+\infty}t^n\E^{-t}\diff t$. Justifier l'existence de $I_n$ et trouver une relation de récurrence entre $I_n$ et $I_{n-1}$. En déduire $I_n$.
\end{Exo}

%\begin{Exo}
%	Calculer pour $0<a<b<1:\displaystyle\int_a^b\frac{\ln(1-t^2)}{t^2}\diff t$, en déduire l'existence et la valeur de $\displaystyle\int_0^1\frac{\ln(1-t^2)}{t^2}\diff t$.
%\end{Exo}

\begin{Exo}\ 
	\begin{enumerate}
		\item Montrer que $\int_{ 0}^{\frac{\pi }{2}}\ln (\sin t)\diff t$ et $\int_{ 0}^{\frac{\pi }{2}}\ln \left( \cos t\right) \diff t$
		convergent et ont même valeur. 
		\item Calculer leur somme, en déduire leur valeur. En déduire la valeur de $\int_{0}^{\pi }t\ln (\sin t)\diff t$
	\end{enumerate}
\end{Exo}

\begin{Exo}
	Soit $I=\displaystyle\int_0^{+\infty}\frac{t}{1+t^3}\diff t$ et $J=\displaystyle\int_0^{+\infty}\frac{\diff t}{1+t^3}$. Montrer que $I,J$ existent et sont égales, en déduire leur valeur.
\end{Exo}

\begin{Exo}
	Montrer à l'aide d'une IPP que $\int_x^{+\infty}e^{-t^2}\diff t\underset{+\infty}{\sim}\frac{e^{-x^2}}{2x}$.
\end{Exo}

\begin{Exo}
	Existence et calcul de $\displaystyle\int_0^{+\infty}\left(\int_x^{+\infty}e^{-t^2}\diff t\right)\diff x$.
\end{Exo}

\begin{Exo}
	Nature de $\int_0^{\infty}\cos\left(t\right)\diff t$ et de $\int_0^{1}\cos\left(\ln t\right)\diff t$.
\end{Exo}

% \begin{Exo}
% 	Etudier la CV de $\int_{\R_{+}^{\ast }}\frac{\sin ^{2}t}{t^{\alpha }}\diff t,\int_{\R_{+}^{\ast }}\frac{t^{\alpha }}{e^{t}-1}\diff t,\int_{\R_{+}^{\ast }}\cos \left( e^{t}\right) \diff t$
% \end{Exo}

%\begin{Exo}
%	 Préciser les valeurs de $n\in \N$ telles que : 
%$\displaystyle
%\int_{\R_{+}}\frac{dt}{\left( t+\sqrt{t^{2}+1}\right) ^{n}}
%$
%converge. 
%\end{Exo}

 \begin{Exo}
	Calculer (après avoir prouvé leur existence) 
\begin{equation*}
I=\int_{0}^{\frac{\pi }{2}}\frac{\cos x^{\sin x}}{\cos x^{\sin x}+\sin x^{\cos	x}}\diff x,J=\int_{0}^{+\infty }\frac{\ln t}{1+t^{2}}\diff t,K=\int_{0}^{+\infty}\ln\left(1+\frac{1}{t^2}\right)\diff t
\end{equation*}

\end{Exo}


\newpage

 \begin{Exo}
	Calculer $\displaystyle\int_{0}^{1}\frac{\ln x}{\sqrt{x}\left( 1-x\right) ^{3/2}}dx$
en posant $x=\sin ^{2}\phi .$
\end{Exo}

\begin{Exo}
	 Etudier l'existence des int\'{e}grales suivantes (discuter suivant $%
\alpha $ et $\beta $)\thinspace : 
\begin{equation*}
\int_{0}^{1}\left\vert 1-x^{\alpha }\right\vert ^{\beta }dx,\int_{0}^{1}%
\frac{\left( \ln \left( 1-x\right) \right) ^{\alpha }}{x^{2}}dx
\end{equation*}
\end{Exo}

\begin{Exo}
	Convergence absolue, convergence de $\displaystyle\int_{\R_+}\frac{e^{\I t}}{1+\I t}\diff t$. Convergence et calcul de $\int_{0}^{+\infty}xe^{-x}\sin x\diff x$.
\end{Exo}
 
 \begin{Exo}
 	Soit $f:\R_+\to \R$ continue et positive. On suppose que $\frac{f(x+1)}{f(x)}\underset{x\to+\infty}{\to}\ell<1$. Montrer que $\int_{ 0}^{+\infty}f$ converge.
 \end{Exo}
 

\begin{Exo}
	 Soit $f\in C\left( \R,\R\right) $ admettant des
limites finies en $\pm \infty .$ Montrer que $\displaystyle\int_{-\infty}^{+\infty}\left(
f\left( x+1\right) -f\left( x\right) \right) dx$ converge et donner sa
valeur.
\end{Exo}

 \begin{Exo}
 	Pour quels polyn\^{o}mes $P$ la fonction $\sqrt{P\left( t\right) }%
-t^{2}-t-1$ admet-elle une int\'{e}grale g\'{e}n\'{e}ralis\'{e}e $\R
_{+}$.
 \end{Exo}

\begin{Exo}
	Soit $f$ la fonction nulle en $0$ vérifiant pour $n>0: 
	\begin{cases}
			f\left( n\right) =n \\ 
			f\left( n-\frac{1}{2n^{3}}\right) =0 \\ 
			f\left( n+\frac{1}{2n^{3}}\right) =0
	\end{cases}$
	et affine entre les points $n-\frac{1}{2n^{3}},n,n+\frac{1}{2n^{3}}$ etc.
	Tracer $f$ et calculer $\int_{\mathbb{R}_{+}}f$.
\end{Exo}

\begin{Exo}[Intégrale de Dirichlet]
On souhaite établir la convergence et calculer $\displaystyle\int_0^{+\infty}\frac{\sin t}{t}\diff t$.
\begin{enumerate}
	\item
		Montrer que $\displaystyle\int_{0}^{\infty}\frac{\sin^2t}{t^2}\diff t$ converge, en déduire via une IPP que $\displaystyle\int_{0}^{\infty}\frac{\sin t}{t}\diff t$ converge et que les intégrales généralisées ont même valeur.
	\item
	Lemme (de Riemann-Lebesgue light): montrer que si $f$ est de classe $C^1$ sur $[a,b]$ alors:
	$$\int_a^b f(t)\sin(nt)\diff t\underset{n\to +\infty}{\to}0$$

	\item
	Montrer que $\phi$ définie sur $[0,\pi/2]$ par $\phi(0)=0$ et $\phi(t)=\frac{1}{\sin t}-\frac{1}{t}$ si $t\in]0,\pi/2]$ est $C^1$ sur $[0,\pi/2]$.
	\item
	On note pour $n\in\N:I_n=\int_{0}^{\pi/2}\frac{\sin(2n+1)t}{\sin t}\diff t$. Montrer que la suite $(I_n)_n$ est bien définie et est constante.
	\item
	On note pour $n\in\N:J_n=\int_{0}^{\pi/2}\frac{\sin(2n+1)t}{ t}\diff t$. Montrer à l'aide du lemme de Riemann-Lebesgue que la suite $(J_n)_n$ est bien définie et a même limite que $(I_n)_n$.
	\item En déduire la valeur de $\displaystyle\int_0^{+\infty}\frac{\sin t}{t}\diff t$ puis celle de $\displaystyle\int_0^{+\infty}\frac{\sin^2 t}{t^2}\diff t$.
\end{enumerate}
\end{Exo}

\end{document}
