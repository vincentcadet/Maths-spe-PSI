  
\documentclass[12pt,a4paper]{article}
% Engine-specific settings
% Detect pdftex/xetex/luatex, and load appropriate font packages.
% This is inspired by the approach in the iftex package.
% pdftex:
%\everymath{\displaystyle}
\usepackage[T1]{fontenc}
\usepackage[utf8]{inputenc}
\usepackage[french]{babel}
\frenchbsetup{StandardLists=true}
\usepackage{enumitem}
\usepackage{systeme}
\usepackage{amsmath,amssymb}
\usepackage[thmmarks,amsmath]{ntheorem}
\usepackage[colorlinks=true]{hyperref}
\usepackage{fullpage}
%\usepackage{eulervm}
\usepackage{graphicx}
\usepackage{array}
\usepackage{multicol}
\usepackage[makestderr]{pythontex}
\restartpythontexsession{\thesection}
\usepackage{geometry}
\geometry{tmargin=1.5cm,bmargin=1.5cm,lmargin=1.5cm,rmargin=1.5cm,headheight=12cm}
\usepackage{array}
\usepackage[svgnames,table]{xcolor}
\usepackage[tikz]{ bclogo}
\usepackage{pifont}
\usepackage{url}
\urlstyle{same}
\usepackage{pifont}
\usepackage{multicol}
\usepackage{diagbox} %oblique dans les tableaux
%\usepackage[framemethod=TikZ]{mdframed}
\usepackage{fancyhdr}
\pagestyle{fancyplain}
\setlength\headsep{2mm}

\lhead{\textit{CSI2B-PSI TD11-1}}
\chead{\textsc{Séries entières 1: rayon de convergence}}
\rhead{\textit{2021-2022}} 

\newcommand{\norme}[1]{\left\lVert#1\right\rVert}
\renewcommand*{\thefootnote}{\fnsymbol{footnote}}
\newcommand{\un}{(u_n)_n}
\newcommand{\R}{\mathbb{R}}
\newcommand{\C}{\mathbb{C}}
\newcommand{\Q}{\mathbb{Q}}
\newcommand{\Z}{\mathbb{Z}}
\newcommand{\N}{\mathbb{N}}
\newcommand{\E}{\mathrm{e}}
\newcommand{\K}{\mathbb{K} }
\newcommand{\diff}{\mathop{}\mathopen{}\mathrm{d}}%element differentiel
\newcommand{\diag}{\mathrm{diag}}
\renewcommand{\Re}{\mathcal{R}e}
\renewcommand{\Im}{\mathcal{I}m}
\newcommand{\abs}[1]{\left\lvert#1\right\rvert}
\DeclareMathOperator{\Ima }{Im}
\DeclareMathOperator{\vect}{Vect}
\DeclareMathOperator{\tr}{Trace}
\newcommand{\conj}[1]{\overline{#1}}

{%
	\theoremstyle{break}
	\theoremprework{%
		\rule{0.5\linewidth}{0.3pt}}
	\theorempostwork{\hfill%
		\rule{0.5\linewidth}{0.3pt}}
	\theoremheaderfont{\scshape}
	\theoremseparator{ ---}
	\newtheorem{Prop}{%
		\textcolor{blue}{Proposition}}[section]
}

{%
	\theoremstyle{break}
	\theoremprework{%
		\rule{0.6\linewidth}{0.5pt}}
	\theorempostwork{\hfill%
		\rule{0.6\linewidth}{0.5pt}}
	\theoremheaderfont{\scshape}
	\newtheorem{Theo}{%
		\textcolor{red}{Théorème}}[section]
}


{%
	\theoremheaderfont{\sffamily\bfseries}
	\theorembodyfont{\sffamily}
	\newtheorem{Def}{%
		\textcolor{green}{Définition}}[section]
}

{%
	\theorembodyfont{\small}
	\theoremsymbol{$\square$}
	\newtheorem*{Dem}{Démonstration}
}

{%
	\theorembodyfont{\small}
	\newtheorem*{Exemple}{Exemple}
}


{%
	\theorembodyfont{\small}
	\newtheorem{Exo}{Exercice}
}

{%
	\theoremnumbering{Roman}
	\theorembodyfont{\normalfont}
	\newtheorem{Rem}{Remarque}
}




%--------------------------------------------------------------------
%-------------DOCUMENT-----------------------------------------------
%--------------------------------------------------------------------


\begin{document}

\emph{\textbf{
	Sans musique la vie serait une erreur. (Nietzsche)
}}

\begin{bclogo}[couleur = green!30, arrondi = 0.1, logo = \bccoeur, barre = zigzag]{Pour mémoire}

Soit $\sum_n a_n z^n$ une série entière. On rappelle que:
$$I=\left\{r\geqslant 0,(a_nr^n)_n\text{ est bornée}\right\}$$ est un intervalle de $\R$ dont la borne supérieure $R_a$ dans $\R\cup\{+\infty\}$ est appelé rayon de convergence de la série entière.

On a ainsi $I=[0,R_a]$ ou $I=[0,R_a[$. Il importe de bien comprendre alors que:
$$\forall r \geqslant 0,r<R_a\implies (a_nr^n)_n\text{ bornée et }(a_nr^n)_n\text{ bornée }\implies r\leqslant R_a$$

On montre alors que:
\begin{enumerate}
	\item
	$\abs{z}<R_a\implies \sum_n a_n z^n$ CVA.
	\item
	$\abs{z}>R_a\implies \sum_n a_n z^n$ DVG car $(a_nz^n)_n$ non bornée.
\end{enumerate}
On appelle cercle d'incertitude le cercle de centre $0$ et de rayon $R_a$. C'est le seul endroit ou on peut trouver $z$ tel que $\sum_a a_n z^n$ soit finement divergente ou convergente sans l'être absolument!
\end{bclogo}
\vspace*{1cm}


\begin{Exo}
	Soit $a,b$ des réels positifs, montrer que:
	$\left(\forall r\in \R_+,r<a\implies r\leqslant b\right)\implies \left(a\leqslant b\right)$.
\end{Exo}

\begin{Exo}
	Soit $\lambda$ dans $\C^*$, montrer que $\sum_n a_nz^n$ et $\sum_n \lambda a_n z^n$ ont même rayon de convergence.
\end{Exo}
\newpage
\begin{Exo}
	Soit $\alpha$ un réel, montrer que $\sum_n a_n z^n$ et $\sum_n n^{\alpha}a_n z^n$ ont même rayon de convergence.
\end{Exo}

\begin{Exo}\ 
	\begin{enumerate}
		\item
		Donner un exemple de série entière qui diverge grossièrement en tout point du cercle d'incertitude.
		\item
		Donner un exemple de série entière qui converge absolument en tout point du cercle d'incertitude.
		\item
		Donner un exemple de série entière qui diverge finement en au moins un point du cercle d'incertitude et qui est semi-convergente en au moins un point du cercle d'incertitude.
		\item Donner un exemple de série entière de rayon de convergence $R>0$.
	\end{enumerate}
	
\end{Exo}


\begin{Exo}
	Etudier la convergence des séries entieres $\sum a_{n}z^{n}$ dans
	les cas suivants:
	\begin{multicols}{3}
		\begin{enumerate}
			\item $a_{n}=\frac{1}{n^{\alpha }}$ ($\alpha >0$)
			
			\item $a_{n}=\ln n$
			
			\item $a_{n}=\frac{n^{n}}{n!}$
			
			\item $a_{n}=\sin n$
			
			\item $a_{n}$ semi-convergente
			
			\item $\lim_{n}a_{n}=\ell\neq 0;$
			\item $a_n=\binom{2n}{n}$.
			\item
			$a_n=\E^{-n^2}$.
			\item
			$a_n=\frac{n^3+n^2\cos(n)-2n}{4n^4+n+1}$
			\item $a_n=n^{(-1)^n}$
			\item
			$a_n=\frac{1}{1+2+3+...+n}$
			\item $a_n=\frac{n+1}{n+2}$
		\end{enumerate}
	\end{multicols}
\end{Exo}



\begin{Exo}
	Rayon de convergence des séries suivantes:
	\begin{multicols}{4}
		\begin{enumerate}
			\item $\sum n!z^{n}$
			
			\item $\sum 2^{n}z^{n!}$
			
			\item $\sum (3+(-1)^{n})^n z^{n}$
			
			\item $\underset{p\text{ premier}}{\sum }z^{p}$
		\end{enumerate}
	\end{multicols}
\end{Exo}



\begin{Exo}
	On suppose que $\sum_n a_n z^n$ a un rayon de convergence $R>0$. Quel est le rayon de convergence de $\sum_n a_n z^{2n}$? De $\sum_n \frac{a_n}{n!}z^n$?
\end{Exo}



\begin{Exo}
	On rappelle que l'on note pour $n\in\N^*:H_n=\sum_{k=1}^{n}\frac{1}{k}$. 
	\begin{enumerate}
		\item
		Montrer que $S(x)=\sum_{n=1}^{+\infty}H_n x^n$ est définie sur $]-1,1[$. 
		\item
		En simplifiant $(1-x)S(x)$ donner une expression de $S(x)$. 
%		\item
%		Retrouver ce résultat via un produit de Cauchy.
	\end{enumerate}


\end{Exo}
% \begin{Exo}
% 	On note $u_0=0$ et pour $n\geqslant 0,u_{n+1}=\frac{u_n}{2}+(-1)^n$.
%\begin{enumerate}
%	\item
%	Montrer que $\lvert u_n\rvert\leqslant 2$, en déduire le RCV de $\sum_n u_n x^n$.
%	\item
%	On note $f(x)=\sum_{n=0}^{+\infty}u_n x^n$, simplifier $(1-x/2)f(x)$. En déduire $u_n$ pour tout $n$.
%\end{enumerate}
% \end{Exo}



	


	
	
	


	
\begin{Exo}
		Soit $\sum_n a_n z^n$ avec $R_a>0$, on note $b_n=\frac{a_n}{1+\lvert a_n\rvert}$. Montrer que $R_b\geqslant \max(1,R_a)$ puis qu'il y a égalité.	
\end{Exo}
	
	
%		  \begin{Exo}
%			On considère une série entiére $\sum a_{n}z^{n}$  et on pose $s_{n}=\sum_{p=0}^n a_{p}$. Vérifier que $R_s\leqslant R_a$ et que $R_s\geqslant \min(1,R_a)$. Que se passe t-il si $R_a\leqslant 1$ ?
%		\end{Exo}
%		
	
%	\item On considére une série entiére $\sum a_{n}z^{n}$ de rayon
%	de convergence $R>0$ et on pose $s_{n}=\sum a_{p}$ et on note $R^{\prime }$
%	le rayon de convergence de la série $\sum s_{n}z^{n}.$ On veut comparer $%
%	R$ et $R^{\prime }.$
%	
%	\begin{enumerate}
%		\item vérifier que $R^{\prime }\leqslant R$ et que $R^{\prime }\geqslant \inf
%		(1,R).$
%		
%		\item que se passe t-il si $R\leqslant 1$ ?
%		
%		\item On suppose que $R>1$
%		
%		\begin{enumerate}
%			\item vérifier que la suite $(s_{n})$ converge. On note $l$ sa limite.
%			
%			\item montrer que si $l\neq 0,$ $R^{\prime }=1.$
%			
%			\item montrer que si $l=0,$ pour tout $r\in \left] 1,R\right[ $ et pour $n$
%			assez grand, on a $\left\vert s_{n}\right\vert \leq \overset{\infty }{%
%				\underset{k=n+1}{\sum }}\frac{1}{r^{k}}.$ en déduire que $R^{\prime
%			}\geq R$ et conclure.
%			
%			\item résumer les différents cas possibles.
%		\end{enumerate}
%	\end{enumerate}
	
%	\item soit $\left[ a_{n}x^{n}\right] $ une série entiére de rayon de
%	convergence infini, de somme $S,$ vérifiant $\forall n\in \mathbb{N}%
%	,a_{n}>0.$ Montrer que 
%	\begin{equation*}
%		\forall p,S(x)\underset{+\infty }{\sim }\overset{\infty }{\underset{n=p+1}{%
%				\sum }}a_{n}x^{n}.
%	\end{equation*}
	
%	\item soit $f$ une fonction $C^{\infty }$ sur $\R$; on suppose que $f
%	$ ainsi que toutes ses dérivées sont positives. Soit $x>0.$
%	
%	\begin{enumerate}
%		\item justifier les trois égalités suivantes:
%		
%		\begin{enumerate}
%			\item $f(x)=f(0)+xf^{\prime }(0)+...+x^{n}\frac{f^{(n)}(0)}{n!}+x^{n+1}\frac{%
%				f^{(n+1)}(c_{n})}{(n+1)!}$
%			
%			\item $f(2x)=f(x)+xf^{\prime }(x)+...+x^{n}\frac{f^{(n)}(x)}{n!}+x^{n+1}%
%			\frac{f^{(n+1)}(d_{n})}{(n+1)!}$
%			
%			\item $f(-x)=f(0)-xf^{\prime }(0)+...+(-1)^{n}\frac{f^{(n)}(0)}{n!}%
%			+(-1)^{n+1}x^{n+1}\frac{f^{(n+1)}(h_{n})}{(n+1)!}$\newline
%			avec $c_{n}\in \left] 0,x\right[ ,$ $d_{n}\in \left] x,2x\right[ $ et $%
%			h_{n}\in \left] -x,0\right[ .$
%		\end{enumerate}
%		
%		\item montrer que la série $\Sigma \frac{f^{(n)}(x)}{n!}x^{n}$ est
%		convergente.
%		
%		\item montrer que la série $\Sigma \frac{f^{(n)}(0)}{n!}x^{n}$ est
%		convergente de somme $f(x),$ puis que la série $\Sigma \frac{f^{(n)}(0)}{%
%			n!}(-x)^{n}$ est convergente de somme $f(-x),$ et enfin que $f$ est DSE sur $%
%		\R.$
%	\end{enumerate}
	

	


\end{document}
