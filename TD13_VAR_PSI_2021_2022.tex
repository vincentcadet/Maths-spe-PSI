  
\documentclass[12pt,a4paper]{article}







%\usepackage{fullpage}
%\usepackage{lmodern}
\usepackage[utf8]{inputenc}
\usepackage{amsmath}
\usepackage{amssymb}
\usepackage[french]{babel}
\usepackage[T1]{fontenc}
\usepackage{multicol}
\usepackage{textcomp}
%\usepackage{tikz}
\usepackage{fancyhdr}
\usepackage{stmaryrd}
\addtolength{\hoffset}{-3cm}
 \addtolength{\voffset}{-2cm}
 \addtolength{\textwidth}{4cm}
\addtolength{\textheight}{4cm}


\pagestyle{fancyplain}
\lhead{\textit{CSI2B-PSI TD 13}}
\chead{\textsc{Variables aléatoires discrètes}}
\rhead{\textit{2021-2022}} 









%ensembles de nombres------------------------------------------------




\DeclareMathOperator{\Q}{\mathbb{Q}}
\DeclareMathOperator{\K}{\mathbb{K}}
\DeclareMathOperator{\R}{\mathbb{R}}
\DeclareMathOperator{\Z}{\mathbb{Z}}
\DeclareMathOperator{\N}{\mathbb{N}}
\DeclareMathOperator{\C}{\mathbb{C}}
\DeclareMathOperator{\U}{\mathbb{U}}




%applications relations----------------------------------------------




\DeclareMathOperator{\Id}{\mathrm{Id}}
\DeclareMathOperator{\Class}{\mathcal{C}}
\DeclareMathOperator{\DL}{\mathrm{DL}}

\newcommand{\lci}{l.c.i. }
\newcommand{\set}[2]{
\left\{
\begin{array}{ccc}
#1 & / & #2
\end{array}
\right\}
}
\newcommand{\appli}[5]
{
%E,F,f,x,y
\left\{
\begin{array}{ccc}
#1 & \overset{#3}{\longrightarrow} & #2 \\
#4 & \longmapsto & #5 
\end{array}
\right.
}
\newcommand{\applilight}[3]
{
%E,F,f
#3 : #1  \longrightarrow  #2
}
\newcommand{\reciproque}[1]{
#1^{*} 
}





%fonctions usuelles--------------------------------------------------




\DeclareMathOperator{\ch}{\mathrm{ch}}
\DeclareMathOperator{\sh}{\mathrm{sh}}
\DeclareMathOperator{\tah}{\mathrm{th}}
\DeclareMathOperator{\sgn}{\mathrm{sgn}}
\DeclareMathOperator{\cotan}{\mathrm{cotan}}

\newcommand{\floor}[1]{\left\lfloor #1 \right\rfloor}
\newcommand{\equi}[3]
{
% f,g,a
#1\underset{#3}{\sim}#2
}
\newcommand{\bigo}[3]
{
% f,g,a
#1\underset{#3}{=}\mathcal{O}\left(#2\right)
}
\newcommand{\plusbigo}[4]
{
% f,g,h,a : f =_a g+O(h)
#1\underset{#4}{=}#2 + \mathcal{O}\left(#3\right)
}
\newcommand{\smallo}[3]
{
% f,g,a
#1\underset{#3}{=}\mathbf{o}\left(#2\right)
}
\newcommand{\plussmallo}[4]
{
% f,g,h,a : f =_a g+o(h)
#1\underset{#4}{=}#2 + \mathbf{o}\left(#3\right)
}



	\newtheorem{Exo}{Exercice}




%complexes--------------------------------------------





\DeclareMathOperator{\Real}{\mathcal{R}e}
\DeclareMathOperator{\Imaginary}{\mathcal{I}m}
\DeclareMathOperator{\I}{\mathbf{i}}





%polynomes----------------------------------------------------





\DeclareMathOperator{\Xd}{\mathrm{X}}
\DeclareMathOperator{\val}{\mathrm{val}}
\DeclareMathOperator{\X}{\mathrm{X}}
%\DeclareMathOperator{\deg}{\mathrm{deg}}





%algebre lineaire---------------------------------------------





\DeclareMathOperator{\Diag}{\mathrm{Diag}}
\DeclareMathOperator{\Vect}{\mathrm{Vect}}
\DeclareMathOperator{\Ima}{\mathrm{Im}}
\DeclareMathOperator{\Ker}{\mathrm{Ker}}
\DeclareMathOperator{\GL}{\mathrm{GL}}
\DeclareMathOperator{\tr}{\mathrm{tr}}
\DeclareMathOperator{\Lcal}{\mathcal{L}}
\DeclareMathOperator{\Mat}{\mathrm{Mat}}
\DeclareMathOperator{\Com}{\mathrm{Com}}
\DeclareMathOperator{\M}{\mathcal{M}}
\DeclareMathOperator{\rg}{\mathrm{rg}}
\newcommand{\tribu}{\mathcal{A}}
\renewcommand{\Pr}{\mathbb{P}} 
\newcommand{\ev}{espace vectoriel }
\newcommand{\evs}{espaces vectoriels }
\newcommand{\transpo}[1]{
 \mathstrut^t\! #1
}





%arithmetique-------------------------------------------------





\DeclareMathOperator{\pgcd}{pgcd}

\newcommand{\modulobis}[3]{
% x,y,alpha
#1\equiv #2\; \left(\text{mod}\;#3\right)
}
\newcommand{\modulo}[3]{
% x,y,alpha
#1\equiv #2\; \left[#3\right]
}





%calcul differentiel et integral----------------------------




\DeclareMathOperator{\der}{\mathrm{d}}




%denombrement probas-----------------------------------------




\DeclareMathOperator{\Card}{\mathrm{Card}}
\DeclareMathOperator{\PR}{\mathbf{P}}
\DeclareMathOperator{\E}{\mathbf{E}}
\DeclareMathOperator{\V}{\mathbf{V}}
\DeclareMathOperator{\COV}{\mathbf{Cov}}

\newcommand{\VA}{v.a. }
\newcommand{\VAR}{v.a.r. }
\newcommand{\bin}[2]{\binom{#1}{#2}}




%geometrie---------------------------------------------------




\newcommand{\fle}[1]{
\overrightarrow{#1}
}
\newcommand{\ang}[2]{
\widehat{
\left(
#1 , #2
\right)
}
}
\newcommand{\scal}[2]{
\left( #1 |  #2 \right)
}




%recurrence--------------------------------------------------




\newcommand{\recurr}[4]{
% u, n_0, u(n_0), u(n+1)
\left\{
\begin{array}{cl}
& #1_{#2}=#3 \\
\forall n\geqslant #2 \quad &    #1_{n+1}=#4   
\end{array}
\right.
}
\newcommand{\recurrdouble}[6]{
% u, n_0,n_1, u(n_0), u(n_1), u(n+2)
\left\{
\begin{array}{cl}
& #1_{#2}=#4 \qquad #1_{#3}=#5\\
\forall n\geqslant #2 \quad &    #1_{n+2}=#6   
\end{array}
\right.
}




%divers----------------------------------------------------




\DeclareMathOperator{\lcro}{\textnormal{\textlbrackdbl}}
\DeclareMathOperator{\rcro}{\textnormal{\textrbrackdbl}}

\newcommand{\ssi}{ssi }
\newcommand{\dis}{\displaystyle}


%--------------------------------------------------------







%--------------------------------------------------------------------
%-------------DOCUMENT-----------------------------------------------
%--------------------------------------------------------------------


\begin{document}

\emph{\textbf{
Les espoirs des hommes instruits valent mieux que la richesse des ignorants.
(Démocrite)
}}


%\item
%Une personne réalise une infinité de lancers indépendants d’une pièce truquée  dont la probabilité de faire pile vaut $p\in]0,1[$.
%\begin{enumerate}
%	\item
%	Quelle est la probabilité que le premier pile apparaisse lors des deux premiers lancers?
%	\item
%	Quelle est la probabilité que le premier pile apparaissent lors d’un numéro de lancer pair?
%\end{enumerate}


\begin{Exo}[problème des rencontres]
	$n$ individus déposent leur portefeuille dans une urne. Puis ils piochent chacun leur tour un portefeuille. On note $X$ la variable aléatoire égale au nombre d'individus ayant récupéré leur portefeuille. Déterminer $\mathbb{E}(X)$. 
\end{Exo}

\begin{Exo}
	$n$ personnes montent dans un ascenseur desservant $p$ étages.  Chaque personne choisit un étage, indépendamment des autres usagers, avec probabilité uniforme. Quel est l'espérance de $X$ le nombre d'arrêts? On pourra calculer la loi de la v.a $X_i$ valant $1$ si l'ascenseur s'arrête à l'étage $i$, $0$ sinon.
\end{Exo}

\begin{Exo}[collectionneur de vignettes]
Chaque paquet de céréales contient une vignette à collectionner, que l'on ne découvre qu'à l'ouverture du paquet. On se demande combien il faut ouvrir de
paquets pour posséder au moins un exemplaire de chacune des $n$ vignettes.
On décompose ce nombre en $N=N_1+...+N_n$ où $N_k$ est le nombre de paquets supplémentaires nécessaires pour obtenir $k$ vignettes différentes quand on en a déjà $k-1$ différentes. Déterminer la loi de $N_k$ puis $\mathbb{E}(N)$.
\end{Exo} 

\begin{Exo}
	On suppose que le nombre $N$ de clients allant en caisse dans une période de temps donnée suit une loi de Poisson de paramètre $\lambda>0$. Le magasin dispose de deux caisses, et chaque client se dirige au hasard et indépendamment des autres clients :
\begin{itemize}
	\item vers la caisse $1$ avec probabilité $p\in[0,1]$
	\item vers la caisse $2$ avec probabilité $1-p$
\end{itemize}

On note $N_1$ et $N_2$ le nombre de clients se présentant respectivement à la caisse $1$ et à la caisse $2$. Soit $n\in\N$. Quelle est la loi conditionnelle de $N_1$ sachant l’événement $\{N=n\}$? En déduire la loi de $N_1$.
\end{Exo}



\begin{Exo}
	Soient $X_1,X_2$ deux variables aléatoires indépendantes suivant des  lois géométriques de paramètre respectif $p_1,p_2$ dans $]0,1[$. Quelle est la loi de la variable aléatoire $X=\min(X_1,X_2)$? On pourra calculer $\Pr(X\geqslant n)$ pour $n\in\N^*$.
\end{Exo}


\begin{Exo}
	 Calculer l’espérance de $\frac{1}{1+X}$ quand 	$X\hookrightarrow \mathcal{B}(n,p)$ puis quand $X\hookrightarrow \mathcal{P}(\lambda)$
\end{Exo}
 


\begin{Exo}
	 Soit $a\in\R$ et $X$ v.a.d à valeurs dans $N^*$ dont la loi
est définie par:
\[\forall n\in\N^*,\Pr(X=n)=\frac{a}{n(n+1)}\]
Déterminer $a$ puis, si elle existe, l'espérance de $X$.
\end{Exo}


\begin{Exo}
	 On lance une pièce équilibrée jusqu’à ce que celle-ci ait produit pour
la première fois la séquence «pile - face». On note $T$ le nombre de lancer avant que le jeu s’arrête. Donner la loi de $T$, montrer que $T$ est presque surement fini et calculer son espérance.

\end{Exo}

\begin{Exo}
	Le nombre $N$ de voitures passant devant une station essence suit une loi de Poisson de paramètre $\lambda>0$. Chaque voiture décide de s'arrêter avec une probabilité $0<p<1$, indépendamment les unes des autres. Quelle est l'espérance de la v.a.d $X$ donnant le nombre de voitures s'arrêtant à la station?
\end{Exo}


\begin{Exo}
	Soit $X$ une variable aléatoire discrète à valeurs dans $\N$ . On suppose que $X$ suit une loi sans mémoire, c’est à dire:
$$\forall (n,m)\in \N^2,\Pr(X>m)>0,\Pr\left(X>m+n|X>m\right)=\Pr(X>n)$$
\begin{enumerate}
	\item Montrer qu’une loi géométrique est sans mémoire.
	
	Dans la suite on suppose $X$ sans mémoire et on note $p=\Pr(X=1)$.

	\item Montrer que pour tout $n$ dans $\N$ on a $\Pr(X>n+1)=\Pr(X>1)\Pr(X>n)$
	\item En déduire que $X$ suit une loi géométrique.
\end{enumerate}
\end{Exo}


\begin{Exo}
		Soit $X$ une v.a.d sur 	$\left(\Omega,\tribu,\Pr\right)$ telle que $X(\Omega)=\N$.
Montrer que: $$\forall n\in\N^*,\sum_{k=0}^nk\Pr(X=k)=\sum_{k=0}^{n-1}\Pr(X>k)-n\Pr(X>n)$$
 En déduire que  $X$ admet une espérance ssi $\sum_n\Pr(X>n)$ converge et qu'alors $$\mathbb{E}(X)=\sum_{k=1}^{+\infty}\Pr(X\geqslant k)=\sum_{k=0}^{+\infty}\Pr(X>k)$$
\end{Exo}

\begin{Exo}
	Soient $X,Y$ v.a.d indépendantes suivant la loi géométrique de paramètre $0<p<1$ et $M=\max(X,Y)$.
\begin{enumerate}
	\item
	Calculer pour $n\in\N^*:\Pr(X\leqslant n)$ puis $\Pr(M\leqslant n)$.
	\item
	Déterminer la loi de $M$ et l'espérance de $M$. On pourra utiliser l'exercice précédent.
\end{enumerate}
\end{Exo}

%\item
%Soient $X,Y,Z$ trois variables aléatoires indépendantes, suivant la même loi binomiale B$\mathcal{B}(n,p)$. Pour tout $\omega\in\Omega$ on pose $A(\omega)=\begin{pmatrix}
%X(\omega)	& Y(\omega) & Z(\omega) \\
%	X(\omega)& Y(\omega) & Z(\omega) \\
%	X(\omega)& Y(\omega) & Z(\omega)
%\end{pmatrix}$
%\begin{enumerate}
%	\item Quelle est la probabilité qu'elle soit diagonalisable ?
%	\item Quelle est la probabilité qu'elle soit une matrice de projecteur ?	
%\end{enumerate}

%\item
%Soient $X,Y$ indépendantes suivant la même loi $\mathcal{G}(p)$ avec $0<p<1$. quelle est la probabilité que $A=\begin{pmatrix}
%X_1	& 1 \\
%0	& X_2
%\end{pmatrix}$ soit inversible? Diagonalisable?




\begin{Exo}
	Soient $X,Y$ v.a.d à valeurs dans $\N$ telles que:
$$\forall(i,j)\in\N^2,\Pr(X=i,Y=j)=\frac{i+j}{i!j!\mathrm{e}}\left(\frac{1}{2^{i+j}}\right)$$
\begin{enumerate}
	\item
	Déterminer la loi de $X$ et de $Y$. $X$ et $Y$ sont-elles indépendantes?
	\item
	Montrer que $2^{X+Y}$ admet une espérance et calculer celle-ci.
\end{enumerate}
\end{Exo}


 \begin{Exo}
 	On suppose qu’une colonie d’insectes produit $N$ oeufs où $N\hookrightarrow\mathcal{P}(\lambda)$. Pour tout $n\in\N$, lorsque $N=n$, le nombre $X$ d’oeufs qui éclosent suit une loi binomiale $\mathcal{B}(n,p)$. Déterminer la loi de $X$.
 \end{Exo}
 
 
 
\begin{Exo}
	On dispose d’une pièce ayant la probabilité $0<p<1$ de donner pile.
On  réalise l’expérience suivante : on lance la pièce autant de fois que nécessaire pour obtenir pile. On note $N$ le nombre de lancers effectués. Puis on relance à nouveau la pièce $N$ fois et on compte le nombre $X$ de piles obtenus. Déterminer la loi de $N$ et de $X$, montrer que $X$ admet une espérance et calculer celle-ci.
\end{Exo}


\begin{Exo}
	Soient $X,Y$ deux v.a.d indépendantes. À l’aide des fonctions génératrices, montrer que :
\begin{enumerate}
	\item
	Si $X\hookrightarrow \mathcal{B}(n,p)$ et $Y\hookrightarrow \mathcal{B}(m,p)$ alors  $X+Y\hookrightarrow \mathcal{B}(m+n,p)$.
	\item
	Si $X\hookrightarrow \mathcal{P}(\lambda)$ et $Y\hookrightarrow \mathcal{P}(\mu)$ alors  $X+Y\hookrightarrow \mathcal{P}(\lambda+\mu)$.
\end{enumerate}
\end{Exo}


\begin{Exo}
	On joue à pile ou face avec une pièce équilibrée.  Déterminer un entier $p$ tel
que pour tout $n\geqslant p$ la probabilité d'avoir entre la moitié et trois quart de piles soit au moins $0,95$. (Utiliser Bienaymé-Tchebytchev)
\end{Exo}



\begin{Exo}
	Montrer que l'on ne peut pas truquer deux dés de sorte que la somme de leur résultat suive une loi uniforme.
\end{Exo}


\begin{Exo}
	Soit $X$ v.a à valeurs dans $\N$ de fonction génératrice $G:t\mapsto a\mathrm{e}^{1+t^2}$.
\begin{enumerate}
	\item
	Déterminer $a$.
	\item
	Déterminer la loi de $X$.
	\item Justifier que $X$ admet une espérance et une variance, les déterminer.
\end{enumerate}
\end{Exo}

\begin{Exo}[banque CCP MP]
	On admet, dans cet exercice, que:
	$\forall\:q\in \mathbb{N}$, $\displaystyle\sum\limits_{k\geqslant q}^{}\dbinom {k}{q}x^{k-q}$ converge et  $\forall \:x\in \left] -1,1\right[$,   $\displaystyle\sum\limits_{k=q}^{+\infty}\dbinom{k}{q}x^{k-q}=\dfrac{1}{(1-x)^{q+1}}$.
	Soit $p\in \left] 0,1\right[ $, $(\Omega,\mathcal{A},P)$ un espace probabilisé et $X$ et $Y$ deux variables aléatoires définies sur  $(\Omega,\mathcal{A},P)$ et à valeurs dans $\mathbb{N}$.\\
	On suppose que la loi de probabilité du  couple $(X,Y)$ est donnée par:\\
	\medskip
	$$\forall\:(k,n)\in\mathbb{N}^2:P((X=k)\cap (Y=n))=
	\begin{cases}
\dbinom{n}{k}\left( \dfrac{1}{2}\right)^np(1-p)^n\:\text{si}\:k\leqslant n\\
	0\:\text{sinon} 
	\end{cases}$$
	\begin{enumerate}
		\item
		Vérifier qu'il s'agit bien d'une loi de probabilité.
		\item
		\begin{enumerate}
			\item
			Déterminer la loi de $Y$.
			\item
			Prouver que $1+Y$ suit une loi géométrique.
			\item
			Déterminer l'espérance de $Y$.
		\end{enumerate}
		\item
		Déterminer la loi de $X$.
		
		
	\end{enumerate}
\end{Exo}


\begin{Exo}[banque CCP MP]
	Soit $(\Omega,\mathcal{A},P)$ un espace probabilisé.\\
	
	\begin{enumerate}
		\item 
		Soit $X$ une variable aléatoire définie sur  $(\Omega,\mathcal{A},P)$ et à valeurs dans $\mathbb{N}$.\\
		On considère la série entière  $\displaystyle\sum t^nP(X=n)$ de variable réelle $t$.\\
		On note $R_X$ son rayon de convergence.\\
		\begin{enumerate}
			\item 
			Prouver que $R_X\geqslant 1$.\:\:\:\:\\
			On pose  $G_X(t)=\displaystyle\sum\limits_ {n=0}^{+\infty}t^nP(X=n)$ et on  note $D_{G_X}$ l'ensemble de définition de $G_X$.\\
			Justifier que   $\left[-1,1 \right] \subset D_{G_X}$.\:\:\:\:\\
			
			Pour tout réel $t$ fixé de $\left[-1,1 \right]$, exprimer $G_X(t)$ sous forme d'une espérance.\:\:\:\:
			\item
			Soit $k\in\mathbb{N}$.
			Exprimer, en justifiant la réponse, $P(X=k)$ en fonction de  $G_X^{(k)}(0)$.\:\:\:\:
			
		\end{enumerate}
		\item 
		\begin{enumerate}
			\item
			On suppose que $X$ suit une loi de Poisson de paramètre $\lambda$.\\
			Déterminer $D_{G_X}$ et, pour tout $\:t\in D_{G_X}$, calculer $G_X(t)$.\:\:\:\:
			
			
			\item
			Soit $X$ et $Y$ deux variables aléatoires définies sur un même espace probabilisé, indépendantes  et suivant des lois de Poisson de paramètres respectifs $\lambda_1$ et $\lambda_2$.\\
			Déterminer, en utilisant les questions précédentes, la loi de $X+Y$.
		\end{enumerate}
	\end{enumerate}
\end{Exo}

\begin{Exo}[banque CCP MP]
	Une secrétaire effectue, une première fois, un appel téléphonique vers $n$ correspondants distincts.\\
On admet que les $n$ appels constituent $n$ expériences indépendantes et que, pour chaque appel, la probabilité d'obtenir le correspondant demandé est $p$ ($p\in{\left]  0,1\right[ }$).\\
Soit $X$ la variable aléatoire représentant le nombre de correspondants obtenus.
\begin{enumerate}
	\item Donner la loi de $X$. Justifier.
	\item
	La secrétaire rappelle une seconde fois, dans les mêmes conditions, chacun des $n-X$ correspondants qu'elle n'a pas pu joindre au cours de la première série d'appels.
	On note $Y$ la variable aléatoire représentant le nombre de personnes jointes au cours de la seconde série d'appels.
	\begin{enumerate}
		\item
		Soit $i\in \llbracket 0,n \rrbracket $.
		Déterminer, pour  $k\in \mathbb{N}, $ $P(Y=k|X=i)$.
		\item
		Prouver que $Z=X+Y$ suit une loi binomiale dont on déterminera le paramètre.\\
		\textbf{Indication} : on pourra utiliser, sans la prouver, l'égalité suivante: $\dbinom{n-i}{k-i}\dbinom{n}{i}=\dbinom{k}{i}\dbinom{n}{k}$.\\
		\item
		Déterminer l'espérance et la variance de $Z$.
	\end{enumerate}
\end{enumerate}

\end{Exo}


\begin{Exo}[banque CCP MP]\ 
	\begin{enumerate}
	\item
	Rappeler l'inégalité de Bienaymé-Tchebychev.
	\item
	Soit $(Y_n)$ une suite de variables aléatoires mutuellement indépendantes, de même loi et admettant un moment d'ordre 2. On pose $S_n=\displaystyle\sum\limits_{k=1}^{n}Y_k$.\\
	
	Prouver que: $\forall\:a\in \left] 0,+\infty\right[ $, $P\left( \left|\dfrac{S_n}{n}-E(Y_1)\right|\geqslant a\right) \leqslant\dfrac{V(Y_1)}{na^2}$.
	\item \textbf{Application}\\
	On effectue des tirages successifs, avec remise, d'une boule dans une urne contenant 2 boules rouges et 3 boules noires.\\
	\`A partir de quel nombre de tirages peut-on garantir à plus de 95\% que la proportion de boules rouges obtenues restera comprise entre $0,35$ et $0,45$?\\
	\textbf{Indication} : considérer la suite $(Y_i)$ de variables aléatoires de Bernoulli où $Y_i$ mesure l'issue du $i^{\text{ème}}$  tirage.
	
\end{enumerate}
\end{Exo}




\begin{Exo}[banque CCP MP]
	Soit $\lambda \in{\left] 0,+\infty\right[ }$ et $X$ une variable aléatoire discrète à valeurs dans $\mathbb{N}^\ast$. On suppose que $\forall n\in\mathbb{N}^\ast$, $P(X=n)=\dfrac{\lambda}{n(n+1)(n+2)} $.
\begin{enumerate}
	\item Décomposer en éléments simples la fraction rationnelle $R$ définie par $R(x)=\dfrac{1}{x(x+1)(x+2)}$.
	\item
	Calculer $\lambda$.
	\item
	Prouver que $X$ admet une espérance, puis la calculer.
	\item
	$X$ admet-elle une variance? Justifier.
	
\end{enumerate}
\end{Exo}

\end{document}
