  
\documentclass[12pt,a4paper]{article}
% Engine-specific settings
% Detect pdftex/xetex/luatex, and load appropriate font packages.
% This is inspired by the approach in the iftex package.
% pdftex:

\usepackage[T1]{fontenc}
\usepackage[utf8]{inputenc}
\usepackage[french]{babel}
\frenchbsetup{StandardLists=true}
\usepackage{enumitem}
\usepackage{systeme}
\usepackage{amsmath,amssymb}
\usepackage[thmmarks,amsmath]{ntheorem}
\usepackage[colorlinks=true]{hyperref}
\usepackage{fullpage}
%\usepackage{eulervm}
\usepackage{graphicx}
\usepackage{array}
\usepackage{multicol}
\usepackage[makestderr]{pythontex}
\restartpythontexsession{\thesection}
\usepackage{geometry}
\geometry{tmargin=1.5cm,bmargin=1.5cm,lmargin=1.5cm,rmargin=1.5cm,headheight=12cm}
\usepackage{array}
\usepackage[svgnames,table]{xcolor}
\usepackage[tikz]{ bclogo}
\usepackage{pifont}
\usepackage{url}
\urlstyle{same}
\usepackage{pifont}
\usepackage{multicol}
\usepackage{diagbox} %oblique dans les tableaux
%\usepackage[framemethod=TikZ]{mdframed}
\usepackage{fancyhdr}
\pagestyle{fancyplain}
\setlength\headsep{2mm}

\lhead{\textit{CSI2B-PSI TD13}}
\chead{\textsc{Réduction}}
\rhead{\textit{2020-2021}} 


\renewcommand*{\thefootnote}{\fnsymbol{footnote}}
\newcommand{\un}{(u_n)_n}
\newcommand{\R}{\mathbb{R}}
\newcommand{\C}{\mathbb{C}}
\newcommand{\Q}{\mathbb{Q}}
\newcommand{\Z}{\mathbb{Z}}
\newcommand{\N}{\mathbb{N}}
\newcommand{\K}{\mathbb{K} }

\newcommand{\diag}{\mathrm{diag}}
\renewcommand{\Re}{\mathcal{R}e}
\renewcommand{\Im}{\mathcal{I}m}
\DeclareMathOperator{\Ima }{Im}
\DeclareMathOperator{\vect}{Vect}
\DeclareMathOperator{\tr}{Trace}
\newcommand{\conj}[1]{\overline{#1}}

{%
\theoremstyle{break}
\theoremprework{%
\rule{0.5\linewidth}{0.3pt}}
\theorempostwork{\hfill%
\rule{0.5\linewidth}{0.3pt}}
\theoremheaderfont{\scshape}
\theoremseparator{ ---}
\newtheorem{Prop}{%
\textcolor{blue}{Proposition}}[section]
}

{%
\theoremstyle{break}
\theoremprework{%
\rule{0.6\linewidth}{0.5pt}}
\theorempostwork{\hfill%
\rule{0.6\linewidth}{0.5pt}}
\theoremheaderfont{\scshape}
\newtheorem{Theo}{%
\textcolor{red}{Théorème}}[section]
}


{%
\theoremheaderfont{\sffamily\bfseries}
\theorembodyfont{\sffamily}
\newtheorem{Def}{%
\textcolor{green}{Définition}}[section]
}

{%
\theorembodyfont{\small}
\theoremsymbol{$\square$}
\newtheorem*{Dem}{Démonstration}
}

{%
\theorembodyfont{\small}
\newtheorem*{Exemple}{Exemple}
}


{%
\theorembodyfont{\small}
\newtheorem{Exo}{Exercice}
}

{%
\theoremnumbering{Roman}
\theorembodyfont{\normalfont}
\newtheorem{Rem}{Remarque}
}




\begin{document}

\emph{\textbf{“La peur de l'effort est plus épuisante que l'effort même. ” (Albert Brie)}}



\begin{Exo}
Soit $u$ un endomorphisme d'un espace vectoriel $E$. Montrer que si $0$ est valeur propre de $u$ alors $u$ n'est pas inversible. La réciproque est-elle toujours vraie?
\end{Exo}


\begin{Exo}
	La somme de deux vecteurs propres est-elle encore un vecteur propre?
\end{Exo}



\begin{Exo}
	Soit $D=\diag(x_1,x_2,...,x_n)$ avec les $x_i$ deux à deux distincts.
	\begin{enumerate}
		\item
		Montrer que $(I_n,D,D^2,...,D^{n-1})$ est une base du sev $\mathcal{D}_n(\K)$ des matrices diagonales de $M_n(\K)$.
		\item
		Quel est le degré minimal d'un polynôme annulateur non nul de $D$?
	\end{enumerate}	
\end{Exo}

%\begin{Exo}[matrices en damier]
%	On dit que $M\in M_n(\K)$ est en damier si:
%	$$\forall (i,j)\in \{1,2,...,n\},i+j\text{ impair }\implies m_{i,j}=0$$
%	On note $Dam$ leur ensemble.
%	
%	\vspace*{2mm}
%	NB: aucune honte à traiter d'abord l'exercice en petite dimension ($n=3,4$) pour mieux appréhender la notion.
%	\begin{enumerate}
%		\item
%		Montrer que $Dam$ est un sev de $M_n(\K)$ dont on donnera une base.
%		\item Montrer que $Dam$ est stable par produit.
%		\item  \textbf{On admet à ce stade du cours que $A^{-1}$ est un polynôme en $A$}. Montrer que si $A$ est en damier et inversible alors $A^{-1}$ est encore en damier.
%	\end{enumerate}
%\end{Exo}

\begin{Exo}
	Trouver un polynôme annulateur non nul de degré minimal pour $A=\begin{pmatrix}
	-2 & 2 & -1 \\
	-1 & 1 & -1 \\
	-1 & 2 & -2
	\end{pmatrix}$
\end{Exo}

%\begin{Exo}
%	Montrer que si un sev $F$ de $E$ est stable par $u\in L(E)$ alors il est stable par $u^{-1}$. Rejustifier le fait que l'inverse d'une matrice en damier est encore en damier.
%\end{Exo}


\begin{Exo}
	Soit $E=\mathbb C^\mathbb N$ l'espace des suites à coefficients complexes, et $\phi$ l'endomorphisme de $E$ qui à une suite $(u_n)$ associe la suite $(v_n)$ définie par $v_0=u_0$ et pour tout $n\geq 1$, 
	$v_n=\frac{u_n+u_{n-1}}2$.	Déterminer les valeurs propres et les vecteurs propres de $\phi$.
\end{Exo}

\begin{Exo}
	Soit $(P,u,x,\lambda)\in \K[X]\times L(E)\times E^*\times \K$. On suppose que $u(x)=\lambda x$, montrer que $P(\lambda)$ est valeur propre de $P(u)$.
\end{Exo}

\begin{Exo}
	Montrer qu'un vecteur propre de $u$ associé à une valeur propre non nulle est dans $\Ima(u)$
\end{Exo}

\begin{Exo}
	Déterminer les valeurs propres et les sevs propres de l'endomorphisme de dérivation $D$ sur $\K_n[X]$ avec $n\geqslant 0$. Idem sur $E=C^{\infty}(\R,\R)$ ev des fonctions indéfiniment dérivables sur $\R$.
\end{Exo}

\begin{Exo}
	Quels sont les sevs de $\mathbb{K}\left[ X\right] $ stables par la dérivation?
\end{Exo}


\begin{Exo}[En dimension finie]
	Monter que si $u$ admet un hyperplan stable	alors il admet une valeur propre (raisonner matriciellement). Montrer aussi qu'un hyperplan est stable par $u$ ssi il existe une valeur propre $\lambda$ de $u$ telle que $\Ima (u-\lambda id_E)\subset H$.
\end{Exo}


\begin{Exo}
		
	 D\'{e}terminer les sevs de $\R^3$ stables par $\left( 
	\begin{array}{lll}
		1 & 1 & 1 \\ 
		0 & 0 & 1 \\ 
		0 & 0 & 1%
	\end{array}%
	\right)$ et par $\begin{pmatrix}
		0 & 1 & 1 \\
		1 & 0 & 0 \\
		0 & 0 & 1
	\end{pmatrix}$.
	
On pourra raisonner sur la dimension d'un tel sev stable.
\end{Exo}

\begin{Exo}
	Diagonaliser :
	$A=\left(\begin{array}{ccc}
		0&2&-1\\
		3&-2&0\\
		-2&2&1
	\end{array}\right),\textrm{ } B=\left(\begin{array}{ccc}
		0&3&2\\
		-2&5&2\\
		2&-3&0
	\end{array}\right), C=\left(\begin{array}{ccc}
		1&0&0\\
		0&1&0\\
		1&-1&2
	\end{array}\right).$
\end{Exo}

\begin{Exo}
	Les matrices 
	$A=\left(\begin{array}{ccc}
		0&0&4\\
		1&0&-8\\
		0&1&5
	\end{array}\right)\textrm{ et }
	B=\left(\begin{array}{ccc}
		2&1&1\\
		0&0&-2\\
		0&1&3
	\end{array}\right)$
	sont-elles semblables?
\end{Exo}

\begin{Exo}
	Montrer que les applications suivantes sont des endomorphismes de $
\R_{n}\left[ X\right] $, pr\'{e}ciser leurs \'{e}l\'{e}ments propres et
dire s'ils sont diagonalisables:
\begin{multicols}{2}
\begin{enumerate}
\item  $\phi \left( P\right) =P\left( X-1\right) .$

\item  $\phi \left( P\right) =\left( X^{2}-1\right) P^{\prime }\left(
X\right) -2nXP$

\item  $\phi \left( P\right) =\left( nX+1\right) P+\left( 1-X^{2}\right)
P^{\prime }$

\item  $\phi \left( P\right) =X\left( X+1\right) P^{\prime }-2nXP.$
\end{enumerate}
\end{multicols}
\end{Exo}

\begin{Exo}
	Soit $A\in M_n(\R)$ diagonalisable telle que $A^{2022}=I_n$, montrer que $A^2=I_n$.
\end{Exo}

\begin{Exo}
	Montrer que $A\in M_n(\K)$ est diagonalisable ssi $A+I_n$ l'est.
\end{Exo}

\begin{Exo}
	pour quelles valeurs des param\'{e}%
	tres r\'{e}els les matrices suivantes sont elles diagonalisables ?
	
	$a)\left( 
	\begin{array}{ccc}
		1-a & a & 0 \\ 
		-a & 1 & a \\ 
		-a & 1-b & a+b%
	\end{array}%
	\right) ,b)\left( 
	\begin{array}{ccc}
		1 & a & 1 \\ 
		0 & 1 & b \\ 
		0 & 0 & c%
	\end{array}%
	\right) ,c)\left( 
	\begin{array}{llll}
		1 & a & b & c \\ 
		0 & 1 & d & e \\ 
		0 & 0 & -1 & f \\ 
		0 & 0 & 0 & -1%
	\end{array}%
	\right) ,d)\left( 
	\begin{array}{llll}
		1 & a & b & c \\ 
		0 & 1 & d & e \\ 
		0 & 0 & 2 & f \\ 
		0 & 0 & 0 & 2%
	\end{array}%
	\right)$
\end{Exo}

\begin{Exo}
	Expliquer sans calcul pourquoi $A=\begin{pmatrix}
		2 & 1 & 7 \\
		0 & 2 & 13 \\
		0 & 0 & 2
	\end{pmatrix}$ n'est pas diagonalisable.
	
\end{Exo}

\begin{Exo}
	Montrer de deux manières différentes que si $A$ est diagonalisable alors sa transposée aussi.
\end{Exo}

\begin{Exo}
	Soit $A\in M_3(\R)$ telle que $A^4=A^2$. Trouver un polynôme annulateur pour $A$. On suppose de plus que $\{-1,1\}\subset sp(A)$. Montrer alors que $A$ est diagonalisable.
\end{Exo}

\begin{Exo}[Mines-Ponts]
	À quelle condition sur $z, M(z)=\left(\begin{array}{lll}0 & 0 & z \\ 1 & 1 & 0 \\ 0 & 1 & 0\end{array}\right)$ est-elle diagonalisable dans $\mathbb{C} ?$
\end{Exo}

\begin{Exo}
	Soit $A=\left( 
\begin{array}{ll}
	5 & 3 \\ 
	1 & 3%
\end{array}%
\right)$.On veut r\'{e}soudre dans $M_{2}\left( \mathbb{R}\right) $ l'%
\'{e}quation $M^{2}+M=A$

\begin{enumerate}
	\item Diagonaliser $A$ avec matrice de passage.
	
	
	
	\item Trouver un polyn\^{o}me annulateur de $M,$ en d\'{e}duire les valeurs
	propres possibles pour $M.$
	
	\item Montrer que $M$ est diagonalisable et pr\'{e}ciser les formes
	diagonales possibles. Conclure.
\end{enumerate}
\end{Exo}

\begin{Exo}
	
 Niaiserie: une matrice \'{e}gale \`{a} son inverse est-elle
diagonalisable?
\end{Exo}

\begin{Exo}
	Donner une condition nécessaire et suffisante pour qu'une matrice de rang 1 soit diagonalisable.
\end{Exo}

 \begin{Exo}
	Soit $A\in M_{n}\left( \mathbb{R}\right) $ telle que $A^{3}=A+I,$
montrer que $\det A>0.$
\end{Exo}

 \begin{Exo}
 	Soit $A\in M_{n}\left( \mathbb{C}\right) $ telle que $%
A^{3}-5A^{2}+4A=0.$ Montre que $tr\left( A\right) \in \mathbb{N}.$

 \end{Exo}

\begin{Exo}
	Soit $A\in M_{2n+1}(\R)$ non nulle telle que $A^3+4A=0$. Montrer que $A$ n'est pas diagonalisable sur $\R$. L'est-elle sur $\C$? Montrer aussi que $A$ n'est pas inversible.
\end{Exo}

\begin{Exo}
	D\'{e}terminer la limite de la suite $\left( A^{n}\right) _{n\in 
		\mathbb{N}}$ avec $A=\frac{1}{2}\left( 
	\begin{array}{ccc}
		-1 & -3 & 0 \\ 
		0 & 2 & 0 \\ 
		-3 & -3 & 2%
	\end{array}%
	\right) .$
\end{Exo}

\begin{Exo}
	Soit $A\in M_{n}\left(\K\right) $ telle que $Tr\left(
	A\right) \neq 0.$ On d\'{e}finit $\phi :\begin{cases}
M_{n}\left( \mathbb{K}\right) \rightarrow M_{n}\left( \mathbb{K}\right) \\ 
M\mapsto Tr\left( A\right) M-Tr\left( M\right) A
	\end{cases}$.	
	V\'{e}rifier (rapidement) que $\phi $ est lin\'{e}aire, d\'{e}terminez $\ker
	\phi $ et $\Ima \phi ,$ montrer que $\phi $ est diagonalisable.
\end{Exo}

\begin{Exo}
	Soit $A\in M_n(\R)$ diagonalisable telle que $\forall(\lambda,\mu)\in sp(A),\lambda+\mu\neq 0$. Soit $M\in M_n(\R)$ telle que $AM+MA=0$, montrer que $M=0$.
\end{Exo}

\begin{Exo}
	Soit $u\in L(E)$ vérifiant $u^3=u$.
	\begin{enumerate}
		\item
		Calculer $u^2(x)$ pour $x\in \Ima(u)$.
		\item
		Montrer que l'endomorphisme induit par $u$ sur $F$ est inversible.
		\item En déduire que $u$ est de rang pair.
	\end{enumerate}
\end{Exo}

\begin{Exo}
	Soit $A$ la matrice 
	$A=\left(\begin{array}{ccc}
		1&0&-1\\
		1&2&1\\
		2&2&3
	\end{array}\right).$
	 Diagonaliser $A$.
		 En déduire toutes les matrices $M$ qui commutent avec $A$.

\end{Exo}

\begin{Exo}
	Soit $A$ une matrice nilpotente, montrer qu'elle admet une seule valeur propre à préciser. Que dire d'une matrice nilpotente ET diagonalisable?
\end{Exo}


\begin{Exo}
	Montrer qu'en dimension $n>0$ un endomorphisme admet une droite ou un plan stable. On pourra raisonner matriciellement et ce n'est pas facile!!!
\end{Exo}

\begin{Exo}
	Donner un plan d'\'{e}tude pour trigonaliser les matrices de taille $2$
	et $3.$ Trigonaliser : 
	\[
	A=\left( 
	\begin{array}{cc}
		1 & 2 \\ 
		-2 & 5%
	\end{array}%
	\right) ,B=\left( 
	\begin{array}{ccc}
		0 & -1 & -1 \\ 
		3 & 4 & 2 \\ 
		1 & 1 & 1%
	\end{array}%
	\right) ,C=\left( 
	\begin{array}{ccc}
		1 & -1 & -1 \\ 
		2 & 4 & 2 \\ 
		-1 & -1 & 1%
	\end{array}%
	\right) ,D=\left( 
	\begin{array}{ccc}
		4 & -2 & 5 \\ 
		1 & 4 & -1 \\ 
		0 & 1 & 1%
	\end{array}%
	\right) 
	\]
\end{Exo}

\begin{Exo}
	Soit $A\in M_n(\C)$ telle que $Tr(A^k)=0$ pour $1\leqslant k\leqslant n$. Montrer que $A$ est nilpotente.
\end{Exo}

\begin{Exo}
	Soit $A\in M_{n}\left(\R\right) $ et $P\in \R\left[ X%
	\right] $ tels que $P\left( A\right) $ triangulaire sup\'{e}rieure \`{a}
	termes diagonaux tous distincts. Montrer que $A$ est diagonalisable puis
	triangulaire sup\'{e}rieure.
\end{Exo}



\begin{Exo}[Un exemple de chaîne de Markov, paresseusement copié sur un TD de Mr Zwolska!]

	Soit $a \in] 0 ; 1$ [. Une puce se déplace entre trois points $A, B$ et $C$ selon la règle suivante:
	\begin{enumerate}
		\item
		Si elle est en $A$, la probabilité qu'elle se déplace en $B$ est égale à $a$, et celle qu'elle se déplace en $C$ est $1-a$;
		\item
		Si elle est en $B$, la probabilité qu'elle se déplace en $A$ est égale à $a$, et celle qu'elle se déplace en $C$ est $1-a$;
		\item
		Si elle est en $C$, la probabilité qu'elle se déplace en $A$ est égale à $a$, et celle qu'elle se déplace en $B$ est $1-a$; On suppose qu'au départ la puce est en $A$.
	\end{enumerate}

	Pour tout $n$ de $\mathrm{N}$, on note $A_{n}$ (resp. $B_{n}$ et $C_{n}$ ) l'événement : "la puce est en $A$ (resp. $B$ et $C$ ) après le $n^e$ déplacement" et $X_{n}$ la matrice colonne $X_{n}=\left(\begin{array}{l}P\left(A_{n}\right) \\ P\left(B_{n}\right) \\ P\left(C_{n}\right)\end{array}\right)$
	
	\begin{enumerate}
		\item
		Montrer que $\forall n \in \mathrm{N}, \quad X_{n+1}=M X_{n}$, où $M=\left(\begin{array}{ccc}0 & a & a \\ a & 0 & 1-a \\ 1-a & 1-a & 0\end{array}\right)$
		\begin{enumerate}
			\item
			Montrer que 1 est valeur propre de ${ }^{t} M$ et préciser le vecteur propre associée.
			\item
			Donner les autres valeurs propres de ${ }^{t} M$ et sous-espaces propres associés.
			\item
			${ }^{t} M$ est elle diagonalisable?
		\end{enumerate}
	\end{enumerate}

	Dans toute la suite de l'exercice, on fixe $a=\frac{1}{4}$.
	\begin{enumerate}
		\item
		Diagonaliser alors la matrice ${ }^{t} M$ et en déduire, pour tout $n$ de $\mathrm{N},\left({ }^{t} M\right)^{n}$ puis $M^{n}$.
		\item
		Préciser la matrice $X_{0}$ et déterminer, pour tout $n$ de $\mathrm{N}, X_{n}$ en fonction de $n$.
		\item
		En déduire, pour tout $n$ de $\mathbb{N}$, les probabilités $P\left(A_{n}\right), P\left(B_{n}\right)$ et $P\left(C_{n}\right)$.
		\item
		Déterminer les limites de ces suites lorsque $n$ tend vers $+\infty$.
	\end{enumerate}
\end{Exo}


\begin{Exo}
	Soient $X,Y,Z$ trois variables aléatoires indépendantes, suivant la même loi binomiale B$\mathcal{B}(n,p)$. Pour tout $\omega\in\Omega$ on pose $A(\omega)=\begin{pmatrix}
		X(\omega)	& Y(\omega) & Z(\omega) \\
		X(\omega)& Y(\omega) & Z(\omega) \\
		X(\omega)& Y(\omega) & Z(\omega)
	\end{pmatrix}$
	\begin{enumerate}
		\item Quelle est la probabilité qu'elle soit diagonalisable ?
		\item Quelle est la probabilité qu'elle soit une matrice de projecteur ?	
	\end{enumerate}
\end{Exo}

\begin{Exo}
	Soient $X,Y$ indépendantes suivant la même loi $\mathcal{G}(p)$ avec $0<p<1$. quelle est la probabilité que $A=\begin{pmatrix}
		X_1	& 1 \\
		0	& X_2
	\end{pmatrix}$ soit inversible? Diagonalisable?
\end{Exo}


\section*{Banque CCP}



\begin{Exo}
	On considère la matrice $A=\begin{pmatrix}
	0&a&1\\
	a&0&1\\
	a&1&0
\end{pmatrix}$ où $a$ est un réel.
\begin{enumerate}
	\item
	Déterminer le rang de $A$.
	\item
	Pour quelles valeurs de $a$, la matrice $A$ est-elle diagonalisable? 
\end{enumerate}
\end{Exo}




\begin{Exo}
	Soit $A=\begin{pmatrix}
	0 & 0 & 1 \\ 
	1 & 0 & 0 \\ 
	0 & 1 & 0
\end{pmatrix} \in \mathcal{M}_{3}\left( \mathbb{C}\right)$~.

\begin{enumerate}
	\item D\'{e}terminer les valeurs propres et les vecteurs propres de $A$. $A$ est-elle diagonalisable?
	\item Soit $\left( a,b,c\right) \in \mathbb{C}^{3}$ et $B=a\mathrm{I}_{3}+bA+cA^{2}$, où $\mathrm{I}_3$ désigne la matrice identité d'ordre 3.\\ D\'{e}duire de la question \textbf{1.} les \'{e}l\'{e}ments propres de $B$.
\end{enumerate}
\end{Exo}


\begin{Exo}
	Soit $u$ et $v$ deux endomorphismes d'un $\mathbb{R}$-espace vectoriel $E$.
	
	\begin{enumerate}
		\item Soit $\lambda$ un réel non nul. Prouver que si $\lambda$ est valeur propre de $u\circ v$, alors $\lambda$ est valeur propre de $v\circ u$.\\
		\:\:\:\:
		\item On considère, sur $E=\mathbb{R}\left[ X\right] $ les endomorphismes $u$ et $v$ définis par 
		$u$:\:$P\longmapsto \displaystyle\int_{1}^{X}P $ et $v$:\:$P\longmapsto P'$ .\\
		Déterminer $ \mathrm{Ker} (u\circ v)$ et $\mathrm{Ker} (v\circ u)$.
		Le résultat de la question 1. reste-t-il vrai pour $\lambda=0$?\:\:\:\:
		\item 
		Si $E$ est de dimension finie, démontrer que le résultat de la première question reste vrai pour $\lambda=0$.\\
		\textbf{Indication }: penser à utiliser le déterminant.\:\:\:\:
		
	\end{enumerate}
\end{Exo}

\begin{Exo}
	Soit la matrice $M=\begin{pmatrix}
	0 & a & c \\ 
	b & 0 & c \\ 
	b & -a & 0
\end{pmatrix}$ o\`{u} $a,b,c$ sont des r\'{e}els.


$M$ est-elle diagonalisable dans $\mathcal{M}_{3}\left(\mathbb{R}\right)$?
$M$ est-elle diagonalisable dans $\mathcal{M}_{3}\left( \mathbb{C}\right)$?
\end{Exo}

\begin{Exo}
	On considère la matrice $A=\begin{pmatrix}
		-1 & -4 \\ 
		1 & 3
	\end{pmatrix}$~.
	
	\begin{enumerate}
		\item Démontrer que $A$ n'est pas diagonalisable.
		
		\item On note $f$ l'endomorphisme de $\mathbb{R}^{2}$ canoniquement associ\'{e} \`{a} $A$. \\Trouver une base $\left( v_{1},v_{2}\right)$ de $\mathbb{R}^{2}$ dans laquelle la matrice de $f$ est de la forme $\begin{pmatrix}
			a & b \\ 
			0 & c
		\end{pmatrix}$.\\
		On donnera explicitement les valeurs de $a$, $b$ et $c$.
		
		\item En déduire la résolution du syst\`{e}me différentiel $\left\{
		\begin{array}{l}
			x^{\prime }=-x-4y \\ 
			y^{\prime }=x+3y
		\end{array}
		\right.$~.
	\end{enumerate}
\end{Exo}



\begin{Exo}
	Soit la matrice $A=\begin{pmatrix}
	1 & -1 & 1 \\ 
	-1 & 1 & -1 \\ 
	1 & -1 & 1
\end{pmatrix}$~.
 Démontrer que $A$ est diagonalisable de trois manières:
	\begin{enumerate}
		\item En calculant directement le déterminant $\text{det}(\lambda \mathrm{I}_3-A)$, où $\mathrm{I}_3$ est la matrice identité d'ordre 3, et en déterminant les sous-espaces propres,
		\item En utilisant le rang de la matrice,
		\item En calculant $A^2$.
	\end{enumerate}
\end{Exo}

\section*{Bonus}

\begin{Exo}
	Soit
	$$
	A=\left(\begin{array}{cccc}
		0 & \cdots & 0 & 1 \\
		\vdots & & \vdots & \vdots \\
		0 & \cdots & 0 & 1 \\
		1 & \cdots & 1 & 2
	\end{array}\right)
	$$
\begin{enumerate}
	\item
	Quelle est la dimension de Ker $A$ ?
	\item
	Trouver les valeurs propres non nulles de $A$. (Indication: résoudre directement l'équation $A X=\lambda X$. On peut aussi calculer le polynôme caractéristique mais les calculs sont plus fastidieux.)
	\item
	La matrice $A$ est-elle diagonalisable?
\end{enumerate}
\end{Exo}

\begin{Exo}\ 
	
	\begin{enumerate}
		\item
		Donner un exemple de matrice de ${M}_{2}(\mathbb{R})$ dont le spectre est vide.
		\item
		Donner un exemple de matrice diagonalisable de ${M}_{2}(\mathbb{R})$ dont le spectre est réduit à $\{1\} .$ Que peut-on dire d'une telle matrice?
		\item
		Donner un exemple de matrice non diagonalisable de ${M}_{2}(\mathbb{R})$ dont le spectre est réduit à $\{1\}$.
		\item
		Donner un exemple de matrice non diagonale de ${M}_{2}(\mathbb{R})$ dont le spectre est $\{-1,1\} .$ Une telle matrice est-elle toujours diagonalisable?
		\item
		Donner un exemple de matrice non nulle de ${M}_{2}(\mathbb{R})$ dont le spectre est réduit à $\{0\}$. Une telle matrice peut-elle être diagonalisable?
		\item
		Donner un exemple de couple $(A, B) \in {M}_{2}(\mathbb{R})^{2}$ telle que $\operatorname{Spec}_{\mathbb{R}}(A)=\operatorname{Spec}_{\mathbb{R}}(B)=$ $\{1\}$ et telle que $A$ et $B$ ne soient pas semblables.
		\item
		Peut-on trouver une paire $(A, B) \in {M}_{2}(\mathbb{R})^{2}$ telle que $\operatorname{Spec}_{\mathbb{R}}(A)=\operatorname{Spec}_{\mathbb{R}}(B)=$ $\{-1,1\}$ et telle que $A$ et $B$ ne soient pas semblables?
	\end{enumerate}
\end{Exo}

\begin{Exo}
Soit $n \in \mathbb{N}$. On définit l'endomorphisme $\varphi: \mathbb{C}_{n}[X] \rightarrow \mathbb{C}_{n}[X]$ par
	$$
	\forall P \in \mathbb{C}_{n}[X], \quad \varphi(P)=\left(X^{2}-1\right) P^{\prime \prime}+(2 X+1) P^{\prime} .
	$$
	\begin{enumerate}
		\item
		Déterminer les valeurs propres de l'endomorphisme $\varphi$.
		\item
		L'endomorphisme $\varphi$ est-il diagonalisable?
	\end{enumerate}
\end{Exo}

\begin{Exo}
Soit $n \in \mathbb{N} .$ On définit l'endomorphisme $\varphi: \mathbb{C}_{n}[X] \rightarrow \mathbb{C}_{n}[X]$ par
	$$
	\forall P \in \mathbb{C}_{n}[X], \quad \varphi(P)=\left(X^{2}-1\right) P^{\prime}-n X P .
	$$
	\begin{enumerate}
		\item
			Déterminer les valeurs propres et les vecteurs propres de l'endomorphisme $\varphi$.
			\item
			 L'endomorphisme $\varphi$ est-il diagonalisable?
	\end{enumerate}
\end{Exo}

\end{document}
