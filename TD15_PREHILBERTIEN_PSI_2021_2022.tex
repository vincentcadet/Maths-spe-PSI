  
\documentclass[12pt,a4paper]{article}
% Engine-specific settings
% Detect pdftex/xetex/luatex, and load appropriate font packages.
% This is inspired by the approach in the iftex package.
% pdftex:

\usepackage[T1]{fontenc}
\usepackage[utf8]{inputenc}
\usepackage[french]{babel}
\frenchbsetup{StandardLists=true}
\usepackage{enumitem}
\usepackage{systeme}
\usepackage{amsmath,amssymb}
\usepackage[thmmarks,amsmath]{ntheorem}
\usepackage[colorlinks=true]{hyperref}
\usepackage{fullpage}
%\usepackage{eulervm}
\usepackage[charter]{mathdesign}

\usepackage{graphicx}
\usepackage{array}
\usepackage{multicol}
\usepackage[makestderr]{pythontex}
\restartpythontexsession{\thesection}
\usepackage{geometry}
\geometry{tmargin=1.5cm,bmargin=1.5cm,lmargin=1.2cm,rmargin=1.2cm,headheight=12cm}
\usepackage{array}
\usepackage[svgnames,table]{xcolor}
\usepackage[tikz]{ bclogo}
\usepackage{url}
\urlstyle{same}
\usepackage{pifont}
\usepackage{multicol}
\usepackage{diagbox} %oblique dans les tableaux
%\usepackage[framemethod=TikZ]{mdframed}
\usepackage{fancyhdr}
\pagestyle{fancyplain}
\lhead{\textit{CSI2B TD15}}
\chead{\textsc{Espaces préhilbertiens}}
\rhead{\textit{2021-2022}} 
\setlength{\headsep}{1mm}
\DeclareMathOperator{\lcro}{\textnormal{\textlbrackdbl}}
\DeclareMathOperator{\rcro}{\textnormal{\textrbrackdbl}}

\newcommand{\norme}[1]{\left\lVert#1\right\rVert}
\newcommand{\ps}[2]{\left\langle#1,#2\right\rangle}

\renewcommand*{\thefootnote}{\fnsymbol{footnote}}
\newcommand{\un}{(u_n)_n}
\newcommand{\R}{\mathbb{R}}
\newcommand{\C}{\mathbb{C}}
\newcommand{\Q}{\mathbb{Q}}
\newcommand{\Z}{\mathbb{Z}}
\newcommand{\N}{\mathbb{N}}
\newcommand{\K}{\mathbb{K} }
\newcommand{\I}{\mathbf{i}}
\renewcommand{\Re}{\mathcal{R}e}
\renewcommand{\Im}{\mathcal{I}m}
\DeclareMathOperator{\Ima }{Im}
\newcommand{\diff}{\mathop{}\mathopen{}\mathrm{d}}%element differentiel
\newcommand{\conj}[1]{\overline{#1}}
\newcommand{\E}{\mathrm{e}}
{%
\theoremstyle{break}
\theoremprework{%
\rule{0.5\linewidth}{0.3pt}}
\theorempostwork{\hfill%
\rule{0.5\linewidth}{0.3pt}}
\theoremheaderfont{\scshape}
\theoremseparator{ ---}
\newtheorem{Prop}{%
\textcolor{blue}{Proposition}}
}

{%
\theoremstyle{break}
\theoremprework{%
\rule{0.6\linewidth}{0.5pt}}
\theorempostwork{\hfill%
\rule{0.6\linewidth}{0.5pt}}
\theoremheaderfont{\scshape}
\newtheorem{Theo}{%
\textcolor{red}{Théorème}}[section]
}


{%
\theoremheaderfont{\sffamily\bfseries}
\theorembodyfont{\sffamily}
\newtheorem{Def}{%
\textcolor{green}{Définition}}[section]
}

{%
\theorembodyfont{\small}
\theoremsymbol{$\square$}
\newtheorem*{Dem}{Démonstration}
}

{%
\theorembodyfont{\small}
\newtheorem*{Exemple}{Exemple}
}


{%
\theorembodyfont{\small}
\newtheorem{Exo}{Exercice}
}

{%
\theoremnumbering{Roman}
\theorembodyfont{\normalfont}
\newtheorem{Rem}{Remarque}
}




\begin{document}


\emph{\textbf{
“Ce n'est pas que je suis si intelligent, c'est que je reste plus longtemps avec les problèmes.” 
Albert Einstein
}}

%\begin{Exo}
%Donner une CNS sur les réels $a,b,c,d$ pour que:
%$$\phi((x,y),(x',y'))=axx'+bxy'+cx'y+dyy'$$définisse un produit scalaire sur $\R^2$.	
%\end{Exo}

\begin{Exo}
	Soit $E=\{P\in\R_n[X],P(0)=P(1)=0\}$
	\begin{enumerate}
		\item
		Montrer que $E$ est un $\R$-ev, en donner une base et la dimension.
		\item
		Montrer ensuite que $(P,Q)\mapsto -\int_0^1(P''Q+PQ'')$ définit un produit scalaire sur $E$.
	\end{enumerate}
\end{Exo}



\begin{Exo}
Montrer que
$\langle f,g\rangle=f(1)g(1)+\int_0^1 f'(t)g'(t)dt$ définit un produit scalaire sur $E=\mathcal C^1([0,1],\mathbb R)$. En déduire pour toute $f$ de $\mathcal C^1([0,1],\mathbb R)$ on a:
$$\left(f(1)+\int_0^1 f'(t)dt\right)^2\leqslant 2\left(f(1)^2+\int_0^1f'(t)^2dt\right)$$
\end{Exo}

%\begin{Exo}
%	Soient $E$ espace vectoriel euclidien, $ a \in E$ et $\alpha,\beta,\gamma \in \R$.
%	Résoudre l'équation $$\alpha \ps{x}{x}+ \beta \ps{x}{a} + \gamma = 0$$
%\end{Exo}

\begin{Exo}
Dans $\R^3$ une famille $(a,b,c,d)$ de $4$ vecteurs non nuls peut-elle être orthogonale? 
\end{Exo}

\begin{Exo}
	Trouver trois vecteurs $u,v,w$  de $\R^2$ tels que $\norme{u+v+w}^2=\norme{u}^2+\norme{v}^2+\norme{w}^2$ sans que $(u,v,w)$ soit une famille orthogonale.
\end{Exo}

\begin{Exo}
Déterminer dans $\R_2[X]$ muni de $\ps{P}{Q}=P(-1)Q(-1)+P(0)Q(0)+P(1)Q(1)$ l'orthogonal de $\R_1[X]$.
\end{Exo}

\begin{Exo}[CCP]\ 
	\begin{enumerate}
	\item Montrer que l'application$ \quad (A,B) \mapsto \operatorname{Tr} ( ^t\!A B) \quad $ définit un produit scalaire sur $\mathcal{M}_2(\mathbb R) $.
	\item Montrer que $E=\left\{\begin{pmatrix} a & b \\ -b & a \end{pmatrix},(a,b)\in \mathbb{R}^2\right\}$ est un sous espace vectoriel de $\mathcal{M}_2(\mathbb{R}) $. 
	\item Trouver une base orthonormale de $E^\perp$.
	\item Calculer la distance de $A=  \begin{pmatrix}1 & 1\\1&1\end{pmatrix} \ \ $ à $E^\perp$.
		\end{enumerate}
\end{Exo}

\begin{Exo}
On note pour $n\in \N:P_n(X)=((X^2-1)^n)^{(n)}$.
\begin{enumerate}
\item
Calculer $P_n$ pour $0\leqslant n  \leqslant 3$.
\item
Montrer que $P_n$ est un polynôme dont on précisera le degré et le coefficient dominant.
\item
Justifier que $P_n$ admet $n$ racines distinctes dans $]-1,1[$.
\item
Préciser la parité de $P_n$, sa valeur en $1$ puis en $-1$.
\item
On munit $\R_n[X]$ du produit scalaire: $\ps{P}{Q}=\int_{-1}^1PQ$, montrer que $(P_0,P_1,...,P_n)$ en est une base orthogonale et préciser la norme de $P_n$.
\end{enumerate}
\end{Exo}

\begin{Exo}
Soit $E$ un espace euclidien et $a$ un vecteur unitaire de $E$. On note pour tout réel $\alpha$:
$$\phi_{\alpha}:x\in E\mapsto x+\alpha \ps{a}{x}a$$
\begin{enumerate}
\item
Montrer que $\phi_{\alpha}$ est un endomorphisme.
\item
Montrer que $\mathcal{A}=\{\phi_{\alpha},\alpha\in \R\}$ est stable par composition, commutatif et contient $id_E$.
\item Déterminer $\ker \phi_{\alpha}$.
\item Montrer que si $\alpha\neq -1$ alors $\phi_{\alpha}$ est inversible \textbf{dans} $\mathcal{A}$.
\end{enumerate}
\end{Exo}


\begin{Exo}
Soit $x,y$ dans $E$ préhilbertien. Montrer qu'ils sont même norme ssi $x-y \perp x+y$.
\end{Exo}


\begin{Exo}
Soient $F$ et $G$ deux sous-espaces vectoriels d'un espace préhilbertien $E$. Montrer que :
$$F\subset (F^{\perp})^{\perp},(F+G)^\perp=F^\perp\cap G^\perp\text{ et }F^\perp+G^\perp\subset (F\cap G)^\perp.$$
Que se passe-t-il en dimension finie?
\end{Exo}

\begin{Exo}
	Si $F$ (donc $E$...) est de dimension infinie on n'a plus nécessairement $E=F\oplus F^{\perp}$. Considérer par exemple l'hyperplan des fonctions nulles en $0$ dans $\mathcal{C}([0,1],\R)$ muni de son produit scalaire usuel.
\end{Exo}

\begin{Exo}
		Soit $x,y$ non nuls dans $(E,\ps{}{})$ préhilbertien, redémontrer l'inégalité de Cauchy-Schwarz et le cas d'égalité en développant $\norme{\frac{x}{\norme{x}}-\frac{y}{\norme{y}}}^2$ et $\norme{\frac{x}{\norme{x}}+\frac{y}{\norme{y}}}^2$.
\end{Exo}

\begin{Exo}
A l'aide de l'inégalité de Cauchy-Schwarz dans $\R^n$ usuel, montrer que:
$$\forall n\geqslant 1,\sum_{k=1}^n k\sqrt{k}\leqslant \frac{n(n+1)\sqrt{2n+1}}{2\sqrt{3}}$$
\end{Exo}

\begin{Exo}
On suppose $x_1,...,x_n$ dans $\R_+^*$ vérifiant $x_1+...+x_n=1$. Montrer que $\sum_{i=1}^n\frac{1}{x_i}\geq n^2$ et préciser le(s) cas d'égalité.
\end{Exo}


\begin{Exo}
Rappeler l'énoncé de l'inégalité de Cauchy-Schwarz. Montrer que pour toute fonction continue d'un segment $[a,b]$ dans
$\R$, on a 
$$
  \left(\int_{a}^{b}{f(t)dt}\right)^{2}\leq
  (b-a)\int_{a}^{b}{\Big(f(t)\Big)^{2}dt}
$$

Pour quelle(s) fonction(s) a-t-on l'égalité ?
\end{Exo}

%\begin{Exo}
%Soit $f$ de classe $\mathcal{C}^1$ sur le segment $[a,b]$ avec $f(a)=0$.
%\begin{enumerate}
%\item
%Soit $x\in [a,b]$, en exprimant $f(x)$ à l'aide d'une intégrale de $f'$, montrer que:
%$$f(x)^2\leqslant (x-a)\int_a^x f'(t)^2dt$$
%\item
%En déduire que $\int_a^b f^2\leqslant \frac{(b-a)^2}{2}\int_a^b f'^2$.
%\end{enumerate}
%\end{Exo}

\begin{Exo}
	Soit $(E,\ps{}{})$ euclidien de BON $(e_1,...,e_n)$ et $f\in L(E)$. Montrer que la trace de $f$ est égale à $\sum_{i=1}^n\ps{f(e_i)}{e_i}$.
\end{Exo}

\begin{Exo}
	
	Soit $(E,\ps{}{})$ euclidien et $u\in L(E)$ tel que pour tout $x$ de $E$ $u(x)$ soit orthogonal à $x$.
\begin{enumerate}
	\item
	Montrer que pour tout $x,y$ de $E$ on a $\ps{u(x)}{y}=-\ps{x}{u(y)}$
	\item En déduire que $\ker u$ et $\mathrm{Im} u$ sont orthogonaux puis supplémentaires.
\end{enumerate}

\end{Exo}

\begin{Exo}
Soit $ E  $ un espace préhilbertien, et $ \left( e_{1},...,e_{n}\right)  $ des
vecteurs unitaires v\'{e}rifiant :$\forall x\in E,\left\| x\right\| ^{2}=\sum_{i=1}^{n}\left\langle
x,e_{i}\right\rangle ^{2}$.
Montrer que $ \left( e_{1},...,e_{n}\right) $  est une base orthonormale de $E$.
\end{Exo}

\begin{Exo}
Soit $E$ euclidien de BON $(e_1,...,e_n)$ et $f:E\to E$ vérifiant $f(0)=0$ et $\forall (x,y)\in E^2,\norme{f(x)-f(y)}=\norme{x-y}$.
\begin{enumerate}
\item Montrer que $f$ conserve la norme puis le produit scalaire. En déduire que $(f(e_1),...,f(e_n)$ est une BON.
\item Montrer que $\forall x\in E,f(x)=\sum_{i=1}^n\ps{x}{e_i}f(e_i)$, en déduire que $f$ est linéaire. Que se passe t-il si on ne suppose plus $f(0)=0$?
\end{enumerate}
\end{Exo}

\begin{Exo}
Soit $H$ un hyperplan de $E$ euclidien et $u\in H^{\perp}$. On note $p$ la projection orthogonale sur $H$ et $s$ la symétrie orthogonale par rapport à $H$. Vérifier que:
$$\forall x \in E:p(x)=x-\frac{\ps{x}{u}}{||u||^2}u\text{ et } s(x)=x-2\frac{\ps{x}{u}}{||u||^2}u$$
\end{Exo}

\begin{Exo}
Soit $(E,\ps{}{})$ préhilbertien, montrer qu'un projecteur $p$ est un projecteur orthogonal ssi $$\forall x\in E,\norme{p(x)}\leq \norme{x}$$
\end{Exo}

\begin{Exo}[un projecteur est symétrique ssi il est orthogonal]
Soit $(E,\ps{}{})$ préhilbertien, montrer qu'un projecteur $p$ est un projecteur orthogonal ssi $$\forall (x,y)\in E^2,\ps{p(x)}{y}=\ps{x}{p(y)}$$
\end{Exo}


%\begin{Exo}
%On munit $E = \R_n[X]$ du produit scalaire :
%Pour $P = \sum_i a_iX^i$ et $Q = \sum_i b_iX^i$,
%$(P\mid Q) = \sum_i a_ib_i$.
%
%Soit $H = \{ P \in E \text{ tq } P(1) = 0 \}$.
%
%\begin{enumerate}
%  \item Trouver une base orthonormale de $H$.
%  \item Calculer $d(X,H)$.
%    
%\end{enumerate}
%
%\end{Exo}


\begin{Exo}
Dans $\mathbb R^3$ muni du produit scalaire canonique, orthonormaliser en suivant le procédé de Schmidt la base suivante :
$$u=(1,0,1),\ v=(1,1,1),\ w=(-1,-1,0)$$
\end{Exo}

\begin{Exo}
Construire une BON de $\R_3[X]$ muni de $\ps{P}{Q}=\int_0^1P(t)Q(t)dt$.
\end{Exo}

\begin{Exo}
Soit $E$ un espace euclidien de dimension 4, ${\cal B} = (\vec e_1, \dots, \vec e_4)$
une base orthonormée de $E$, et $F$ le sous-espace vectoriel d'équations dans $\cal B$ :
$$\begin{cases} x+y+z+t = 0\cr x+2y+3z+4t =0 \cr\end{cases}$$

\begin{enumerate}
  \item Trouver une base orthonormée de $F$.
  \item Donner la matrice dans $\cal B$ de la projection orthogonale sur $F$.
  \item Calculer $d(\vec e_1, F)$.
\end{enumerate}
\end{Exo}



\begin{Exo}
On munit $\R^n$ du produit scalaire usuel.
Soit $H = \left \{ (x_1,\dots,x_n) \in \R^n \text{ tq } a_1x_1 + \dots + a_nx_n = 0\right \}$
où $a_1,\dots,a_n$ sont des réels donnés non tous nuls.
Chercher la matrice dans la base canonique de la projection orthogonale
sur $H$.

\end{Exo}







\begin{Exo}
Soit $E = \R_n[X]$ et $\ps{P}{Q} =  \int_{0}^1 P(t)Q(t)\,d t$.

\begin{enumerate}
  \item Montrer que $\left(E,\ps{}{} \right)$, est un espace euclidien.
  \item Soit $K = \R_{n-1}[X]^\bot$ et $P \in K\setminus\{0\}$.
    Quel est le degré de~$P$~?
  \item Soit $\Phi\ :\ x  \mapsto  \int_{t=0}^1 P(t)t^x\,d t$.
    Montrer que $\Phi$ est une fonction rationnelle.
    
  \item Trouver $\Phi$ à une constante multiplicative près.
    
  \item En déduire les coefficients de~$P$ puis  une base orthogonale de~$E$.
\end{enumerate}

\end{Exo}

\begin{Exo}[Représentation de Riesz]
\ 
\begin{enumerate}
\item
Montrer que pour toute forme linéaire $\phi$ sur $E$ euclidien il existe un unique vecteur  $a$ de $E$ tel que $\phi:x\in E\mapsto \ps{x}{a}$ Préciser son noyau.
\item
Montrer qu'il existe un unique polynôme $A$ dans $R_n[X]$ tel que pour tout $P$ de $R_n[X]$ $P(0)=\int_0^1A(t)M(t)\diff t$. Montrer aussi que $A$ est de degré $n$. Que se passe-t-il si on remplace $\R_n[X]$ par $R[X]$?
\end{enumerate}
\end{Exo}

\begin{Exo}[Polynômes de Laguerre]\ 
	\begin{enumerate}
		\item
		Montrer que $\ps{P}{Q}=\int_0^{+\infty}\E^{-t}P(t)Q(t)\diff t$ définit un produit scalaire sur $\R[X]$.
		\item
		Calculer $\ps{X^p}{X^q}$ avec $p,q$ dans $\N$.
		\item
		Construire une BON du sev $\R_2[X]$, en déduire $\inf_{\left( a,b\right)\in\R^3}\int_{0}^{+\infty}\E^{-t}(t^3-at-b)\diff t$.
	\end{enumerate}
	
\end{Exo}



\begin{Exo}[Polynômes de Tchebychev]
\

\begin{enumerate}
\item Montrer que $\left\langle f,g\right\rangle =\int_{0}^{\pi }f\left(
\cos \theta \right) g\left( \cos \theta \right) d\theta $ d\'{e}finit un ps
sur $E=C\left( \left[ -1,1\right] ,\R\right) .$

\item Montrer qu'il existe une unique suite de polynômes $\left(
T_{n}\right) _{n\in \N}$ v\'{e}rifiant:
$
\forall \theta \in \R,\cos \left( n\theta \right) =T_{n}\left( \cos
\theta \right) 
$.
On pourra commencer par déterminer les $3$ premiers termes puis
exploiter les formules trigonométriques développant $\cos \left(
\left( n+1\right) \theta \right) $ et $\cos \left( \left( n-1\right) \theta
\right) ...$ Préciser le degré et le coefficient dominant de $T_{n},$
dit $n^{e}$ polynôme de Tchebychev.

\item Montrer que $\left( T_{n}\right) _{n\in \N}$ est une famille
orthogonale et calculer $\left\Vert T_{n}\right\Vert .$

\item En d\'{e}duire $\inf_{\left( a_{0},a_{1},...,a_{n-1}\right)
\in \R^n}\int_{0}^{\pi }\left( \cos ^{n}\theta +a_{n-1}\cos ^{n-1}\theta
+...+a_{1}\cos \theta +a_{0}\right) ^{2}d\theta $ ()On pourra exprimer: $\cos
^{n}\theta +a_{n-1}\cos ^{n-1}\theta +...+a_{1}\cos \theta +a_{0}$ avec les
polynômes de Tchebychev et utiliser le théorème de Pythagore.)
\end{enumerate}
\end{Exo}

\begin{Exo}[CCP PSI]
	Soit $a_0,\dots,a_n$ des réels deux à deux distincts. On pose :
	$\forall (P,Q)\in\mathbb R_n[X]^2,\ (P|Q)=\displaystyle\sum_{k=0}^nP(a_k)Q(a_k).$
	\begin{enumerate}
		\item
		Montrer qu'il s'agit d'un produit scalaire.
		\item
		On pose : $F=\left\{P\in\mathbb R_n[X]\ /\ \displaystyle\sum_{k=0}^nP(a_k)=0\right\}$.
			Justifier rapidement que $F$ est un sous-espace vectoriel de $\mathbb R_n[X]$, calculer sa dimension ainsi que son orthogonal puis calculer la distance de $X^n$ à $F$.
	\end{enumerate}
\end{Exo}


\begin{Exo}[CCP PSI]
	L'espace euclidien $E=\mathscr M_n(\mathbb R)$ est muni du produit scalaire canonique défini par $\forall(A,B)\in E\times E,\quad (A|B)=\mathrm{tr}\big( ^tAB\big).$
	$\mathscr S_n(\mathbb R)$ est l'ensemble des matrices symétriques réelles et $\mathscr A_n(\mathbb R)$ celui des matrices antisymétriques réelles de $E$.
	\begin{enumerate}
		\item
		Montrer que $\mathscr S_n(\mathbb R)$ et $\mathscr A_n(\mathbb R)$ sont des sous-espaces supplémentaires orthogonaux dans $E$.
		\item
		Exprimer en fonction de $M\in E$ la distance de $M$ à $\mathscr S_n(\mathbb R)$. 
		\item
		Faire le calcul pour $M=\begin{pmatrix}1&1&\ldots&1\\2&2&\ldots&2\\\vdots&\vdots&&\vdots\\n&n&\ldots&n\end{pmatrix}$. 
	\end{enumerate}
\end{Exo}




%
%\begin{Exo}
%Soit $\mathcal{S} _{n}$ le groupe des permutations de $\left\{ 1,2,...,n\right\} $
%et $f:\sigma \in \mathcal{S}_{n}\mapsto \sum_{k=1}^{n}k\sigma \left( k\right) $
%
%\begin{enumerate}
%\item Calculer $f\left( id_{\mathcal{S} _{n}}\right) $ et $f\left( \sigma
%_{0}\right) $ ou $\sigma _{0}:k\mapsto n+1-k.$
%
%\item Justifier que $f$ a un minimum et un maximum sur $\mathcal{S}_{n}$, déterminer \`{a} l'aide de Cauchy-Schwarz $\max_{\mathcal{S}_{n}}f.$
%
%\item Soit $\sigma \in \mathcal{S}_{n},$ montrer que $f\left( \sigma _{0}\circ
%\sigma \right) =\frac{n\left( n+1\right) ^{2}}{2}-f\left( \sigma \right) $.
%
%\item En d\'{e}duire que $f\left( \sigma \right) \geq \frac{n\left(
%n+1\right) ^{2}}{2}-\max_{\mathcal{S}_{n}}f.$ Calculer cette quantité, la
%comparer à $f\left( \sigma _{0}\right) .$
%
%\item Conclure quand à $\min_{\mathcal{S}_n}f$
%\end{enumerate}
%
%\end{Exo}


\begin{Exo}[CCP PC]
	Soit $E$ un espace euclidien de dimension $n$.
	
	Soit $f$ un endomorphisme de $E$ tel que : $\forall(x,y)\in E^{2},\ \ (x|y)=0\Rightarrow(f(x)|f(y))=0$.
	Soit $(e_{1},\ldots,e_{n})$ une base orthonormée de $E$. 
	\begin{enumerate}
		\item
		$\forall(i,j)\in\{1,\ldots,n\}$, calculer $(f(e_{i}+e_{j})|f(e_{i}-e_{j}))$.
		\item
		Montrer qu'il existe $\alpha\in\mathbb{R_{+}}$ tel que $\forall i\in\{1,\ldots,n\}$, $\|f(e_{i})\|=\alpha$.
	\end{enumerate}
\end{Exo}


\begin{Exo}
Soit $P \in \R[X]$ de degré inférieur ou égal à 3 tel que
$ \int_{-1}^1 P^2(t)\,d t = 1$.

Montrer que $\sup\{ |P(x)|,x\in [-1,1]\} \leqslant 2\sqrt2$.

Indications : Pour $a\in\R$ montrer qu'il existe $P_a\in\R_3[X]$ tel que :
$\forall\ P\in\R_3[X],\ P(a) =  \int_{-1}^1 P(t)P_a(t)\,d t$.
Calculer explicitement $P_a$, et appliquer l'inégalité de Cauchy-Schwarz.
\end{Exo}

\begin{Exo}
	Résoudre pour $n\in \N^*$ et $x_1,...,x_n$ dans $\R$:
	$\begin{cases}
		x_1+...+x_n=n\\x_1^2+...+x_n^2=n
	\end{cases}$.
\end{Exo}

\begin{Exo}[famille obtusangle]
	Soit $E$ de dimension $n$ et $(e_1,...,e_p)$ tels que $\ps{e_i}{e_j}<0$ pour tout $i\neq j$.
	\begin{enumerate}
		\item
		Représenter une telle famille dans $\R^2$ usuel.
		\item
		On suppose que $\sum_{i=1}^{p}\lambda_i e_i=0$ avec les $\lambda_i$ réels, montrer que $\sum_{i=1}^{p}|\lambda_i| e_i=0$ (notant $u=\sum_{i=1}^{p}\lambda_i e_i$ et $v=\sum_{i=1}^{p}|\lambda_i| e_i=0$, calculer $\ps{u-v}{u+v}$.)
		\item
		Montrer alors que $(e_1,...,e_{p-1})$ est libre. Quel est le cardinal maximal d'une telle famille?
	\end{enumerate}
\end{Exo}

\begin{Exo}
	Soit $a,b$ deux vecteurs unitaires de $E$ euclidien et $f:x\in E\mapsto x-\ps{a}{x}b$.
	\begin{enumerate}
		\item
		Montrer que $f$ est un endomorphisme de $E$.
		\item
				Donner une condition nécessaire et suffisante sur $\ps{a}{b}$ pour que $f$ soit inversible.
		\item
		Déterminer $\ker(f-id_E)$ et $f(b)$ puis donner une condition nécessaire et suffisante sur $\ps{a}{b}$ pour que $f$ soit diagonalisable.
	\end{enumerate}
\end{Exo}

\begin{Exo}
	Soit $ E  $ un espace euclidien de dimension $n$, et $ \left( e_{1},...,e_{n}\right)  $ des
	vecteurs  v\'{e}rifiant :$\forall x\in E,\left\| x\right\| ^{2}=\sum_{i=1}^{n}\left\langle
	x,e_{i}\right\rangle ^{2}$.
	Montrer que $ \left( e_{1},...,e_{n}\right) $  est une base  de $E$. Notant $A=\left(\ps{e_i}{e_j}\right)_{1\leqslant i,j \leqslant n}$, montrer que $A$ est inversible égale à son carré, en déduire que $ \left( e_{1},...,e_{n}\right) $  est une base orthonormale  de $E$. 
\end{Exo}
\end{document}
