  
\documentclass[12pt,a4paper]{article}
% Engine-specific settings
% Detect pdftex/xetex/luatex, and load appropriate font packages.
% This is inspired by the approach in the iftex package.
% pdftex:

\usepackage[T1]{fontenc}
\usepackage[utf8]{inputenc}
\usepackage[french]{babel}
\frenchbsetup{StandardLists=true}
\usepackage{enumitem}
\usepackage{systeme}
\usepackage{amsmath,amssymb}
\usepackage[thmmarks,amsmath]{ntheorem}
\usepackage[colorlinks=true]{hyperref}
\usepackage{fullpage}
%\usepackage{eulervm}
\usepackage[charter]{mathdesign}

\usepackage{graphicx}
\usepackage{array}
\usepackage{multicol}
\usepackage[makestderr]{pythontex}
\restartpythontexsession{\thesection}
\usepackage{geometry}
\geometry{tmargin=1.5cm,bmargin=1.5cm,lmargin=1.2cm,rmargin=1.2cm,headheight=12cm}
\usepackage{array}
\usepackage[svgnames,table]{xcolor}
\usepackage[tikz]{ bclogo}
\usepackage{url}
\urlstyle{same}
\usepackage{pifont}
\usepackage{multicol}
\usepackage{diagbox} %oblique dans les tableaux
%\usepackage[framemethod=TikZ]{mdframed}
\usepackage{fancyhdr}
\pagestyle{fancyplain}
\lhead{\textit{CSI2B TD16}}
\chead{\textsc{Endomorphismes symétriques- Isométries}}
\rhead{\textit{2021-2022}} 
\setlength{\headsep}{1mm}
\DeclareMathOperator{\lcro}{\textnormal{\textlbrackdbl}}
\DeclareMathOperator{\rcro}{\textnormal{\textrbrackdbl}}

\newcommand{\norme}[1]{\left\lVert#1\right\rVert}
\newcommand{\ps}[2]{\left\langle#1,#2\right\rangle}

\renewcommand*{\thefootnote}{\fnsymbol{footnote}}
\newcommand{\un}{(u_n)_n}
\newcommand{\R}{\mathbb{R}}
\newcommand{\C}{\mathbb{C}}
\newcommand{\Q}{\mathbb{Q}}
\newcommand{\Z}{\mathbb{Z}}
\newcommand{\N}{\mathbb{N}}
\newcommand{\K}{\mathbb{K} }
\newcommand{\I}{\mathbf{i}}
\renewcommand{\Re}{\mathcal{R}e}
\renewcommand{\Im}{\mathcal{I}m}
\DeclareMathOperator{\Ima }{Im}
\newcommand{\diff}{\mathop{}\mathopen{}\mathrm{d}}%element differentiel
\newcommand{\conj}[1]{\overline{#1}}
\newcommand{\E}{\mathrm{e}}
{%
\theoremstyle{break}
\theoremprework{%
\rule{0.5\linewidth}{0.3pt}}
\theorempostwork{\hfill%
\rule{0.5\linewidth}{0.3pt}}
\theoremheaderfont{\scshape}
\theoremseparator{ ---}
\newtheorem{Prop}{%
\textcolor{blue}{Proposition}}
}

{%
\theoremstyle{break}
\theoremprework{%
\rule{0.6\linewidth}{0.5pt}}
\theorempostwork{\hfill%
\rule{0.6\linewidth}{0.5pt}}
\theoremheaderfont{\scshape}
\newtheorem{Theo}{%
\textcolor{red}{Théorème}}[section]
}


{%
\theoremheaderfont{\sffamily\bfseries}
\theorembodyfont{\sffamily}
\newtheorem{Def}{%
\textcolor{green}{Définition}}[section]
}

{%
\theorembodyfont{\small}
\theoremsymbol{$\square$}
\newtheorem*{Dem}{Démonstration}
}

{%
\theorembodyfont{\small}
\newtheorem*{Exemple}{Exemple}
}


{%
\theorembodyfont{\small}
\newtheorem{Exo}{Exercice}
}

{%
\theoremnumbering{Roman}
\theorembodyfont{\normalfont}
\newtheorem{Rem}{Remarque}
}




\begin{document}


\emph{\textbf{
\og Le travail est un trésor.\fg (Jean de La Fontaine)
}}


\begin{Exo}
	Soit $A\in GL_n(\R)$, comparer $^t(A^{-1})$ et $(^tA)^{-1}$. Que dire de l'inverse d'une matrice symétrique?
\end{Exo}

\begin{Exo}
	Soit $A\in M_n(\R)$, montrer que $^tAA$ est diagonalisable. On suppose que $A^tAA=I_n$, que vaut $A$?
\end{Exo}

\begin{Exo}
	Soit $A=\begin{pmatrix}
		a & b \\
		b & c
	\end{pmatrix}
	\in S_2(\R)$ de valeurs propres $\lambda \leqslant \mu$. Montrer que $\lambda\leqslant a\leqslant \mu$.
\end{Exo}

\begin{Exo}
	Soit $a$ unitaire dans $E$ euclidien, et $k$ réel. CNS sur $k$ pour que $f:x\in E\mapsto x+k\ps{x}{a}a$ soit un automorphisme orthogonal.
\end{Exo}

\begin{Exo}
	Soit $A=\begin{pmatrix}
		a & b & b \\
		b & a & b \\
		b & b & a
	\end{pmatrix}$, CNS sur $a,b$ réels pour que $A$ soit orthogonale. Etudier dans chaque cas la nature de l'endomorphisme de matrice $A$ dans la base canonique de $\R^3$.
	
\end{Exo}

%\begin{enumerate}
%	\item  On pose pour tout $A,B\in M_{p,q}(\Bbb{R}),\left\langle
%	A,B\right\rangle =Tr\left( ^{t}AB\right) .$ Soit $A\in M_{p}\left( \Bbb{R}%
%	\right) ,B\in M_{q}\left( \Bbb{R}\right) $ et $f:M\in M_{p,q}\left( \Bbb{R}%
%	\right) \mapsto f\left( M\right) =AM-MB.$ D\'{e}terminez $f^{*}.$
%	
%	\item  \textbf{Polyn\^{o}mes de Legendre:} on munit $\Bbb{R}\left[ X\right] $
%	du produit scalaire $\left\langle P,Q\right\rangle =\int_{-1}^{1}P\left(
%	t\right) Q\left( t\right) dt$ et on pose $P_{n}\left( x\right) =\left(
%	\left( x^{2}-1\right) ^{n}\right) ^{\left( n\right) }$.
%	
%	\begin{enumerate}
%		\item  Montrez que $P_{n}$ est un polyn\^{o}me de degr\'{e} $n$ admettant $n$
%		racines distinctes dans $\left] -1,1\right[ .$ Pr\'{e}cisez le coefficient
%		dominant.
%		
%		\item  Montrez que $\left( P_{n}\right) _{n}$ est une suite de \textbf{polyn%
%			\^{o}mes orthogonaux} et d\'{e}duisez-en une base orthonormale $\left(
%		L_{n}\right) _{n}$ pour $\left( \Bbb{R}\left[ X\right] ,\left\langle
%		,\right\rangle \right) $. Les $L_{n}$ sont appel\'{e}s polyn\^{o}mes de
%		Legendre.
%		
%		\item  Soit $E=\Bbb{R}_{n}\left[ X\right] $ muni du produit scalaire $%
%		\left\langle P,Q\right\rangle =\int_{-1}^{1}P\left( t\right) Q\left(
%		t\right) dt$ et $d$ l'endomorphisme de d\'{e}rivation. On cherche \`{a} d%
%		\'{e}terminer $d^{*}.$
%		
%		\begin{enumerate}
%			\item  R\'{e}pondre \`{a} la question pour $n=2.$
%			
%			\item  Soit $a\in \Bbb{R};$ montrez qu'il existe un unique polyn\^{o}me $%
%			H_{a}\in \Bbb{R}_{n}\left[ X\right] $ tel que $\forall P\in E,P\left(
%			a\right) =\left\langle H_{a},P\right\rangle .$ Calculer $H_{a}$ \`{a} l'aide
%			des polyn\^{o}mes de Legendre.
%			
%			\item  Exprimez $d^{*}$ \`{a} l'aide de $d,H_{-1}$ et $H_{1}.$
%			
%			\item  Traitez int\'{e}gralement le cas $n=5.$
%		\end{enumerate}
%	\end{enumerate}
%	
%	\item  Quelles sont les matrices de $M_{n}\left( \Bbb{R}\right) $ \'{e}gales 
%	\`{a} leur comatrice ?
\begin{Exo}
	Quelles sont les matrices orthogonales triangulaires supérieures?
\end{Exo}

\begin{Exo}
	Montrer que si $u$ est un endomorphisme symétrique d'un espace euclidien $E$ alors $\ker(u)$ et $\Ima(u)$ sont supplémentaires orthogonaux.
\end{Exo}

\begin{Exo}
	Quelles sont les matrices symétriques réelles nilpotentes?
\end{Exo}
	
\begin{Exo}
	Soit $u,v$ deux vecteurs libres de $E$ euclidien et $f:x\in E\mapsto \ps{x}{u}v+\ps{v}{x}u$. Montrer que $\phi$ est diagonalisable et préciser ses éléments propres.
\end{Exo}
	
  \begin{Exo}
		Soit $A=\left( a_{i,j}\right)_{i,j} \in O_{n},$ montrer que $\left|
	\sum_{i,j}a_{ij}\right| \leqslant n\leqslant \sum_{i,j}\lvert a_{i,j}\rvert\leqslant n\sqrt{n}$.
	\end{Exo}
	
	  \begin{Exo}
	  	Soit $E$ euclidien et $f\in O\left( E\right) ,$ montrez que: 
	\[
	f^{2}=-id_{E}\Leftrightarrow \left( \forall x\in E,f\left( x\right) \perp
	x\right) \Leftrightarrow \left( \forall \left( x,y\right) \in
	E^{2},\left\langle f\left( x\right) ,y\right\rangle =-\left\langle x,f\left(
	y\right) \right\rangle \right) 
	\]
	  \end{Exo}
	
%	\item  Soit $M\in GL_{n}\left( \Bbb{R}\right) ,$ montrez qu'il existe $P$
%	orthogonale et $T$ triangulaire sup\'{e}rieure, uniques, telles que $M=PT.$
%	(interpr\'{e}tez le proc\'{e}d\'{e} d'orthonormalisation de Schmidt). D\'{e}%
%	duisez-en que si $E$ est un ev euclidien de BON $\left(
%	e_{1},...,e_{n}\right) $ et $u_{1},...,u_{n}$ des vecteurs de $E$ alors $%
%	\left| \det_{\left( e_{i}\right) }\left( u_{j}\right) \right| \leq \left\|
%	u_{1}\right\| ...\left\| u_{n}\right\| $ (in\'{e}galit\'{e} d'Hadamard) et pr%
%	\'{e}cisez le cas d'\'{e}galit\'{e}.
	  \begin{Exo}
		Trouvez les sev de $\Bbb{R}^{3}$ stables par $A=\left( 
	\begin{array}{lll}
		1 & 1 & 0 \\ 
		-3 & -2 & 0 \\ 
		0 & 0 & 1
	\end{array}
	\right) .$
	\end{Exo}

\begin{Exo}
	Diagonaliser avec matrice de passage orthogonale $A=\begin{pmatrix}
		1 & -2 & 0 \\
		-2 & 0 & 2 \\
		0 & 2 & -1
	\end{pmatrix}$ et $B=\begin{pmatrix}
	1 & -1 & 1 \\
	-1 & 1 & -1 \\
	1 & -1 & 1
\end{pmatrix}$.

	
\end{Exo}
	
	  \begin{Exo}
	  	Diagonaliser en moins d'une minute la matrice: $\frac{1}{2}\left( 
	\begin{array}{cccc}
		1 & 1 & 1 & 1 \\ 
		1 & 1 & -1 & -1 \\ 
		1 & -1 & 1 & -1 \\ 
		1 & -1 & -1 & 1
	\end{array}
	\right) .$
	  \end{Exo}
	
	 \begin{Exo}
	 	 D\'{e}crire les isom\'{e}tries de $\Bbb{R}^{3}$ dont les matrices
	sont: $$\frac{1}{3}\left( 
	\begin{array}{ccc}
		2 & -2 & 1 \\ 
		2 & 1 & -2 \\ 
		1 & 2 & 2
	\end{array}
	\right) ,\frac{1}{3}\left( 
	\begin{array}{ccc}
		-1 & 2 & 2 \\ 
		2 & -1 & 2 \\ 
		2 & 2 & -1
	\end{array}
	\right) $$
	
	$$\frac{1}{7}\left( 
	\begin{array}{ccc}
		-2 & 6 & -3 \\ 
		6 & 3 & 2 \\ 
		-3 & 2 & 6
	\end{array}
	\right) ,\frac{1}{4}\left( 
	\begin{array}{ccc}
		3 & 1 & -\sqrt{6} \\ 
		1 & 3 & -\sqrt{6} \\ 
		-\sqrt{6} & \sqrt{6} & 2
	\end{array}
	\right) ,\frac{1}{4}\left( 
	\begin{array}{ccc}
		-3 & -1 & -\sqrt{6} \\ 
		1 & 3 & -\sqrt{6} \\ 
		-\sqrt{6} & \sqrt{6} & 2
	\end{array}
	\right) $$
	 \end{Exo}
	
%	\item  Soit $a_{1},...,a_{n}\in \Bbb{R}$ et $A=\left( a_{i}a_{j}\right)
%	_{1\leq i,j\leq n},$ montrez que $A$ est diagonalisable et determinez ses
%	\'{e}l\'{e}ments propres.
%	
%	\item  Soit $u,v$ deux endomorphismes sym\'{e}triques qui commutent, montrez
%	qu'il existe une BON commune de diagonalisation. Enoncez ce r\'{e}sultat
%	matriciellement.
	\newpage
	 \begin{Exo}
	 	Une matrice $M\in S_{n}$ est dite positive (resp. strictement
	positive) ssi $^{t}MXM\geq 0$ (resp. $>0$) pour tout $X$ de $\Bbb{R}^{n}.$
	On note alors $M\in S_{n}^{+}$ (resp. $S_{n}^{++}$).
	
	\begin{enumerate}
		\item  Montrez que $A\in S_{n}^{+}\iff  Sp\left( A\right) \subset 
		\R_{+}\Leftrightarrow \exists M\in M_n(\R) ;A=$ $%
		^{t}MM.$ Proposez et d\'{e}montrez un r\'{e}sultat analogue pour $%
		S_{n}^{++}. $
		
		\item  Montrez que $\left( 
		\begin{array}{lllll}
			2 & 1 & 0 & \cdots & 0 \\ 
			1 & 2 & 1 & \ddots & \vdots \\ 
			0 & 1 & \ddots & \ddots & 0 \\ 
			\vdots & \ddots & \ddots & \ddots & 1 \\ 
			0 & \cdots & 0 & 1 & 2
		\end{array}
		\right) \in S_{n}^{++}.$
		
		\item  Montrez que si $A\in S_{n}^{+}$ il existe une unique matrice $M\in
		S_{n}^{+}$ telle que $A=M^{2}$. On note $M=\sqrt{A}$. Calculer $M$ si $A=\left( 
		\begin{array}{ll}
			1 & 2 \\ 
			2 & 5
		\end{array}
		\right) $
		\item Pensez vous que  $\begin{pmatrix}
			0 & 1 \\
			0 & 0
		\end{pmatrix}$ admette une racine carrée? Et $\begin{pmatrix}
		1& 0 \\
		0 & -1
	\end{pmatrix}$?
	\end{enumerate}
	 \end{Exo}
 
%Pour l'unicité si $A=M^2=N^2$ on écrit $M=PD^tP,N=Q\Delta ^tQ$. $M^2=N^2$ équivaut à $RD^2=\Delta^2R$ et en calculant les termes $(i,j)$ de chaque produit on trouve $r_{i,j}\lambda_i^2=r_{i,j}\mu_j^2$, par suite soit $r_{i,j}$ est nul soit $\lambda _i=\mu_j$ d'où $RD=\Delta R$ ie $M=n$.
	
	  \begin{Exo}
	  	Soit $A\in M_{n}(\R)$, montrer que $\det \left(I_n+^{t}AA\right) >0.$
	  \end{Exo}
	
	\begin{Exo}
		Soit $A\in S_n(\R)$ de valeurs propres $\lambda_1,...,\lambda_n$, montrer que $\sum_{1\leqslant i,j\leqslant n}a_{i,j}^2=\sum_{i=1}^n \lambda_i^2$. En déduire les valeurs propres de la matrice $A=\begin{pmatrix}
			0 & 1 & \dots & 1 & 0 \\
			1 & 0 & \dots & 0 & 1 \\
			\vdots & \vdots &  & \vdots & \vdots \\
			1 & 0 & \dots & 0 & 1 \\
			0 & 1 & \dots & 1 & 0
		\end{pmatrix}$
		 
	\end{Exo}
	
%	\item  Soit $u\in L\left( E\right) ,$ v\'{e}rifier que $u^{*}\circ u$ est
%	d\'{e}fini positif. En d\'{e}duire que que $\left\| \left| u\right| \right\|
%	^{2}=\max \left\{ \left| \lambda \right| ,\lambda \in Sp\left( u^{*}\circ
%	u\right) \right\} .$ Que dire si $u$ est sym\'{e}trique ?
%	
%	\item  Soit $u\in L\left( E\right) ,$ montrer que $\left\| \left| u\right|
%	\right\| :=\sup_{\left\| x\right\| =1}\left\| u\left( x\right) \right\|
%	=\sup_{\left\| x\right\| =\left\| y\right\| =1}\left\langle u\left( x\right)
%	,y\right\rangle .$
	
%	\item  On suppose $A\in M_{n}\left( \Bbb{R}\right) ,A^{3}=^{t}AA,$ $A$
%	est-elle diagonalisable sur $\Bbb{R}$ , sur $\Bbb{C}$ ?
%	
	 \begin{Exo}
	 	 On munit $E=\Bbb{R}_{n}\left[ X\right] $ du produit scalaire $%
	\left\langle P,Q\right\rangle =\int_{-1}^{1}PQ$ et on note $u\in L\left(
	E\right) :$%
	\[
	u:P\longmapsto 2XP^{\prime }+\left( X^{2}-1\right) P^{\prime \prime } 
	\]
	Montrer que $u$ est diagonalisable et que des vecteurs propres associ\'{e}s
	\`{a} des valeurs propres distinctes sont orthogonaux. Pr\'{e}ciser ses
	\'{e}l\'{e}ments de r\'{e}ductions dans le cas $n=3.$
	 \end{Exo}
	
%	\item  \textbf{Endomorphisme associ\'{e} \`{a} une forme bilin\'{e}aire}
%	
%	\begin{enumerate}
%		\item  Soit $\phi $ une forme bilin\'{e}aire sur $E$ euclidien. Montrer
%		qu'il existe un unique endomorphisme (dit associ\'{e} \`{a} $\phi $) $u\in
%		L\left( E\right) $ tel que: 
%		\[
%		\forall x,y\in E,\phi \left( x,y\right) =\left\langle u\left( x\right)
%		,y\right\rangle 
%		\]
%		
%		\item  En d\'{e}duire une nouvelle introduction de l'adjoint d'un
%		endomorphisme.
%		
%		\item  Montrer que $\phi $ est sym\'{e}trique ssi $u$ l'est.
%		
%		\item  Montrer que si $\phi $ est une $fbs$ il existe une BON de $E$ qui
%		soit $\phi -$orthogonale (ie $i\neq j$ implique $\phi \left(
%		e_{i},e_{j}\right) =0$). Interpr\'{e}tation matricielle ?
%	\end{enumerate}
%	
%	\item  Application: soit $\phi $ une fbs sur $E$ et $q\left( x\right) =\phi
%	\left( x,x\right) $ la forme quadratique associ\'{e}e. Montrer que les
%	bornes de $\left\{ \frac{q\left( x\right) }{\left\| x\right\| ^{2}},x\neq
%	0\right\} =\left\{ q\left( x\right) ,\left\| x\right\| =1\right\} $ sont les
%	valeurs propres extr\^{e}males de l'endomorphisme associ\'{e} \`{a} $\phi .$
%	D\'{e}terminer $\sup \left\{ \frac{P\left( 0\right) P\left( 1\right) }{%
%		\int_{0}^{1}P^{2}},P\in \Bbb{R}_{2}\left[ X\right] \right\} $ (prendre $\phi
%	\left( P,Q\right) =\frac{1}{2}\left( P\left( 0\right) Q\left( 1\right)
%	+P\left( 1\right) Q\left( 0\right) \right) $)
%%\end{enumerate}

\begin{Exo}

Soit $u_1,...,u_p$ des endomorphismes symétriques de $E$ euclidien, vérifiant $\sum_{i=1}^{p}rg(u_i)=\dim E$ et $\forall x\in E,\ps{\sum_{i=1}^{p}u_i(x)}{x}=0$.
\begin{enumerate}
	\item
	Montrer que $\sum_{i=1}^{p}u_i=id_E$ (observer que cet endomorphisme est symétrique...)
	\item
	Montrer que $E=\bigoplus_{i=1}^p\Ima (u_i)$.
	\item
	Montrer que $u_i$ est le projecteur orthogonal sur $\Ima (u_i)$.
	\end{enumerate}
\end{Exo}

\begin{Exo}
	Soit $A\in M_n(\R)$ non nulle telle que $A^3+9A=0$.
	\begin{enumerate}
		\item
		Etudier la diagonalisabilité de $A$ sur $\R$ puis sur $\C$.
		\item
		Montrer que si $n$ est impair alors $A$ n'est pas inversible.
		\item
		Montrer que $A$ ne peut pas être symétrique.
	\end{enumerate}
\end{Exo}

\begin{Exo}
	Déterminer les $A$ dans $S_n(\R)$ telles que $A^3+2A^2+6A+5I_n=0_n$.
\end{Exo}

\begin{Exo}
		Soit $(E,\ps{}{})$ un espace euclidien de dimension $n>0$ et $a$ un vecteur de norme $1$. On pose pour $x$ dans $E$:
	$$f(x)=x+\ps{a}{x}a$$
	\begin{enumerate}
		\item
		Montrer que $f$ est un endomorphisme  symétrique de $E$.
		\item
		Déterminer $\ker(f-id_E)$et sa dimension? En déduire une première valeur propre de $f$ et sa multiplicité.
		\item
		Calculer $f(a)$, en déduire un second sev propre de $f$.
		\item
		Donner le polynôme caractéristique puis la trace de $f$.
	\end{enumerate}
\end{Exo}

\begin{Exo}
	Soit $A\in S_n^+(\R)$, montrer qu'il existe $B\in M_n(\R)$ telle que $A=\ ^tBB$.
\end{Exo}

\begin{Exo}
	Soit $u\in S(E)$ de valeurs propres $\lambda_1\leqslant \lambda_2\leqslant \dots \leqslant \lambda_n$. Montrer que pour tout $x$ de $E$ on a:
	$$\lambda_1 \norme{x}^2\leqslant \ps{x}{u(x)}\leqslant \lambda_n\norme{x}^2$$
\end{Exo}

\begin{Exo}
	Ecrire la matrice dans la base canonique $(i,j,k)$ de:
	\begin{enumerate}
		\item
		La symétrie orthogonale d'axe le plan d'équation $x-2y+z=0$.
		\item
		La rotation d'axe dirigé par $i-j+k$ et d'angle $\frac{\pi}{3}$.
	\end{enumerate}
\end{Exo}

\begin{Exo}
	Soit $A\in M_n(\R)$ et $M=\ ^tAA$.
	\begin{enumerate}
		\item
		Montrer que $M$ est symétrique de spectre inclus dans $\R_+$.
		\item
		Montrer que $A$ et $M$ ont même noyau, puis qu'elles ont même rang.
		\item
		Montrer enfin que $\Ima(M)=\Ima(\ ^tA)=(\ker(A))^{\perp}$.
	\end{enumerate}
\end{Exo}

\end{document}
