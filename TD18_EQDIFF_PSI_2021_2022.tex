  
\documentclass[12pt,a4paper]{article}







%\usepackage{fullpage}
%\usepackage{lmodern}
\usepackage[utf8]{inputenc}
\usepackage{amsmath}
\usepackage{amssymb}
\usepackage[french]{babel}
\usepackage[T1]{fontenc}
\usepackage{multicol}
\usepackage{textcomp}
%\usepackage{tikz}
\usepackage{fancyhdr}
\usepackage{geometry}
\addtolength{\hoffset}{-1cm}
 \addtolength{\voffset}{-2cm}
 \addtolength{\textwidth}{4cm}
\addtolength{\textheight}{4cm}


\pagestyle{fancyplain}
\lhead{\textit{CSI2B-PSI TD 18}}
\chead{\textsc{Equations différentielles linéaires}}
\rhead{\textit{2021-2022}} 


\everymath{\displaystyle}






%ensembles de nombres------------------------------------------------




\DeclareMathOperator{\Q}{\mathbb{Q}}
\DeclareMathOperator{\K}{\mathbb{K}}
\DeclareMathOperator{\R}{\mathbb{R}}
\DeclareMathOperator{\Z}{\mathbb{Z}}
\DeclareMathOperator{\N}{\mathbb{N}}
\DeclareMathOperator{\C}{\mathbb{C}}
\DeclareMathOperator{\U}{\mathbb{U}}




%applications relations----------------------------------------------




\DeclareMathOperator{\Id}{\mathrm{Id}}
\DeclareMathOperator{\Class}{\mathcal{C}}
\DeclareMathOperator{\DL}{\mathrm{DL}}

\newcommand{\lci}{l.c.i. }
\newcommand{\set}[2]{
\left\{
\begin{array}{ccc}
#1 & / & #2
\end{array}
\right\}
}
\newcommand{\appli}[5]
{
%E,F,f,x,y
\left\{
\begin{array}{ccc}
#1 & \overset{#3}{\longrightarrow} & #2 \\
#4 & \longmapsto & #5 
\end{array}
\right.
}
\newcommand{\applilight}[3]
{
%E,F,f
#3 : #1  \longrightarrow  #2
}
\newcommand{\reciproque}[1]{
#1^{*} 
}





%fonctions usuelles--------------------------------------------------




\DeclareMathOperator{\ch}{\mathrm{ch}}
\DeclareMathOperator{\sh}{\mathrm{sh}}
\DeclareMathOperator{\tah}{\mathrm{th}}
\DeclareMathOperator{\sgn}{\mathrm{sgn}}
\DeclareMathOperator{\cotan}{\mathrm{cotan}}

\newcommand{\floor}[1]{\left\lfloor #1 \right\rfloor}
\newcommand{\equi}[3]
{
% f,g,a
#1\underset{#3}{\sim}#2
}
\newcommand{\bigo}[3]
{
% f,g,a
#1\underset{#3}{=}\mathcal{O}\left(#2\right)
}
\newcommand{\plusbigo}[4]
{
% f,g,h,a : f =_a g+O(h)
#1\underset{#4}{=}#2 + \mathcal{O}\left(#3\right)
}
\newcommand{\smallo}[3]
{
% f,g,a
#1\underset{#3}{=}\mathbf{o}\left(#2\right)
}
\newcommand{\plussmallo}[4]
{
% f,g,h,a : f =_a g+o(h)
#1\underset{#4}{=}#2 + \mathbf{o}\left(#3\right)
}





%complexes--------------------------------------------





\DeclareMathOperator{\Real}{\mathcal{R}e}
\DeclareMathOperator{\Imaginary}{\mathcal{I}m}
\DeclareMathOperator{\I}{\mathbf{i}}





%polynomes----------------------------------------------------





\DeclareMathOperator{\Xd}{\mathrm{X}}
\DeclareMathOperator{\val}{\mathrm{val}}
\DeclareMathOperator{\X}{\mathrm{X}}
%\DeclareMathOperator{\deg}{\mathrm{deg}}





%algebre lineaire---------------------------------------------





\DeclareMathOperator{\Diag}{\mathrm{Diag}}
\DeclareMathOperator{\Vect}{\mathrm{Vect}}
\DeclareMathOperator{\Ima}{\mathrm{Im}}
\DeclareMathOperator{\Ker}{\mathrm{Ker}}
\DeclareMathOperator{\GL}{\mathrm{GL}}
\DeclareMathOperator{\tr}{\mathrm{tr}}
\DeclareMathOperator{\Lcal}{\mathcal{L}}
\DeclareMathOperator{\Mat}{\mathrm{Mat}}
\DeclareMathOperator{\Com}{\mathrm{Com}}
\DeclareMathOperator{\M}{\mathcal{M}}
\DeclareMathOperator{\rg}{\mathrm{rg}}
\newcommand{\tribu}{\mathcal{A}}
\renewcommand{\Pr}{\mathbb{P}} 
\newcommand{\ev}{espace vectoriel }
\newcommand{\evs}{espaces vectoriels }
\newcommand{\transpo}[1]{
 \mathstrut^t\! #1
}





%arithmetique-------------------------------------------------





\DeclareMathOperator{\pgcd}{pgcd}

\newcommand{\modulobis}[3]{
% x,y,alpha
#1\equiv #2\; \left(\text{mod}\;#3\right)
}
\newcommand{\modulo}[3]{
% x,y,alpha
#1\equiv #2\; \left[#3\right]
}





%calcul differentiel et integral----------------------------




\DeclareMathOperator{\der}{\mathrm{d}}




%denombrement probas-----------------------------------------




\DeclareMathOperator{\Card}{\mathrm{Card}}
\DeclareMathOperator{\PR}{\mathbf{P}}
\DeclareMathOperator{\E}{\mathrm{e}}
\DeclareMathOperator{\V}{\mathbf{V}}
\DeclareMathOperator{\COV}{\mathbf{Cov}}

\newcommand{\VA}{v.a. }
\newcommand{\VAR}{v.a.r. }
\newcommand{\bin}[2]{\binom{#1}{#2}}




%geometrie---------------------------------------------------




\newcommand{\fle}[1]{
\overrightarrow{#1}
}
\newcommand{\ang}[2]{
\widehat{
\left(
#1 , #2
\right)
}
}
\newcommand{\scal}[2]{
\left( #1 |  #2 \right)
}




%recurrence--------------------------------------------------




\newcommand{\recurr}[4]{
% u, n_0, u(n_0), u(n+1)
\left\{
\begin{array}{cl}
& #1_{#2}=#3 \\
\forall n\geqslant #2 \quad &    #1_{n+1}=#4   
\end{array}
\right.
}
\newcommand{\recurrdouble}[6]{
% u, n_0,n_1, u(n_0), u(n_1), u(n+2)
\left\{
\begin{array}{cl}
& #1_{#2}=#4 \qquad #1_{#3}=#5\\
\forall n\geqslant #2 \quad &    #1_{n+2}=#6   
\end{array}
\right.
}




%divers----------------------------------------------------




\DeclareMathOperator{\lcro}{\textnormal{\textlbrackdbl}}
\DeclareMathOperator{\rcro}{\textnormal{\textrbrackdbl}}

\newcommand{\ssi}{ssi }
\newcommand{\dis}{\displaystyle}


%--------------------------------------------------------







%--------------------------------------------------------------------
%-------------DOCUMENT-----------------------------------------------
%--------------------------------------------------------------------


\begin{document}

\emph{\textbf{
La guerre, c'est le massacre de gens qui ne se connaissent pas, au profit de gens qui se connaissent et ne se massacrent pas. (Paul Valéry)
}}

%\emph{\textbf{
%		\emph{\textbf{
%				Quel que soit le temps passé à faire des mathématiques, ce n'est jamais du temps perdu. (Cédric Villani)
%		}
%}}

\begin{enumerate}
%\item
%
%Etudier $x\left( x+1\right) y^{\prime }+y=\arctan x.$ On s'attachera
%au probl\'{e}me de raccordement.

\item Soient $a,b$ continues sur $\R$ et impaires. Montrer que les solutions de $y'+ay=b$ sont paires.

\item
 R\'{e}soudre $3xy^{\prime }-4y=x$ sur $\R_-^*$ et $\R_+^*$. Y-a-t'il des solutions définies sur $\R$?
\item R\'{e}soudre $y^{\prime \prime }-y^{\prime }-e^{2x}y=e^{3x}$ en
effectuant le changement de variable $u=e^{x}.$

\item D\'{e}terminer la fonction $f$ continue sur $\mathbb{R}$ vérifiant pour tout $x$ réel $:f\left( x\right) -\int_{0}^{x}tf\left( t\right) dt=1$.
\item On se propose de r\'{e}soudre $\left( E\right) :f^{\prime \prime
}\left( -x\right) +f\left( x\right) =x+\cos x.$

Soit $\phi :x\mapsto f\left( x\right) +f\left( -x\right) $ et $\psi
:x\mapsto f\left( x\right) -f\left( -x\right) .$ Trouver des ED simples v%
\'{e}rifi\'{e}es par $\phi $ et $\psi .$ En d\'{e}duire les solutions de $%
\left( E\right) .$
\item R\'{e}soudre sur $\mathbb{R}$ les \'{e}quations $xy^{\prime }+2y=\frac{%
	x^{2}}{1+x^{2}}$ et $xy^{\prime }+y=e^{x}.$
\item Trouver les solutions polyn\^{o}miales de:%
$
	\left( E\right) :x^{2}y^{\prime \prime }\left( x\right) -xy^{\prime }\left(
	x\right) -3y\left( x\right) =0
$
R\'{e}soudre ensuite sur $\mathbb{R}_{+}^{\ast }$ et $\mathbb{R}_{-}^{\ast }.$
Dimension de l'espace des solutions sur $\mathbb{R}?$

\item \textbf{Etude qualitative}
Soit $q$ une fonction continue int\'{e}grable sur $\mathbb{R}_{+}$ et
soit $\left( E\right) :y^{\prime \prime }+qy=0.$

\begin{enumerate}
	\item Montrer que si $f$ a une limite en $+\infty $ et si $\int_{0}^{+\infty
	}f$ converge alors cette limite est nulle.
	
	\item Montrer que, si $f$ est une solution born\'{e}e de l'\'{e}quation alors $\lim_{\infty }f^{\prime }=0$.
	\item Soit $f,g$ deux solutions born\'{e}es et $w=f^{\prime }g-fg^{\prime }$
	leur wronskien. Calculer $w^{\prime }$ et en d\'{e}duire que $f$ et $g$ sont
	li\'{e}es.
	\item En déduire que $(E)$ admet nécessairement une solution non bornée sur $\R_+$.
\end{enumerate}

%\item \textbf{Etude qualitative}
%Soit $p:\mathbb R\to\mathbb R_+$ une fonction continue non identiquement nulle. 
%On se propose de démontrer que toutes les solutions de l'équation différentielle $y''(x)+p(x)y(x)=0$
%s'annulent. Pour cela, on raisonne par l'absurde et on suppose que $f$ est une solution ne s'annulant pas.
%\begin{enumerate}
%	\item
%	Justifier que $f$ est de signe constant. Dans la suite, quitte à changer $f$ en $-f$, on supposera $f>0$.
%	\item
%	Soit $a\in\mathbb R$ quelconque. Justifier que la courbe représentative de $f$ est en-dessous de sa tangente en $(a,f(a))$.
%	\item
%	En déduire que $f'(a)=0$ puis conclure.
%\end{enumerate}
\item 
\begin{enumerate}
	\item
	Résoudre sur $\R:y'''-2y''-y'+2y=0$ via un système différentiel d'ordre $1$.
	\item Résoudre sur $\R:\begin{cases}
		x''=y\\y'=-2x+x'+2y
	\end{cases}$
\end{enumerate}

\item 	Chercher les solutions développables en séries entières de:
$$(E):(1+x^2)y''(x)+4xy'(x)+2y(x)=0$$ En déduire toutes les solutions de $(E)$.

\item
On considère le système différentiel (S) définie par
$$
\left\{\begin{array}{l}
	x^{\prime}=x+2 y-z \\
	y^{\prime}=2 x+4 y-2 z \\
	z^{\prime}=-x-2 y+z
\end{array} \quad\right. \text { avec } \quad\left\{\begin{array}{l}
	x(0)=1 \\
	y(0)=0 \\
	z(0)=0
\end{array}\right.
$$
Justifier que le système $(S)$ admet une unique solution $f=(x, y, z)$ sur $\mathbb{R}$ puis montrer que $f(\mathbb{R})$ est inclus dans le plan d'équation $x+z=1$.

%\item Justifier que si $a,b$ sont des constantes et si $P$ est un polynôme alors $y''+ay'+by=\E^{\lambda t}P(t)$ admet une solution particulière de la forme $t\mapsto Q(t)\E^{\lambda t}$ avec $\deg(Q)=\deg(P)$ augmenté de la multiplicité de $\lambda$ comme racine de $X^2+aX+b$. (faire le changement de fonction inconnue $y(t)=z(t)\E^{\lambda t})$.

\item \textbf{Réflechir avant de se lancer...}
Résoudre $X'=AX$ avec $A=\begin{pmatrix}
	4 & 0 & 2 & 0 \\
	0 & 4 & 0 & 2 \\
	2 & 0 & 4 & 0 \\
	0 & 2 & 0 & 4
\end{pmatrix}$
\item Résoudre $\begin{cases}
x''=3x+y\\y''=2x+2y
\end{cases}$

\item
Trouver les solutions développables en série entière de $2xy'+y=\frac{2}{1-x}$.
\item Résoudre $y''-2y'+5y=-4\E^{-x}\cos(x)$.
\item Résoudre $y''-2y'+y=x,y(0)=y'(0)=0$. (chercher une solution particulière simple...)
\item
On considère sur $\mathbb{R}$ l'équation différentielle:
$
(t+1)^{2} y^{\prime \prime}-2(t+1) y^{\prime}+2 y=0
$
\begin{enumerate}
	

\item Déterminer les solutions polynomiales de $(E)$
\item En déduire les solutions de $(E)$ sur les intervalles $]-\infty,-1[$ et $]-1,+\infty[$.
\item En déduire les solutions de $(E)$ sur $\mathbb{R}$.
\end{enumerate}

%\item Déterminer une équation différentielle linéaire du second ordre admettant $x \mapsto \mathrm{e}^{x^{2}}$ et $x \mapsto \mathrm{e}^{-x^{2}}$ comme solution.
\item En posant $z=\left(1+\mathrm{e}^{x}\right) y$, résoudre sur $\mathbb{R}$ l'équation différentielle
$$
\left(1+\mathrm{e}^{x}\right) y^{\prime \prime}+2 \mathrm{e}^{x} y^{\prime}+\left(2 \mathrm{e}^{x}+1\right) y=x \mathrm{e}^{x}
$$
\item En posant $z=x y$, résoudre sur $\mathbb{R}$ l'équation différentielle
$
x y^{\prime \prime}+2(x+1) y^{\prime}+(x+2) y=0
$.
\item  En posant $z=y-y^{\prime}$, résoudre sur $\mathbb{R}$ l'équation différentielle 
$
x y^{\prime \prime}-(1+x) y^{\prime}+y=1
$.

\item
Résoudre  $X^{\prime}=A X+B$ où
$
A(t)=\left(\begin{array}{cc}
	1+t & t \\
	-t & 1-t
\end{array}\right)  \text { et }  B(t)=\left(\begin{array}{c}
	-\mathrm{e}^{t} \\
	\mathrm{e}^{t}
\end{array}\right)
$.
\item Déterminer les fonctions $f, g: \mathbb{R} \rightarrow \mathbb{R}$ continues telles que
$$
\forall x \in \mathbb{R}, \quad f(x)=\int_{0}^{x} g(t) \mathrm{d} t \quad \text { et } \quad g(x)=\int_{0}^{x} f(t) \mathrm{d} t
$$

\item

Résoudre les systèmes suivants.
\begin{multicols}{3}
\begin{enumerate}
	\item
	$\left\{\begin{array}{l}
		x^{\prime}=4 x-2 y \\
		y^{\prime}=x+y
	\end{array} \quad\right.$
	\item
	$\left\{\begin{array}{l}
		x^{\prime}=4 x-2 y \\
		y^{\prime}=x+y
	\end{array} \quad\right.$
	\item
	$\left\{\begin{array}{l}
	x^{\prime}=x-4 y \\
	y^{\prime}=x-3 y
\end{array} \quad\right.$
	\item
	$\left\{\begin{array}{l}
		x^{\prime}=-x+3 y \\
		y^{\prime}=-2 x+4 y
	\end{array}\right.$
\end{enumerate}
\end{multicols}
\item Résoudre $X'=AX$ avec:
$
(i) A=\left(\begin{array}{ccc}
	1 & 1 & 0 \\
	-1 & 2 & 1 \\
	1 & 0 & 1
\end{array}\right)\quad(ii) A=\left(\begin{array}{ccc}
	0 & 1 & -1 \\
	1 & 4 & -2 \\
	2 & 6 & -3
\end{array}\right) \text { . }
$
\item

		R\'{e}soudre le syst\'{e}me $\left\{ 
\begin{array}{l}
	x^{\prime }=\frac{x}{2}+\frac{e^{t}}{2}y  \\ 
	y^{\prime }=\frac{e^{-t}}{2}x-\frac{y}{2}+e^{t}%
\end{array}%
\right. $ en remarquant que $\left( 
\begin{array}{c}
	e^{t} \\ 
	1%
\end{array}%
\right) $ est solution du syst\'{e}me homog\'{e}ne.

%		
\item
Montrer que les sevs propres de $M\left( t\right) =\left( 
\begin{array}{cc}
	t-2 & 1 \\ 
	-3 & 2+t%
\end{array}%
\right) $ sont ind\'{e}pendants de $t.$ R\'{e}soudre ensuite le syst\`{e}me:%
\begin{equation*}
	\left\{ 
	\begin{array}{l}
		x^{\prime }\left( t\right) =\left( t-2\right) x\left( t\right) +y\left(
		t\right)  \\ 
		y^{\prime }\left( t\right) =-3x\left( t\right) +\left( 2+t\right) y\left(
		t\right) 
	\end{array}%
	\right. 
\end{equation*}
\item On consid\`ere l'\'equation diff\'erentielle :$y''+6y'+9y=d(x) \quad (E)$
\begin{enumerate}
	\item R\'esoudre l'\'equation diff\'erentielle
	homog\`ene associ\'ee \`a $(E)$.
	\item Trouver une solution particuli\`ere de $(E)$
	lorque, respectivement, on pose :
	$ d(x) = \E^{-3x}$ et $d(x)=\cos x.$
	\item Donner la forme g\'en\'erale des solutions de $(E)$
	lorsque :
	$d(x) = 2\E^{-3x}+50\cos x.$
\end{enumerate}

\end{enumerate}
\end{document}
