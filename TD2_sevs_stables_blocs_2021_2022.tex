  
\documentclass[12pt,a4paper]{article}
% Engine-specific settings
% Detect pdftex/xetex/luatex, and load appropriate font packages.
% This is inspired by the approach in the iftex package.
% pdftex:

\usepackage[T1]{fontenc}
\usepackage[utf8]{inputenc}
\usepackage[french]{babel}
\frenchbsetup{StandardLists=true}
\usepackage{enumitem}
\usepackage{systeme}
\usepackage{amsmath,amssymb}
\usepackage[thmmarks,amsmath]{ntheorem}
\usepackage[colorlinks=true]{hyperref}
\usepackage{fullpage}
%\usepackage{eulervm}
\usepackage{graphicx}
\usepackage{array}
\usepackage{multicol}
\usepackage[makestderr]{pythontex}
\restartpythontexsession{\thesection}
\usepackage{geometry}
\geometry{tmargin=1.5cm,bmargin=1.5cm,lmargin=1.5cm,rmargin=1.5cm,headheight=12cm}
\usepackage{array}
\usepackage[svgnames,table]{xcolor}
\usepackage[tikz]{ bclogo}
\usepackage{pifont}
\usepackage{url}
\urlstyle{same}
\usepackage{pifont}
\usepackage{multicol}
\usepackage{diagbox} %oblique dans les tableaux
%\usepackage[framemethod=TikZ]{mdframed}
\usepackage{fancyhdr}
\pagestyle{fancyplain}
\setlength\headsep{2mm}

\lhead{\textit{CSI2B-PSI TD2}}
\chead{\textsc{sous-espace stables, blocs}}
\rhead{\textit{2021-2022}} 


\renewcommand*{\thefootnote}{\fnsymbol{footnote}}
\newcommand{\un}{(u_n)_n}
\newcommand{\R}{\mathbb{R}}
\newcommand{\C}{\mathbb{C}}
\newcommand{\Q}{\mathbb{Q}}
\newcommand{\Z}{\mathbb{Z}}
\newcommand{\N}{\mathbb{N}}
\newcommand{\K}{\mathbb{K} }

\newcommand{\diag}{\mathrm{diag}}
\renewcommand{\Re}{\mathcal{R}e}
\renewcommand{\Im}{\mathcal{I}m}
\DeclareMathOperator{\Ima }{Im}
\DeclareMathOperator{\vect}{Vect}
\DeclareMathOperator{\tr}{Trace}
\newcommand{\conj}[1]{\overline{#1}}

{%
\theoremstyle{break}
\theoremprework{%
\rule{0.5\linewidth}{0.3pt}}
\theorempostwork{\hfill%
\rule{0.5\linewidth}{0.3pt}}
\theoremheaderfont{\scshape}
\theoremseparator{ ---}
\newtheorem{Prop}{%
\textcolor{blue}{Proposition}}[section]
}

{%
\theoremstyle{break}
\theoremprework{%
\rule{0.6\linewidth}{0.5pt}}
\theorempostwork{\hfill%
\rule{0.6\linewidth}{0.5pt}}
\theoremheaderfont{\scshape}
\newtheorem{Theo}{%
\textcolor{red}{Théorème}}[section]
}


{%
\theoremheaderfont{\sffamily\bfseries}
\theorembodyfont{\sffamily}
\newtheorem{Def}{%
\textcolor{green}{Définition}}[section]
}

{%
\theorembodyfont{\small}
\theoremsymbol{$\square$}
\newtheorem*{Dem}{Démonstration}
}

{%
\theorembodyfont{\small}
\newtheorem*{Exemple}{Exemple}
}


{%
\theorembodyfont{\small}
\newtheorem{Exo}{Exercice}
}

{%
\theoremnumbering{Roman}
\theorembodyfont{\normalfont}
\newtheorem{Rem}{Remarque}
}




\begin{document}

\emph{\textbf{“Tout est difficile avant d'être simple.
		” (Thomas Feller)}}
%
%\begin{center}
%\textsc{Les exercices ... sont à chercher pour vendredi}   
%\end{center}

\begin{Exo}
	Soit $F=\vect(e_1,...,e_p)$ un sev du $\K$-ev $E$ et $f\in L(E)$. Montrer que $F$ est stable par $f$ ssi pour tout $i\in\{1,...,p\}$ on a $f(e_i)\in F$.
\end{Exo}

\begin{Exo}
	Montrer que si $u$ laisse stable toutes les droites vectorielles de $E$ alors $u$ est une homothétie vectorielle. Complément: soit $k$ dans $\N^*$, montrer que si $u$ laisse stable tous les sevs de $E$ de dimension $k$ alors $u$ est une homothétie vectorielle.
\end{Exo}

\begin{Exo}
Déterminer les sevs stables par $u\in L(\R^2)$ de matrice dans la base canonique $A=\begin{pmatrix}
0 & 1 \\
-1 & 0
\end{pmatrix}$. Idem si $A=\begin{pmatrix}
1 & 1 \\
0 & 1
\end{pmatrix}$.
\end{Exo}

\begin{Exo}
	Montrer que si $u$ et $v$ de $L(E)$ commutent alors $\ker u$ et $\Ima u$ sont stables par $v$.
\end{Exo}

\begin{Exo}
	Soit $F$ sev du $\K$-ev $E$,$u\in L(E)$ et $\lambda \in \K$.
	\begin{enumerate}
		\item
		Montrer que $\ker(u-\lambda id_E)$ est $u$-stable et préciser l'endomorphisme induit.
		\item
		Montrer que $F$ est $u$-stable ssi il est stable par $u-\lambda id_E$.
		\item
		Montrer que si $F$ est $u$-stable alors il est stable par $u^2$. Réciproque?
	\end{enumerate}
\end{Exo}

\begin{Exo}
	Montrer que si $F$ sev de $E$ est stable par $f$ et $g$ de $L(E)$ alors il l'est   par $f+g$ et par $f\circ g$. Retrouver que le produit de deux matrices triangulaires supérieures l'est encore.
\end{Exo}



\begin{Exo}
	Soit $p$ un projecteur de $E$ et $u$ dans $L(E)$. Montrer que $u$ et $p$ commutent ssi $\ker p$ et $\Ima p$ sont stables par $u$.
\end{Exo}

\begin{Exo}
	Déterminer les sous espaces de $\R[X]$ stables par l'endomorphisme de dérivation.
\end{Exo}



\begin{Exo}
	Soit $D:P\mapsto P'$ l'opérateur de dérivation sur $\R[X]$. Supposons qu'il existe un endomorphisme $\Delta$ de $\R[X]$ tel que $D=\Delta^2$.
	\begin{enumerate}
		\item
		Montrer que  $\R_1[X]$ est stable par $\Delta$.
		\item Donner la matrice dans la base canonique de $\R_1[X]$ de l'endomorphisme induit par $D$ sur $\R_1[X]$.
		\item Conclure quand à l'existence de $\Delta$.
	\end{enumerate}
\end{Exo}

\begin{Exo}
	Soit $E$ un $\K$-ev,$u\in L(E)$ et $\lambda \in \K$. Montrer que si un hyperplan $H$ de $E$ contient $\Ima(u-\lambda id_E)$ alors $H$ est stable par $u$.
\end{Exo}

\begin{Exo}
	Montrer que si un sev $F$ de $E$ (avec $\dim(E)$ finie) est stable par $u\in L(E)$ alors il est stable par $u^{-1}$. Rejustifier le fait que l'inverse d'une matrice en damier est encore en damier.
\end{Exo}
%
\begin{Exo}
	Montrer qu'un endomorphisme d'un $\C$-ev de dimension finie admet toujours une droite stable. Ce résultat subsiste-t-il dans un $\R$-ev?
\end{Exo}


\begin{Exo}
	Soit $u\in L(\R^4)$ de matrice dans la base canonique $A$ la matrice Attila.
	\begin{enumerate}
		\item
		Déterminer le noyau et l'image de $A$
		\item
		En déduire deux sevs $F,G$ de dimension $1$ et $3$ stables par $u$ et supplémentaires.
		\item
		Préciser $u_F,u_G$ et donner la matrice dans une base adaptée.
	\end{enumerate}
\end{Exo}
\newpage
\begin{Exo}
	Soit $u\in L(\R^3)$ et $B=(e_1,e_2,e_3)$ la base canonique de $\R^3$. On suppose que $u(e_1)=e_2,u(e_2)=e_3,u(e_3)=e_1$.
	\begin{enumerate}
		\item
		Former la matrice $A$ de $u$ dans $B$.
		\item
		Montrer qu'il n'existe qu'une seule droite $D$ stable par $u$. En donner un vecteur de base et préciser $u_D$ l'endomorphisme induit par $u$ sur $D$.
		\item
		Vérifier que $P=\vect(e_1-e_2,e_1-e_3)$ est stable par $u$.
		\item Justifier que $P$ est le seul plan stable par $u$, montrer ensuite que $F$ et $P$ sont supplémentaires et donner la matrice de $u$ dans une base adaptée à cette somme directe.
	\end{enumerate}
\end{Exo}

\begin{Exo}
	Soit $u:P\in \R[X]\mapsto XP$.
	\begin{enumerate}
		\item
		Montrer que $u$ est endomorphisme de $\R[X]$.
		\item
		$u$ admet-il des sevs stables de dimension finie?
		\item
		Soit $F=\{P\in\R[X],P(1)=0\}$, montrer que $F$ est un sev de $\R[X]$. Est-il stable par $u$?
	\end{enumerate}
\end{Exo}

\begin{Exo}
	On rappelle que toute matrice de rang $r$ est équivalente à la matrice $J_r$ dont tous les coefficients sont nuls sauf les $a_{1,1},a_{2,2},...,a_{r,r}$ qui valent $1$. Déterminer alors le rang de $\begin{pmatrix}
		A_1 & 0 \\
		0 & A_2
	\end{pmatrix}$ en fonction de celui de $A_1$ et de celui de $A_2$.
	
\end{Exo}

\begin{Exo}
	On suppose $A,B$ inversibles, montrer que $\begin{pmatrix}
		0 & A \\
		B & 0
	\end{pmatrix}$ l'est et expliciter son inverse.
\end{Exo}

\begin{Exo}
	Soit $A,B,C,D$ dans $M_n(\K)$ telles que $D$ soit inversible et $CD=DC$. Montrer que:
	\[\det\begin{pmatrix}
		A & B \\
		C & D
	\end{pmatrix}=\det(AD-BC)
	\]
	On pourra commencer avec $n=1$...
\end{Exo}

\end{document}
