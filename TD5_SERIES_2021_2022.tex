  
\documentclass[12pt,a4paper]{article}
% Engine-specific settings
% Detect pdftex/xetex/luatex, and load appropriate font packages.
% This is inspired by the approach in the iftex package.
% pdftex:

\usepackage[T1]{fontenc}
\usepackage[utf8]{inputenc}
\usepackage[french]{babel}
\frenchbsetup{StandardLists=true}
\usepackage{enumitem}
\usepackage{systeme}
\usepackage{amsmath,amssymb}
\usepackage[thmmarks,amsmath]{ntheorem}
\usepackage[colorlinks=true]{hyperref}
\usepackage{fullpage}
%\usepackage{eulervm}
\usepackage{graphicx}
\usepackage{array}
\usepackage{multicol}
\usepackage{geometry}
\geometry{tmargin=1.5cm,bmargin=1.5cm,lmargin=1.5cm,rmargin=1.5cm,headheight=12cm,headsep=5pt}
\usepackage{array}
\usepackage[svgnames,table]{xcolor}
\usepackage[tikz]{ bclogo}
\usepackage{pifont}
\usepackage{url}
\urlstyle{same}
\usepackage{pifont}
\usepackage{multicol}
\usepackage{diagbox} %oblique dans les tableaux
%\usepackage[framemethod=TikZ]{mdframed}
\usepackage{fancyhdr}
\pagestyle{fancyplain}
\lhead{\textit{CSI2B-PSI TD5}}
\chead{\textsc{Séries numériques}}
\rhead{\textit{2021-2022}} 


\renewcommand*{\thefootnote}{\fnsymbol{footnote}}
\newcommand{\un}{(u_n)_n}
\newcommand{\R}{\mathbb{R}}
\newcommand{\C}{\mathbb{C}}
\newcommand{\Q}{\mathbb{Q}}
\newcommand{\Z}{\mathbb{Z}}
\newcommand{\N}{\mathbb{N}}
\newcommand{\K}{\mathbb{K} }
\newcommand{\I}{\mathbf{i}}
\renewcommand{\Re}{\mathcal{R}e}
\renewcommand{\Im}{\mathcal{I}m}
\DeclareMathOperator{\Ima }{Im}
\newcommand{\diff}{\mathop{}\mathopen{}\mathrm{d}}
\everymath{\displaystyle}
\newcommand{\conj}[1]{\overline{#1}}
%\renewcommand{\headrulewidth}{0pt}%pas de ligne en haut
{%
\theoremstyle{break}
\theoremprework{%
\rule{0.5\linewidth}{0.3pt}}
\theorempostwork{\hfill%
\rule{0.5\linewidth}{0.3pt}}
\theoremheaderfont{\scshape}
\theoremseparator{ ---}
\newtheorem{Prop}{%
\textcolor{blue}{Proposition}}[section]
}

{%
\theoremstyle{break}
\theoremprework{%
\rule{0.6\linewidth}{0.5pt}}
\theorempostwork{\hfill%
\rule{0.6\linewidth}{0.5pt}}
\theoremheaderfont{\scshape}
\newtheorem{Theo}{%
\textcolor{red}{Théorème}}[section]
}


{%
\theoremheaderfont{\sffamily\bfseries}
\theorembodyfont{\sffamily}
\newtheorem{Def}{%
\textcolor{green}{Définition}}[section]
}

{%
\theorembodyfont{\small}
\theoremsymbol{$\square$}
\newtheorem*{Dem}{Démonstration}
}

{%
\theorembodyfont{\small}
\newtheorem*{Exemple}{Exemple}
}


{%
\theorembodyfont{\small}
\newtheorem{Exo}{Exercice}
}

{%
\theoremnumbering{Roman}
\theorembodyfont{\normalfont}
\newtheorem{Rem}{Remarque}
}




\begin{document}
\textit{Point n'est besoin d'espérer pour entreprendre, ni de réussir pour persévérer. (Guillaume d'Orange)}

\begin{Exo}
Convergence et somme de $$\sum_{n\geqslant 0} \frac{2^{n-2}}{3^{n+2}}\quad \sum_{n\geqslant 0}\frac{2^n+1}{4^n}\quad\sum_{n\geqslant 0}\frac{(1+\I)^n}{(1+2\I)^n}$$
\end{Exo}

\begin{Exo}
Montrer que $\sum_n \cos n$ diverge.
\end{Exo}

%\begin{Exo}[Equivalent de $H_n$]
%\`A l'aide d'une comparaison série-intégrale démontrer que $H_n\sim\ln n$ ou l'on a noté $H_n=\displaystyle\sum\limits_{k=1}^{n}\frac{1}{k}$. Retrouver la nature de la série dite \textbf{harmonique} $\sum_n\frac{1}{n}$.
%\end{Exo}

%\begin{Exo}[Un calcul de somme]\ 
%	
%	\begin{enumerate}
%		\item
%		En observant que pour tout entier $k>0$ on a $\frac{1}{k}=\int_{0}^{1}t^{k-1}dt$, montrer que:
%		$$\sum_{k=1}^{n}\frac{(-1)^k}{k}=\int_{0}^{1}\frac{1-(-t)^n}{1+t}dt$$
%		\item En déduire la convergence et la somme de la série $\sum_{n>0}\frac{(-1)^n}{n}$.
%		\item Une série convergente est-elle absolument convergente?
%	\end{enumerate}
%	
%\end{Exo}

\begin{Exo}[Un calcul de somme]\ 
	
	\begin{enumerate}
		\item
		En observant que pour tout entier $k \geqslant 0$ on a $\frac{1}{2k+1}=\int_{0}^{1}t^{2k}dt$, montrer que:
		$$\sum_{k=0}^{n}\frac{(-1)^k}{2k+1}=\int_{0}^{1}\frac{1-(-t^2)^{n+1}}{1+t^2}dt$$
		\item En déduire la convergence et la somme de la série $\sum_{n\geqslant 0}\frac{(-1)^n}{2n+1}$.
		\item Une série convergente est-elle absolument convergente?
	\end{enumerate}
	
\end{Exo}

\begin{Exo}
	Convergence et somme de $\sum_n\left(\frac{1}{\sqrt{n+1}}+\frac{1}{\sqrt{n-1}}-\frac{2}{\sqrt{n}}\right)$ et de $\sum_n \ln\left(1-\frac{1}{n^2}\right)$.
\end{Exo}


\begin{Exo}
	 Convergence et somme de $\sum_{n>0}\frac{1}{n^2+3n}$, $\sum_{n>0}\frac{(-1)^n}{n(n+1)}$ et $\sum_{n>0}\frac{1}{n^2(n+1)}$.
\end{Exo}

%\begin{Exo}[Quelques calculs de sommes]
%	\ 
%\begin{enumerate}
%	\item
%	Calculer les sommes (après avoir justifié leur existence...): $$\sum_{n=0}^{\infty}\frac{(-1)^n}{2^nn!}\quad\sum_{n=0}^{\infty}\frac{n}{n!}\quad\sum_{n=0}^{\infty}\frac{n(n-1)}{n!}\quad\sum_{n=0}^{\infty}\frac{2n^2+3n-1}{n!}$$
%	\item
%	CV et somme de $\sum_{n>0}\frac{1}{n^2+3n}$, $\sum_{n>0}\frac{(-1)^n}{n(n+1)}$ et $\sum_{n>0}\frac{1}{n^2(n+1)}$.
%	\item CV et somme de $\sum_n \ln\left(1-\frac{1}{n^2}\right)$
%	%\item (Plus difficile) Montrer que $\sum_{n \geqslant 1}(-1)^n \ln \left(1+\frac{1}{n}\right)$ converge, expliciter $S_{2n}$ sa somme partielle de rang $2n$ et donner sa limite via Stirling.
%\end{enumerate}
%
%\end{Exo}


\begin{Exo}
	Nature et somme de $\sum_n \ln\left(1+\frac{(-1)^n}{n}\right)$. On pourra calculer $u_{2n}+u_{2n+1}$.
\end{Exo}



\begin{Exo}[Sommation par tranche]
	\ 
	
	Convergence et somme de la série de terme général $u_n=\frac{\lfloor \sqrt{n+1}\rfloor-\lfloor \sqrt{n}\rfloor }{n}$
\end{Exo}

\begin{Exo}[Comparaison à une série de Riemann]
	Soit $\sum_n u_n$ à termes positifs. Que dire de sa nature quand:
	\begin{enumerate}
		\item
		Il existe $a\in \R$ tel que $n^au_n\to \ell\in \R_+^*$.
		\item 
		$nu_n\to +\infty$
		\item 
		$n^2u_n\to 0$
	\end{enumerate}
\end{Exo}

\begin{Exo}
\ 
Nature des séries  de terme général:
\begin{multicols}{3}
\begin{enumerate}
\item $\sin \left( \sqrt{4n^{2}+1}\pi \right)$
\item $\left( \cos \frac{1}{n}\right) ^{n^{3}}$
\item $\frac{1}{2^{\sqrt{n}}}$
\item $e-\left( 1+\frac{1}{n}\right) ^{n}$
\item $\frac{1}{n^{n+\frac{1}{n}}}$
\item $\frac{n^{3}+2^{n}}{n^{2}+3^{n}}$
\item $\arctan \frac{n}{n^{2}+1}$
\item $\int_{0}^{\pi /n}\frac{\sin ^{3}x}{1+x}dx$
\item $\sqrt{\ln (n+1)}-\sqrt{\ln(n)}$
\item $\left( \frac{1}{\ln n}\right) ^{\ln n}$
\item $\int_{n}^{2n}\frac{dt}{1+t^{3/2}}$
\item $\frac{n^{\ln n}}{\left( \ln n\right) ^{n}}$
\item $\dfrac{2.4.6\dots (2n)}{n^{n}}$
\item $\dfrac{a^{n}}{1+a^{2n}}$
\item $\frac{(2n)!}{n!a^{n}n^{n}},\left( a>0\right)$
\item $\frac{n^{a}n!e^{n}}{(n+1)^{n}}$
\item $e^{\cos \frac{1}{n}}-e^{\cos \frac{2}{n}}$
\item $n^{n^{-a}}-1$ avec $a>0$
\item $\frac{1}{\sqrt{n^2-1}}-\frac{1}{\sqrt{n^2+1}}$
\item $\cos(n\pi+1/n)$
%\item $(-1)^n \frac{\ln n}{\sqrt{n}}$
%\item $(-1)^n \sqrt{n}\sin \frac{1}{n}$
\item $\ln\left(1+\frac{(-1)^n}{n} \right)$
\item $\left(\frac{\sqrt{n}}{1+\sqrt{n}}\right)^n$
\end{enumerate}
\end{multicols}

\end{Exo}


\begin{Exo}[Constante d'Euler $\gamma$]
\ 
On note pour $n>0:H_n=\sum_{k=1}^{n}\frac{1}{k}$.
\begin{enumerate}
	\item
	Montrer que $\forall n>0:\int_1^{n+1}\frac{\diff t}{t} \leqslant H_n\leqslant 1+\int_1^n \frac{\diff t}{t}$, en déduire un équivalent de $H_n$.
	\item
	Posons $u_n=H_n-\ln n$. Déterminer un équivalent de $u_{n+1}-u_n$, en déduire que $(u_n)_n$ converge vers une limite que l'on notera $\gamma$ (constante d'Euler, environ $0,57$). Ainsi:
	\[H_n=\ln n +\gamma+\varepsilon_n\text{ avec }\varepsilon_n\to 0\]
	
	\textbf{Applications:}
		\item
	Convergence et somme de la somme de la série harmonique alternée $\sum_{n>0}\frac{(-1)^{n+1}}{n}$.
	\item
	Calculer la somme de la série $\sum_{n>0}\frac{1}{1^2+2^2+...+n^2}$.
	\item
	Déterminer suivant $a>0$ la nature de $\sum_n a^{H_n}$.
\end{enumerate}
\end{Exo}



\begin{Exo}
	Soit $(u_{n})_{n}$ une suite décroissante de réels positifs.
On suppose que la série $\sum_n u_{n}$ converge.

\begin{enumerate}
	\item Montrez que $\lim_{n}nu_{n}=0.$
	
	\item On pose $v_{n}=nu_{n}-nu_{n+1}.$ Montrez que $\sum_n v_{n}$ converge et a même somme que $\sum_n u_{n}$.
\end{enumerate}
\end{Exo}

%\begin{Exo}
%Soit $ u_n=\frac{(-1)^{n+1}}{\sqrt{n+1}}-\frac{(-1)^n}{\sqrt{n}}$ et $v_n=u_n+\frac{1}{n}$. Montrer que $\sum_n u_n$ converge et que $u_n\sim v_n$. Nature de $\sum_n v_n$?
%\end{Exo}

\begin{Exo}[Un classique]
Montrer que $(2+\sqrt{3})^n+(2-\sqrt{3})^n \in 2\Z$. Préciser la nature de $\sum_n \sin\left ( (2-\sqrt{3})^n \pi \right)$ puis celle de $\sum_n \sin\left ( (2+\sqrt{3})^n \pi \right)$.
\end{Exo}


\begin{Exo}
	Soit $\sum_n u_{n}$ une série réelle semi-convergente. Pour tout $n$, on pose $u_{n}^{+}=\max \left(0,u_{n}\right) $ et $u_{n}^{-}=\max \left( 0,-u_{n}\right) $. Montrez que $\sum_n u_{n}^{+}$ et $\sum_n u_{n}^{-}$ sont divergentes.
\end{Exo}

\begin{Exo}
	Nature suivant $b>0$ de $\sum_n\frac{n^n}{n!b^n}$.
\end{Exo}

\begin{Exo}
	Etudier $\sum_n \frac{\sqrt{1}+\sqrt{2}+...+\sqrt{n}}{n^{\alpha}}$.
\end{Exo}


\begin{Exo}[Comparaison série-intégrale]\ 
	
	\begin{enumerate}
		\item
		Trouver la partie entière de $\sum_{k=1}^{10^9}\frac{1}{k^{2/3}}$.
		
		\item
		Trouver un équivalent de $\sum_{k=n+1}^{+\infty}\frac{1}{k^{2}}$ et de de $\sum_{k=1}^n\sqrt{k}$.
		\item 
		Nature de $\sum_{n}\frac{1}{n^a \ln n}$ suivant $a\in\R$.
	\end{enumerate}
	
\end{Exo}

\begin{Exo}
	Soit $(a_{n})_{n}$ suite positive et $(u_{n})_{n}$ définie par $u_{0}>0$ et pour tout $n:u_{n+1}=u_{n}+\frac{a_{n}}{u_{n}}$.
	\begin{enumerate}
	\item
	Montrer que $u$ est croissante et à termes strictement positifs.
	\item On suppose que $\sum_{n a_{n}}$ converge, montrer que $u$ converge.
	\item On suppose que $u$ converge, montrer que $\sum_{n} a_{n}$ converge.
	\end{enumerate}
	
\end{Exo}


\begin{Exo}
	Etudier $\sum_n \frac{\sqrt{1}+\sqrt{2}+...+\sqrt{n}}{n^{\alpha}}$.
\end{Exo}

\begin{Exo}[terme général défini par une suite récurrente]
	Soit la suite définie par $x_0>0$ et pour tout $n$ de $\N:x_{n+1}=\frac{\mathrm{e}^{-x_n}}{n+1}$. Déterminer la nature de   $(x_n)_n$, un équivalent de $x_n$ puis la nature de $\sum_n x_n$.
\end{Exo}

\begin{Exo}[terme général défini implicitement]
	\ 
	\begin{enumerate}
		\item
		Montrer que pour tout $n>2$ il existe exactement deux solutions dans $\R_+^*$ de $x^n=\mathrm{e}^x$ notée $x_n,y_n$ tels que $x_n<y_n$.
		\item
		Montrer que $x_n$ tend vers $1$.
		\item
		Déterminer la nature de la série de terme général $u_n=\frac{x_n-1}{y_n}$.
	\end{enumerate}
\end{Exo}

\begin{Exo}[Séries alternées]
\ 

Nature de:
\begin{multicols}{3}
\begin{enumerate}
	\item
	$\sum_n \sin\left( \pi\sqrt{n^2+1}\right)$
	\item
	$\sum_n \frac{(-\ln n)^n}{n^{\ln n}}$
	\item
	$\sum_{n}u_n=\frac{(-1)^n}{n!^{1/n}}$
\end{enumerate}
\end{multicols}

\end{Exo}

\begin{Exo}[Attention!]
	On note $u_n=
	\frac{(-1)^n}{\sqrt{n}}$ et $v_n=\frac{(-1)^n}{\sqrt{n}+(-1)^{n+1}}$.
	\begin{enumerate}
		\item
		Montrer que $\sum_n u_n$ converge.
		\item
		Déterminer $a$ réel tels que $v_n=\frac{(-1)^n}{\sqrt{n}}+\frac{a}{n}+o\left(\frac{1}{n}\right)$, en déduire que $\sum_n v_n$ est divergente.
		\item
		$\sum_n u_n$ et $\sum_n v_n$ sont-elles de même nature? Qu'illustre cet exemple?
	\end{enumerate}
\end{Exo}

\begin{Exo}
	Nature et somme de $\sum_n u_{n}$ avec $u_{n}=\frac{1}{n}\cos \left( 2n\frac{\pi }{3}\right)$.
\end{Exo}

\begin{Exo}
	On admet que $\sum_{n=1}^{+\infty}\frac{1}{n^2}=\frac{\pi^2}{6}$. Préciser: $\sum_{n=1}^{+\infty}\frac{1}{(2n)^2},\sum_{n=1}^{+\infty}\frac{1}{(2n+1)^2},\sum_{n=1}^{+\infty}\frac{(-1)^n}{(n)^2}$.
\end{Exo}

\begin{Exo}[Avec une intégrale] On note $u_n=\int_0^{\pi/4} \tan^n xdx$.
\begin{enumerate}
\item Calculer $u_{n+2}+u_n$ et montrer que $u$ est décroissante.
\item En comparant $u_n$ et $u_n+u_{n+2}$ en déduire la nature de $\sum_n u_n$.
\item
Donner la nature de $\sum_n(-1)^n I_n$.
\end{enumerate}
\end{Exo}

\begin{Exo}
	 Soient $\sum u_{n}$, $\sum v_{n}$, $\sum w_{n}$  séries réelles telles que $\sum u_{n}$ et $\sum w_{n}$ convergent,
et $u_{n}\leqslant v_{n}\leqslant w_{n}$ pour tout $n$. Montrez que $\sum v_{n}$ converge.

\end{Exo}


\begin{Exo}[Plus général...]
\ 
Toutes les séries sont à termes réels...
\begin{enumerate}
\item On suppose \textbf{ici} que $\sum u_n$ est à termes positifs, montrer que si $\sum_n u_n$ CV alors $\sum_n u_n^2,\sum_n \frac{u_n}{1+u_n}$ et $\sum_n u_n u_{2n}$ CV. 
\item Donner un exemple où $\sum u_n$ CV et $\sum u_n^2$ DV.
\item On suppose que $\sum u_n$  et $\sum u_n^2$ CV, montrer que $\sum_n \frac{u_n}{1+u_n}$ CV. Et si on suppose juste que $\sum u_n$ CV?
\item Soit $u,v$ dans $\R_+^{\N}$, montrer que si $\sum_n u_n$ et $\sum_n v_n$ convergent alors $\sum_n \sqrt{u_n v_n}$ converge.
%\item Soit $u$ dans $\R_+^{\N}$, on pose pour $n\in \N:v_n=\frac{1}{1+n^2u_n}$. Exprimer $\sqrt{u_nv_n}$ en fonction de $n,v_n$, en déduire que $\sum_n u_n$ et $\sum_n v_n$ ne peuvent converger simultanément.
\end{enumerate}
\end{Exo}







\begin{Exo}[Calculs approchés]
\ 

\begin{enumerate}
\item Calculer une valeur approchée à $10^{-4}$ près de $\sum_{n=1}^{+\infty}\frac{1}{n^n}$.
\item Calculer une valeur approchée à $10^{-6}$ près de $e=\sum_{n=0}^{+\infty}\frac{1}{n!}$. On montrera que $0\leq R_n\leqslant \frac{1}{n!n}$.
\item Calculer une valeur approchée à $10^{-4}$ près de $\sum_{n=0}^{+\infty}\frac{(-1)^n}{n^5}$.
\item Calculer une valeur approchée à $10^{-4}$ près de $\sum_{n=1}^{+\infty}\frac{n^2+n+1}{2^n}$. Savez-vous calculer sa valeur exacte?
\end{enumerate}
\end{Exo}

%
%\begin{Exo}[Accélération élémentaire de convergence]
%Soit $u_n=\frac{1}{n(n+1)}$ et $v_n=\frac{1}{n^2}$.
%\begin{enumerate}
%\item Recalculer la somme de la série $\sum_{n>0}\frac{1}{n(n+1)}$.
%\item "Intuiter" que la série $\sum_n (u_n-v_n)$ converge plus rapidement que $\sum_n u_n$ (chercher par exemple un équivalent de $u_n-v_n$). Le vérifier via Python.
%\item Comment améliorer à l'aide $\sum_n \frac{1}{n(n+1)(n+2)}$ (Commencer par calculer sa somme!)?
%\end{enumerate}
%\end{Exo}



\begin{Exo}[Exponentielle]
\ 
On pourra admettre ici que pour tout réel $x$ la série  $\sum_{n\geqslant 0} \frac{x^n}{n!}$ converge et que sa somme vaut $\mathrm{e}^x$.
\begin{enumerate}
\item Soit $P \in \R[X]$, montrez que $\displaystyle\sum_{n\geqslant 0}\frac{P(n)}{n!}$ CV. On note $S_P$ sa somme.
\item Déterminer pour $k\in \N^*: S_{P_k}$ avec $P_k(X)=X(X-1)...(X-k+1)$
\item En déduire $S_{X^3+3X^2-X+2}$.
\end{enumerate}
\end{Exo}

%\begin{Exo}[constante d'Euler]\ 
%	
%	\begin{enumerate}
%		\item
%			Montrer que la suite $(u_n=H_n-\ln n)_n$ converge vers une limite strictement positive $\gamma$ appelée constante d'Euler. (On pourra montrer que $\sum (u_{n+1}-u_n)$ converge). On peut montrer que $\gamma \simeq 0,57$.
%			
%			On a ainsi $H_n=\ln n+\gamma +o(1)$.
%		\item
%		Retrouver la somme de la série harmonique alternée $\sum_{n>0}\frac{(-1)^{n+1}}{n}$.
%		\item
%		Calculer la somme de la série $\sum_{n>0}\frac{1}{1^2+2^2+...+n^3}$.
%	\end{enumerate}
%
%\end{Exo}

\begin{Exo}
	Montrer via un produit de séries que $\displaystyle\sum_{n\geqslant 0} \frac{n+1}{3^n}$ converge et préciser sa somme.
\end{Exo}

\begin{Exo}
	On suppose que $\displaystyle\sum_nu_n$ CVA, montrer que $\displaystyle\sum_n \left(\frac{1}{2^n}\sum_{k=0}^{n}2^k u_k\right)$ converge et préciser sa somme.
\end{Exo}

\begin{Exo}
	Nature et somme de $\sum_{n\geqslant 1}w_n$ avec $w_n=\sum_{k=1}^{n}\frac{1}{k^2(n-k)!}$.
\end{Exo}

%\begin{Exo}[Critère de la loupe]
%\ 
%
%\begin{enumerate}
%\item Soit $(u_n)_{n\geq 1}$ décroissante positive. On note $v_n=2^n u_{2^n}$. Montrer que:
%$$\forall n \in \N^*:\frac{1}{2}\left( v_1+v_2+...+v_n\right) \leqslant \sum_{k=1}^{2^n-1}u_k\leqslant v_0+v_1+...+v_n$$
%\item En déduire que $\sum u_n$ et $\sum v_n$ ont même nature.
%\item Déterminer les valeurs de $a\in \R$ pour lesquelles $\sum \frac{1}{n^a}$ et $\sum \frac{1}{n \ln ^a n}$ convergent.
%
%\end{enumerate}
%\end{Exo}

%\begin{Exo}[Somme de la série de Bâle]
%\ 
%
%On pose $I_{n}=\int_{0}^{\frac{\pi }{2}}t^{2}\cos ^{2n}tdt$ et $S_{n}=\sum_{p=1}^{n}\frac{1}{p^{2}}.$ On rappelle que $C_{n}=\int_{0}^{\frac{\pi }{2}}\cos ^{2n}tdt=\frac{3\times 5\times ...\times \left( 2n-1\right) }{2\times 4\times 6\times ...\times \left( 2n\right) }\frac{\pi }{2}$
%
%\begin{enumerate}
%\item A l'aide de Stirling justifier:
%\begin{equation*}
%\frac{2\times 4\times 6\times ...\times \left( 2n\right) }{3\times 5\times
%...\times \left( 2n-1\right) }\sim \sqrt{n\pi }
%\end{equation*}
%
%\item Trouver une relation entre $I_{n}$ et $I_{n-1}$ et $C_{n}$, en déduire que $\left( \frac{\pi ^{2}}{6}-S_{n}\right) \frac{\pi }{4}\leq \frac{2\times 4\times 6\times ...\times \left( 2n\right) }{3\times 5\times
%...\times \left( 2n-1\right) }I_{n}$
%
%\item Démontrer (observer que sur $[0,\pi/2]$ on a $x\leq \frac{\pi}{2}\sin x$) que $I_{n}\leq \frac{\pi ^{2}}{4\left( 2n+1\right) }$ et
%conclure.
%\end{enumerate}
%
%\end{Exo}
%\begin{Exo}
%	On rappelle que: $H_{n}=1+\frac{1}{2}+...+\frac{1}{n}=\ln (n)+\gamma
%	+0(1)$ o\`{u} $\gamma $ est la constante d'Euler. Soit $u_{n}=\frac{\nu (n)}{n(n+1)}\,o\grave{u}$ $\nu (n)$ est le nombre de 1 figurant dans l'écriture de binaire de $n$. Le but de l'exercice est d'\'{e}tudier la série  de terme général $u_n$
%	
%	\begin{enumerate}
%		\item Montrer que:
%		
%		\begin{enumerate}
%			\item $\nu (2n)=\nu (n)$ et $\nu (2n+1)=1+\nu (n)$.
%			
%			\item $\nu (n)\leq 1+\frac{\ln (n)}{\ln (2)}$.
%		\end{enumerate}
%		
%		\item Montrer que la s\'{e}rie $\sum_{n\geqslant 1} u_{n}$ est
%		convergente.
%		
%		\item On pose: $v_{n}=u_{2n}+u_{2n+1}$ et $w_{n}=\frac{1}{(2n+1)(2n+2)}$.
%		
%		\begin{enumerate}
%			\item Montrer que la série $\sum_{n\geq 1} v_{n}$ est
%			convergente et calculer sa somme en fonction de celle de la série $\sum_{n\geq 1} u_{n}$.
%			
%			\item Montrer que: $v_{n}=\frac{u_{n}}{2}+w_{n}$.
%			
%			\item Calculer la somme de la série $\sum_{n\geq 1} w_{n}$. En
%			déduire celle de la série $\sum_{n\geq 1} u_{n}$.
%		\end{enumerate}
%	\end{enumerate}
%\end{Exo}

\end{document}
