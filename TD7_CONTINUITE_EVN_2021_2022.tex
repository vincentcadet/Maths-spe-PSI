  
\documentclass[12pt,a4paper]{article}
% Engine-specific settings
% Detect pdftex/xetex/luatex, and load appropriate font packages.
% This is inspired by the approach in the iftex package.
% pdftex:

\usepackage[T1]{fontenc}
\usepackage[utf8]{inputenc}
\usepackage[french]{babel}
\frenchbsetup{StandardLists=true}
\usepackage{enumitem}
\usepackage{systeme}
\usepackage{amsmath,amssymb}
\usepackage[thmmarks,amsmath]{ntheorem}
\usepackage[colorlinks=true]{hyperref}
\usepackage{fullpage}
%\usepackage{eulervm}
\usepackage{graphicx}
\usepackage{array}
\usepackage{multicol}
\usepackage[makestderr]{pythontex}
\restartpythontexsession{\thesection}
\usepackage{geometry}
\geometry{tmargin=1.5cm,bmargin=1.5cm,lmargin=1.5cm,rmargin=1.5cm,headheight=12cm}
\usepackage{array}
\usepackage[svgnames,table]{xcolor}
\usepackage[tikz]{ bclogo}
\usepackage{pifont}
\usepackage{url}
\urlstyle{same}
\usepackage{pifont}
\usepackage{multicol}
\usepackage{diagbox} %oblique dans les tableaux
%\usepackage[framemethod=TikZ]{mdframed}
\usepackage{fancyhdr}
\pagestyle{fancyplain}
\setlength\headsep{2mm}

\lhead{\textit{CSI2B-PSI TD7}}
\chead{\textsc{Espaces vectoriels normés 2}}
\rhead{\textit{2021-2022}} 

\newcommand{\norme}[1]{\left\lVert#1\right\rVert}
\renewcommand*{\thefootnote}{\fnsymbol{footnote}}
\newcommand{\un}{(u_n)_n}
\newcommand{\R}{\mathbb{R}}
\newcommand{\C}{\mathbb{C}}
\newcommand{\Q}{\mathbb{Q}}
\newcommand{\Z}{\mathbb{Z}}
\newcommand{\N}{\mathbb{N}}
\newcommand{\K}{\mathbb{K} }

\newcommand{\diag}{\mathrm{diag}}
\renewcommand{\Re}{\mathcal{R}e}
\renewcommand{\Im}{\mathcal{I}m}

\DeclareMathOperator{\Ima }{Im}
\DeclareMathOperator{\vect}{Vect}
\DeclareMathOperator{\tr}{Trace}
\newcommand{\conj}[1]{\overline{#1}}

{%
\theoremstyle{break}
\theoremprework{%
\rule{0.5\linewidth}{0.3pt}}
\theorempostwork{\hfill%
\rule{0.5\linewidth}{0.3pt}}
\theoremheaderfont{\scshape}
\theoremseparator{ ---}
\newtheorem{Prop}{%
\textcolor{blue}{Proposition}}[section]
}

{%
\theoremstyle{break}
\theoremprework{%
\rule{0.6\linewidth}{0.5pt}}
\theorempostwork{\hfill%
\rule{0.6\linewidth}{0.5pt}}
\theoremheaderfont{\scshape}
\newtheorem{Theo}{%
\textcolor{red}{Théorème}}[section]
}


{%
\theoremheaderfont{\sffamily\bfseries}
\theorembodyfont{\sffamily}
\newtheorem{Def}{%
\textcolor{green}{Définition}}[section]
}

{%
\theorembodyfont{\small}
\theoremsymbol{$\square$}
\newtheorem*{Dem}{Démonstration}
}

{%
\theorembodyfont{\small}
\newtheorem*{Exemple}{Exemple}
}


{%
\theorembodyfont{\small}
\newtheorem{Exo}{Exercice}
}

{%
\theoremnumbering{Roman}
\theorembodyfont{\normalfont}
\newtheorem{Rem}{Remarque}
}




\begin{document}

%\emph{\textbf{“Je crois beaucoup en la chance ; et je constate que plus je travaille, plus la chance me sourit” (Thomas Jefferson)}}
%
%\begin{center}
%\textsc{Les exercices ... sont à chercher pour vendredi}   
%\end{center}

\begin{Exo}
		 Soit $I$ intervalle de $\R$ et $f:I\to \R$. Montrer que $f$ est lipschitzienne sur $I$ ssi $f'$  bornée sur $I$.
\end{Exo}
%\begin{Exo}
%	 Montrer que toute matrice de $M_n(\K)$ est limite d'une suite de matrices inversibles. L'ensemble $GL_n(\K)$ est-il ouvert? Fermé? Convexe? Borné?
%\end{Exo}
% \begin{Exo}
% 	Soit $F=\left\{M\in M_n(\K),M^2=M\right\}$. Montrer que $F$ est fermé. Est-il ouvert? Borné? Est-ce un sev? Est-il convexe?
% \end{Exo}

%\begin{Exo}
%	Soit $F$ l'ensemble des matrices triangulaires supérieures inversibles de $M_3(\C)$.
%
%	\begin{enumerate}
%	\item
%	Préciser $A^{-1}$ pour $A=\begin{pmatrix}
%		a & b & c \\
%		0 & d & e \\
%		0 & 0 & f
%	\end{pmatrix}$ dans $F$.
%	\item En déduire que $f:M\in F\mapsto M^{-1}$ est continue.
%	\item Justifier que plus généralement l'application qui à une matrice inversible associe son inverse est continue sur $GL_n(\K)$.
%\end{enumerate}
%\end{Exo}

\begin{Exo}
	Justifier que l'application qui à une matrice inversible associe son inverse est continue sur $GL_n(\K)$.
\end{Exo}

%	\begin{Exo}
%		 Soit $f:\R^{2}\rightarrow \R$ définie par $f\left(x,y\right) =\frac{xy^{2}}{x^{2}+y^{4}}$ et $f\left( 0,0\right) =0.$ Etudier
%	la continuité de $f$ en $\left( 0,0\right)$.
%	\end{Exo}
	
%\begin{Exo}
%	
%	 Soit $f:\R^{2}\rightarrow \R$ définie par $f\left(x,y\right) =\frac{3x^{2}+xy}{\sqrt{x^{2}+y^{2}}}$ et $f\left( 0,0\right) =0.$ Etudier
%l'existence d'une limite en $\left( 0,0\right)$.
%\end{Exo}

% \begin{Exo}
%	Etudier la continuité sur $\R^2$ de $f:(x,y)\mapsto \begin{cases}
%2x^2+y^2-1\text{ si }x^2+y^2>1\\x^2 \text{ sinon}
%\end{cases}$.
%\end{Exo}

\begin{Exo}
	Continuité sur $\R^2$ de $f:(x,y)\mapsto\begin{cases}
		\left(\frac{\sin(xy)}{x^2+y^2},\sqrt{|x|+|y|}\right)\text{ si }(x,y)\neq (0,0)\\ (0,0)\text{ si }(x,y)=(0,0)
	\end{cases}.$
\end{Exo}
%\begin{Exo}
%	 Soit $\left( E,\left\| \,\right\| \right) $ un evn et $A$ une partie
%de $E$ ouverte et fermée. (exemple ?)
%
%	\begin{enumerate}
%	\item On note $\chi _{A}$ la fonction caractéristique de $A,$ définie par: $\chi _{A}\left( x\right) =\left\{ 
%	\begin{array}{l}
%		1\text{ si }x\in A \\ 
%		0\text{ sinon}
%	\end{array}
%	\right.$. Montrer que $\chi _{A}$ est continue.
%	
%	\item Soit $a\in A$ et $b\notin A$. Montrez que $\phi :t\in \left[ 0,1\right]
%	\mapsto \chi _{A}\left( ta+\left( 1-t\right) b\right) $ est continue.
%	Conclure sur $A$.
%	\end{enumerate}
%\end{Exo}


%\begin{Exo}
%	 Parmi les parties suivantes de $\R$ usuel quelles sont celles qui sont ouvertes? Fermées? $$]1,2[,]1,2],[1,2],[1,2]\cup [3,4],[1,2]\cup\{3\},\Z,\R\setminus \Z,\Q,\left\{\frac{1}{n},n>0\right\},\left\{\frac{1}{n},n>0\right\}\cup\{0\}$$
%\end{Exo}
\begin{Exo}
	 Soit $(E,\norme{})$ un evn et $f:x\mapsto \frac{x}{\norme{x}+1}$. Montrer que $f$ est continue sur $E$, bijective de $E$ sur $B(0,1)$, de réciproque continue.
\end{Exo}

% \begin{Exo}
% 	Montrer de diverses façons que $\left\{(x,y)\in\R^2,xy=1\right\}$ est un fermé de $\R^2$. Est-il borné? Ouvert?
% \end{Exo}

\begin{Exo}\ 
		Soit $u\in L(E,F)$, on rappelle qu'en utilisant la définition quantifiée de la continuité en $0$ on peut montrer que $u$ est continue sur $E$ ssi il existe $K>0$ tel que:
		$$\forall x\in E,\norme{u(x)}_F\leqslant K\norme{x}_E$$
	\begin{enumerate}
		\item
		Déterminer la  limite de la suite $\left(\frac{X^n}{n}\right)_n$ dans $(\R[X],\norme{}_{\infty})$. (Rappel: si $P=\sum_{k=0}^{N}a_kX^k$ alors $\norme{P}_{\infty}=\sum_{k=0}^N|a_k|$).
		$D:P\mapsto P'$ est-elle continue sur $(\R[X],\norme{}_{\infty})$?\item
		Pour $P=\sum_k a_kX^k:N(P)=\sum_k k!|a_k|$. (On admet que $N$ est une norme sur $\R[X]$). Montrer que $D$ est continue sur $(\R[X],N)$.
		\item
		Ici $E=C^{\infty}(\R,\R)$ et on note pour $\lambda$ réel $e_{\lambda}:x\mapsto \exp(\lambda x)$. Soit $N$ une norme sur $E$, comparer $N(e_{\lambda})$ et $N(D(e_{\lambda}))$. Existe-t-il une norme sur $E$ pour laquelle $D$ soit continue?
		\item
		Soit $c\in\R$ et $n>0$. Montrer que $u:P\mapsto P(c)$ est continue sur $\R_n[X]$.
		\item
		On munit $\R[X]$ de sa norme $\infty$. On fixe $c\in]-1,1[$, montrer que $v:P\mapsto P(c)$ est continue sur $\R[X]$.
		\item On note	$w:P\mapsto P(1)$. Calculer $w(P_n)$ avec $P_n=\frac{1}{n+1}(1+X+X^2+...+X^n)$. Quelle est la limite de $(P_n)_n$ pour $\norme{ }_{\infty}$? $w$ est-elle continue sur $\R[X]$ muni de la norme $\infty$?
	\end{enumerate}
	
\end{Exo}



\begin{Exo}
	 Soit $A$ une partie fermée et bornée d'un evn $\left( E,\left\Vert
\,\right\Vert \right)$ de dimension finie et $f$ une application de $A$ dans $A$ vérifiant: 
\begin{equation*}
	\forall x,y\in A,x\neq y\implies \left\Vert f(x)-f(y)\right\Vert
	<\left\Vert x-y\right\Vert \text{ }(\ast )
\end{equation*}


\begin{enumerate}
	\item A l'aide de l'application $h:x\mapsto \left\Vert x-f(x)\right\Vert $	montrer qu'il existe $a$ dans $A$ tel que pour tout $x$ de $A:\left\Vert
	a-f(a)\right\Vert \leqslant \left\Vert x-f(x)\right\Vert$.
	
	\item En déduire que $f$ possède un unique point fixe qui est $a$.

\end{enumerate}
\end{Exo}

\begin{Exo}
	Soit $A$ un fermé de $\R$ et $f:A\to A$ contractante, c'est à dire qu'il existe $k\in[0,1[$ tel que $f$ soit  $k$-lipschitzienne sur $A$.
	\begin{enumerate}
		\item
		Montrer que $f$ admet au plus un point fixe dans $A$.
		
		Soit $a$ dans $A$ et $(x_n)_n$ définie par $x_0=a$ et $\forall n\in\N:x_{n+1}=f(x_n)$.
		\item
		Montrer que pour tout entier $n:|x_{n+1}-x_n|\leqslant k^n|x_1-x_0|$. 
		\item
		En déduire que$\sum_n(x_{n+1}-x_n)$ converge absolument.
		\item
		En déduire que $(x_n)_n$ converge et que $f$ a un unique point fixe dans $A$.
	\end{enumerate}

\end{Exo}

\begin{Exo}\ 
\begin{enumerate}
	\item
	Soit $f:\R\to\R$ continue $T$ périodique. Montrer qu'il existe $c\in\R$ tel que $f(\R)=f([c,c+T/2])$.
	\item
	Soit $f:\R^2\to \R$ continue et $C$ le cercle de centre $0$ et de rayon $r>0$. Montrer qu'il existe deux points de $C$ diamétralement opposés en lesquels $f$ coïncide.
\end{enumerate}
\end{Exo}

\end{document}
