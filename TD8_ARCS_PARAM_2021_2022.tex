  
\documentclass[12pt,a4paper]{article}
% Engine-specific settings
% Detect pdftex/xetex/luatex, and load appropriate font packages.
% This is inspired by the approach in the iftex package.
% pdftex:

\usepackage[T1]{fontenc}
\usepackage[utf8]{inputenc}
\usepackage[french]{babel}
\frenchbsetup{StandardLists=true}
\usepackage{enumitem}
\usepackage{systeme}
\usepackage{amsmath,amssymb}
\usepackage[thmmarks,amsmath]{ntheorem}
\usepackage[colorlinks=true]{hyperref}
\usepackage{fullpage}
%\usepackage{eulervm}
\usepackage{graphicx}
\usepackage{array}
\usepackage{multicol}
\usepackage{geometry}
\geometry{tmargin=1.5cm,bmargin=1.5cm,lmargin=1.5cm,rmargin=1.5cm,headheight=12cm}
\usepackage{array}
\usepackage[svgnames,table]{xcolor}
\usepackage[tikz]{ bclogo}
\usepackage{pifont}
\usepackage{url}
\urlstyle{same}
\usepackage{pifont}
\usepackage{multicol}
\usepackage{diagbox} %oblique dans les tableaux
%\usepackage[framemethod=TikZ]{mdframed}
\usepackage{fancyhdr}
\pagestyle{fancyplain}
\lhead{\textit{CSI2B-PSI TD8}}
\chead{\textsc{Fonctions vectorielles-Arcs paramétrés}}
\rhead{\textit{2021-2022}} 


\renewcommand*{\thefootnote}{\fnsymbol{footnote}}
\newcommand{\un}{(u_n)_n}
\newcommand{\R}{\mathbb{R}}
\newcommand{\C}{\mathbb{C}}
\newcommand{\Q}{\mathbb{Q}}
\newcommand{\Z}{\mathbb{Z}}
\newcommand{\N}{\mathbb{N}}
\newcommand{\K}{\mathbb{K} }
\newcommand{\I}{\mathbf{i}}
\renewcommand{\Re}{\mathcal{R}e}
\renewcommand{\Im}{\mathcal{I}m}
\DeclareMathOperator{\Ima }{Im}
\newcommand{\E}{\mathrm{e}}
\newcommand{\conj}[1]{\overline{#1}}
\newcommand{\diff}{\mathop{}\mathopen{}\mathrm{d}}%element differentiel
{%
\theoremstyle{break}
\theoremprework{%
\rule{0.5\linewidth}{0.3pt}}
\theorempostwork{\hfill%
\rule{0.5\linewidth}{0.3pt}}
\theoremheaderfont{\scshape}
\theoremseparator{ ---}
\newtheorem{Prop}{%
\textcolor{blue}{Proposition}}[section]
}

{%
\theoremstyle{break}
\theoremprework{%
\rule{0.6\linewidth}{0.5pt}}
\theorempostwork{\hfill%
\rule{0.6\linewidth}{0.5pt}}
\theoremheaderfont{\scshape}
\newtheorem{Theo}{%
\textcolor{red}{Théorème}}[section]
}


{%
\theoremheaderfont{\sffamily\bfseries}
\theorembodyfont{\sffamily}
\newtheorem{Def}{%
\textcolor{green}{Définition}}[section]
}

{%
\theorembodyfont{\small}
\theoremsymbol{$\square$}
\newtheorem*{Dem}{Démonstration}
}

{%
\theorembodyfont{\small}
\newtheorem*{Exemple}{Exemple}
}


{%
\theorembodyfont{\small}
\newtheorem{Exo}{Exercice}
}

{%
\theoremnumbering{Roman}
\theorembodyfont{\normalfont}
\newtheorem{Rem}{Remarque}
}




\begin{document}
	

\begin{Exo}
	Donner un arc paramétré $(I,f)$ de support: 
	\begin{enumerate}
		\item
		La droite du plan affine passant par $A=(a_1,a_2)$ et dirigée par le vecteur $u=(u_1,u_2)$,
		\item
		La droite du plan affine passant par $A=(a_1,a_2)$ et  $B=(b_1,b_2)\neq A$,
		\item
		Le segment joignant $A$ et $B$.
	\end{enumerate}
	
\end{Exo}

\begin{Exo}
	Soit l'arc paramétrique $f(t)=\left((t-2)^3,t^2-4\right)$, déterminer les points d'inflexion ainsi que l'équation de la tangente en ces points.
\end{Exo}

\begin{Exo}
		Soit l'arc paramétrique $f(t)=\left(\exp(t),t^2\right)$, déterminer les points d'inflexion ainsi que l'équation de la tangente en ces points.
\end{Exo}

\begin{Exo}
	Donner le tableau de variation de $x,y$ si la courbe ci-dessous est celle de l'arc $f=(x,y)$.
	\begin{center}
	%\begin{figure}[h]
	\includegraphics[scale=.4]{courbeparam4.png}
	%\end{figure}
	\end{center}
\end{Exo}

\begin{Exo}
	\'Etudier et tracer la courbe de Lissajous $t\mapsto (\sin(2t),\cos(3t))$.
\end{Exo}

\begin{Exo}[Folium de Descartes]

	On considère la courbe paramétrée 
	$$t\mapsto \left(\frac{t}{1+t^3},\frac{t^2}{1+t^3}\right).$$
	\begin{enumerate}
		\item Que déduit-on du changement de variables $t\mapsto 1/t$? Sur quel intervalle peut-on réduire l'étude?
		\item Construire la courbe. On étudiera ses branches infinies, et on précisera la position de la courbe par rapport à sa ou ses asymptotes.
	\end{enumerate}
	
\end{Exo}

\begin{Exo}
	\'Etudier la courbe paramétrée suivante : $t\mapsto \left(t+\frac 1t,t+\frac 1{2t^2}\right)$, $t\in\mathbb R^*$.
	On étudiera en particulier la position par rapport aux asymptotes, et la tangente aux points stationnaires. On pourra aussi chercher un point d'inflexion.
\end{Exo}

\begin{Exo}
	Tracer l'arc $f(t)=\left(\frac{3t}{1+t^3},\frac{3t^2}{1+t^3}\right)$.
\end{Exo}

\begin{Exo}
	Démontrer que la courbe paramétrée $t\mapsto \left(\displaystyle 2t-\frac 1{t^2},2t+t^2\right)$ possède un point double dont on donnera les coordonnées.

\end{Exo}




\begin{Exo}
	\'Etudier les branches infinies de la courbe paramétrée $t\mapsto \left(\frac{t^3}{t^2-9},\frac{t(t-2)}{t-3}\right).$

\end{Exo}


\begin{Exo}
	Déterminer la longueur de la courbe représentative de $f:t\in [0,1]\mapsto t^{3/2}$.
\end{Exo}

\begin{Exo}
	Montrer que l’arc paramétré défini par $M(t)=(e^{t-1}-t,t^3-3t)$	comporte un point de rebroussement de seconde espèce que l’on précisera ; on situera, au voisinage de	ce point, les deux branches de la courbe de part et d’autre d’un arc de parabole.
\end{Exo}

\begin{Exo}[Déltoïde]
	Tracer le support de l'arc défini par $M(t)=2\cos t+\cos(2t),2\sin t-\sin(2t))$ et $t\in[-\pi,\pi]$. On pourra comparer $M(-t)$ et $M(t)$ et étudier la nature de $M(2\pi/3)$.  Préciser aussi la longueur de l'arc
\end{Exo}

\begin{Exo}
	Etudier et tracer le support de $(\R,M)$ avec $M:t\mapsto ((1-t)^2e^t,2(1-t)e^t)$.
\end{Exo}

\begin{Exo}
	Soit l'arc param\'{e}tr\'{e} $(\R^*,M)$ avec pour $t\in\R^*:M(t)=(x(t),y(t))$:%
	\begin{equation*}
		\left\{ 
		\begin{array}{c}
			x(t)=t^{2}+\frac{2}{t} \\ 
			y(t)=t^{2}+\frac{1}{t^{2}}%
		\end{array}%
		\right. .
	\end{equation*}
	
	\begin{enumerate}
				\item Faire les tableaux de variation de $x$ et $y$.
		\item Pr\'{e}ciser les branches infinies et les asymptotes \'{e}ventuelles et la
		position de la courbe par rapport \`{a} celles-ci.
		
		\item D\'{e}terminer le point double de cette courbe (ie chercher $t,t'$ tels que $M(t)=M(t')$).
		
		\item D\'{e}terminer le(s) point(s) singulier(s). Donner la nature et d\'{e}%
		terminer la tangente en ce(s) point(s).
		

		
		\item Tracer avec soin cette courbe.
	\end{enumerate}
\end{Exo}
\end{document}
