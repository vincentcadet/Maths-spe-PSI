  
\documentclass[12pt,a4paper]{article}
% Engine-specific settings
% Detect pdftex/xetex/luatex, and load appropriate font packages.
% This is inspired by the approach in the iftex package.
% pdftex:

\usepackage[T1]{fontenc}
\usepackage[utf8]{inputenc}
\usepackage[french]{babel}
\frenchbsetup{StandardLists=true}
\usepackage{enumitem}
\usepackage{systeme}
\usepackage{amsmath,amssymb}
\usepackage[thmmarks,amsmath]{ntheorem}
\usepackage[colorlinks=true]{hyperref}
\usepackage{fullpage}
%\usepackage{eulervm}
\usepackage{graphicx}
\usepackage{array}
\usepackage{multicol}
\usepackage[makestderr]{pythontex}
\restartpythontexsession{\thesection}
\usepackage{geometry}
\geometry{tmargin=1.5cm,bmargin=1.5cm,lmargin=1.5cm,rmargin=1.5cm,headheight=12cm}
\usepackage{array}
\usepackage[svgnames,table]{xcolor}
\usepackage[tikz]{ bclogo}
\usepackage{pifont}
\usepackage{url}
\urlstyle{same}
\usepackage{pifont}
\usepackage{multicol}
\usepackage{diagbox} %oblique dans les tableaux
%\usepackage[framemethod=TikZ]{mdframed}
\usepackage{fancyhdr}
\pagestyle{fancyplain}
\setlength\headsep{2mm}

\lhead{\textit{CSI2B-PSI TD9}}
\chead{\textsc{Suites et séries de fonctions}}
\rhead{\textit{2021-2022}} 

\newcommand{\norme}[1]{\left\lVert#1\right\rVert}
\renewcommand*{\thefootnote}{\fnsymbol{footnote}}
\newcommand{\un}{(u_n)_n}
\newcommand{\R}{\mathbb{R}}
\newcommand{\C}{\mathbb{C}}
\newcommand{\Q}{\mathbb{Q}}
\newcommand{\Z}{\mathbb{Z}}
\newcommand{\N}{\mathbb{N}}
\newcommand{\K}{\mathbb{K} }
\newcommand{\E}{\mathrm{e}}
\newcommand{\diag}{\mathrm{diag}}
\renewcommand{\Re}{\mathcal{R}e}
\renewcommand{\Im}{\mathcal{I}m}
\newcommand{\diff}{\mathop{}\mathopen{}\mathrm{d}}%element differentiel
\DeclareMathOperator{\Ima }{Im}
\DeclareMathOperator{\vect}{Vect}
\DeclareMathOperator{\tr}{Trace}
\newcommand{\conj}[1]{\overline{#1}}
\everymath{\displaystyle}
{%
	\theoremstyle{break}
	\theoremprework{%
		\rule{0.5\linewidth}{0.3pt}}
	\theorempostwork{\hfill%
		\rule{0.5\linewidth}{0.3pt}}
	\theoremheaderfont{\scshape}
	\theoremseparator{ ---}
	\newtheorem{Prop}{%
		\textcolor{blue}{Proposition}}[section]
}

{%
	\theoremstyle{break}
	\theoremprework{%
		\rule{0.6\linewidth}{0.5pt}}
	\theorempostwork{\hfill%
		\rule{0.6\linewidth}{0.5pt}}
	\theoremheaderfont{\scshape}
	\newtheorem{Theo}{%
		\textcolor{red}{Théorème}}[section]
}


{%
	\theoremheaderfont{\sffamily\bfseries}
	\theorembodyfont{\sffamily}
	\newtheorem{Def}{%
		\textcolor{green}{Définition}}[section]
}

{%
	\theorembodyfont{\small}
	\theoremsymbol{$\square$}
	\newtheorem*{Dem}{Démonstration}
}

{%
	\theorembodyfont{\small}
	\newtheorem*{Exemple}{Exemple}
}


{%
	\theorembodyfont{\small}
	\newtheorem{Exo}{Exercice}
}

{%
	\theoremnumbering{Roman}
	\theorembodyfont{\normalfont}
	\newtheorem{Rem}{Remarque}
}




%--------------------------------------------------------------------
%-------------DOCUMENT-----------------------------------------------
%--------------------------------------------------------------------


\begin{document}

\emph{\textbf{
		Les vrais informaticiens confondent toujours Halloween et Noël car pour eux 
}}

\emph{\textbf{
	Oct 31 = Dec 25.
		(Andrew Rutherford)
}}

\begin{Exo}
	Montrer que la limite simple d'une suite de fonctions croissantes sur $I$ est croissante sur $I$.
\end{Exo}

%\begin{Exo}
%Soit $f_n:x\in\R\mapsto \frac{1}{(1+x^2)^n}$. CVS,CVU sur $\R$? CVU sur $[a,+\infty[$ avec $a>0$?
%\end{Exo}

\begin{Exo}
	

\end{Exo}
\begin{Exo}
Soit $f_n:x\in\R_+\mapsto \mathrm{e}^{-nx}\sin(nx)$. CVS,CVU sur $\R_+$? CVU sur $[a,+\infty[$ avec $a>0$?
\end{Exo}

\begin{Exo}
	Soit $f_n:x\in\R\mapsto \frac{nx}{1+n^2x^2}$. Etudier la CVS de $(f_n)_n$ sur $\R$, déterminer $\lim_n f_n(1/n)$, étudier  la CVU de $(f_n)_n$ sur $\R$ puis montrer qu'il y a CVU sur tout segment de $\R_+^*$.

\end{Exo}

%\begin{Exo}
%	Soit $f_n:x\in[0,1]\mapsto \begin{cases}
%0\text{ si }x=0\\nx^n\ln x
%	\end{cases}$. 	Etudier la CVS et la CVU de $(f_n)_n$ sur $[0,1]$ puis la CVU sur tout segment de $[0,1[$.
%
%\end{Exo}

\begin{Exo}
Pour $a\geqslant 0$ on note $f_n:x\in[0,1]\mapsto n^a x^n(1-x)$. Déterminer la limite simple de $(f_n)_n$ puis déterminer les valeurs de $a$ pour lesquelles il y a CVU.
\end{Exo}

\begin{Exo}
	Soit $\left( f_{n}\right) ,\left( g_{n}\right) $ deux suites de fonctions continues sur $\left[ a,b\right] $ à valeurs r\'{e}elles, convergeant	uniformément vers $f$ et $g$ sur $\left[ a,b\right]$. Montrez que $f_{n}g_{n}$ converge uniformément vers $fg$ sur $\left[ a,b\right]$.Montrez que ce résultat est en défaut si on ne travaille pas sur un
	segment.
\end{Exo}

%\begin{Exo}
%	Soit $f_{n}:\left[ a,b\right] \rightarrow R$ des fonctions convexes
%	convergeant simplement vers une fonction $f$. Montrer que la convergence est
%	uniforme. (ind: {rendre une subdivision r\'{e}guli\`{e}re de $[a,b]$ et
%		encadrer $f_{n}$ par les cordes associ\'{e}es.)}
%\end{Exo}

%\begin{Exo}
%	Soit $f_{n}$ convergeant uniform\'{e}ment vers $f$, et $g$ une fonction uniformément continue. Démontrer que $g\circ f_{n}\to g\circ f$ uniformément.
%\end{Exo}

\begin{Exo}
	Soit $f_{n}$ convergeant uniform\'{e}ment vers $f$, montrez que $\frac{f_{n}}{1+f_{n}^{2}}$ CVU vers $\frac{f}{1+f^{2}}$.
\end{Exo}

\begin{Exo}
	Soit $f_{n}$ convergeant uniform\'{e}ment sur $\R$ vers $f$, montrez que $\sin \left( f_{n}\right)$ CVU sur $\R$. (On pourra observer que $\sin$ est lipschitzienne sur $\R$.)
\end{Exo}

%\begin{Exo}
%	Soit $\left( f_{n}\right) _{n}$ une suite de fonctions de $[a,b]$ dans $\R$, lipschtiziennes de m\^{e}me rapport $M>0$. On suppose que la	suite $\left( f_{n}\right) _{n}$ est simplement convergente sur $[a,b]$, vers une application $f$. Montrer que $f$ est encore lipschitzienne et que la convergence est
%	uniforme. (Indication : commencer par traiter le cas où $f$ est
%	l'application nulle.)
%\end{Exo}

%\begin{Exo}
%	On pose $f_{n}(x)=x^{n}(1-x)$ et $g_{n}(x)=x^{n}\sin (\pi x)$. Montrer que	la suite $(f_{n})$ converge uniform\'{e}ment vers la fonction nulle sur $[0,1]$. En déduire qu'il en est de même pour la suite $(g_{n})$. (On utilisera la concavité de sin sur $[0,\pi ]$). 
%\end{Exo}

\begin{Exo}
	Soit $f_{n}(x)=n\cos ^{n}x\sin x$. Chercher la limite simple, $f$, des
	fonctions $f_{n},$ calculer $\lim_{n\infty}\int_{0}^{\pi/2}f_{n}(t)dt$.
	Y-a-t'il CVU de $(f_n)n$ vers $f$ sur $[0,\pi/2]$ ? 
\end{Exo}

%\begin{Exo}
%	D\'{e}terminer la limite simple des fonctions $f_{n}:x\mapsto \dfrac{x^{n}e^{-x}}{n!}$ sur $\R_{+}$ et montrer qu'il y a convergence uniforme.
%	Calculer ensuite $\lim_{n\infty }\int_{0}^{+\infty }f_{n}$. Que conclure ?
%\end{Exo}

%\begin{Exo}
%	On pose $f_{n}\left(x\right)=3^{n}\left(x^{2^{n}}-x^{2^{n+1}}\right)$
%	pour $x\in[0,1]$. Etudiez la convergence simple de $\left(f_{n}\right) _{n}.$ Comparer $\int_{0}^{1}\lim_{n}f_{n}$ et $\lim_{n}\int_{0}^{1}f_{n}.$ Que conclure ?
%\end{Exo}

%\begin{Exo}
%	On pose $f_{n}\left( x\right) =1+x^{2}\sin \left( \frac{1}{nx}\right) $ si $x\neq 0,$ $f_{n}\left( 0\right) =1$. Montrez qu'il y a $CVU$  sur tout segment de $\R$ mais pas $CVU$ sur $\R$
%\end{Exo}

\begin{Exo}
	Soit $(P_n)_n$ une suite de fonctions polynômiales convergeant uniformément vers $f$ sur $\R$. Montrer qu'il existe $n_0\in\N$ tel que $\forall x\in\R,\forall n\geqslant n_0,\lvert P_n(x)-f(x)\rvert\leqslant 1$. En déduire que $f$ est une fonction polynôme.
\end{Exo}

\begin{Exo}
	Soit $P_n:x\mapsto \left(1+\frac{x}{n}\right)^n$. Préciser la limite simple, y-a-t'il CVU sur $\R$?
\end{Exo}

\begin{Exo}[difficile]
On pose pour $x\geqslant 0:f_n(x)=\left(1+\frac{x}{n}\right)^{-n}$. Montrer que $(f_n)_n$ CVU sur $\R_+$ vers une fonction à préciser.	
\end{Exo}

\begin{Exo}
	Soit $(f_n)_n$ une suite de fonctions convergeant uniformément sur un intervalle $I$ vers une fonction $f$ continue sur $I$, et soit $(x_n)_n$ suite de $I$ de limite $x$ dans $I$. Montrer que $f_n(x_n)$ tend vers $f(x)$ quand $n$ tend vers l'infini.
\end{Exo}


%\begin{Exo}
%	Soit $f_n:x\in\R_+\mapsto \frac{x}{x^2+n^2}$.
%	\begin{enumerate}
%		\item
%		Montrer que $\sum_nf_n$ converge simplement sur $\R_+$.
%		\item Etudier la CVN sur $\R_+$ puis sur $[a,+\infty[$ avec $a>0$.
%		\item
%		Etudier la CVU sur $\R_+$ puis sur $[a,+\infty[$ avec $a>0$.
%		\item Montrer que $\sum_n (-1)^n f_n$ CVU sur $\R_+$.
%	\end{enumerate}
%\end{Exo}

\begin{Exo}
	CVS et somme de $\sum_n f_n$ avec $f_n:x\mapsto \frac{\sin(nx)}{n2^n}$.
\end{Exo}

\begin{Exo}
	Soit $f_n:x\mapsto \frac{1}{n+n^2x}$.
	\begin{enumerate}
		\item
		Déterminer l'ensemble de définition de $S=\sum_{n=1}^{+\infty}f_n$. Préciser sa monotonie.
		\item Montrer que $S$ est continue sur $\R_+^*$.
		\item Préciser sa limite en $+\infty$.
		\item A l'aide d'une comparaison série intégrale donner un équivalent en $0$ et en $+\infty$ de $S(x)$.
	\end{enumerate}
\end{Exo}

%\begin{Exo}
%	Soit $f_n:x\in\R_+\mapsto nx^2\E^{-\sqrt{n}}$.
%	\begin{enumerate}
%		\item
%		Montrer que $\sum_nf_n$ converge simplement sur $\R_+$.
%		\item Etudier la CVN sur $\R_+$ puis sur $[a,+\infty[$ avec $a>0$.
%		\item
%		Etudier la CVU sur $\R_+$ puis sur $[a,+\infty[$ avec $a>0$.
%		
%	\end{enumerate}
%\end{Exo}



\begin{Exo}
	Etudier CVS,CVU,CVN sur $\R$ de $\sum_n f_n$ avec:
	\begin{enumerate}
		\item
		$f_n(x)=\frac{x}{\left(1+x^2\right)^n}$ avec $n\in\N$.
		\item
		$f_n(x)=(-1)^{n}\frac{x}{\left(1+x^2\right)^n}$ avec $n\in\N$.
		\item
		$f_n(x)=\frac{(-1)^n}{n+x^2}$ avec $n\in\N^*$.
	\end{enumerate}
	
\end{Exo}


\begin{Exo}
	Etudier sur $\R_+$ la CVS,CVN,CVU de la série de fonctions $\sum_n f_n$ avec $f_n=\frac{1}{n+1}\mathbf{1}_{[n,n+1[}$.
\end{Exo}

\begin{Exo}
	Notons pour $n>0$ et $x\in\R:f_n(x)=\frac{x}{x^2+n^2}$.
	\begin{enumerate}
		\item
		Montrer que $\sum_n f_n$ CVU sur tout $[-a,a]$ avec $a>0$. En déduire que $S=\sum_{n=1}^{+\infty}f_n$ est continue sur $\R$.
		\item
		Calculer $\int_0^{+\infty}\frac{x}{x^2+t^2}\diff t$.
		\item
		A l'aide d'une comparaison série-intégrale trouver un équivalent de $S(x)$ quand $x$ tend vers $+\infty$.
		\item
		Y-a-t'il CVU sur $\R_+$?
	\end{enumerate}
\end{Exo}

\begin{Exo}
	On considére la série de fonctions $\Sigma f_{n}$ avec $%
	f_n(x) =\frac{x\E^{-nx}}{\ln n}$ sur $\R_+.$
	
	\begin{enumerate}
		\item A l'aide d'une comparaison série intégrale, montrer que $\sum_n \frac{1}{n\ln n}$ diverge.
		\item Montrer qu'il y a CVS sur $\R_{+},$ CVN sur $\left[
		a,+\infty \right[ $ pour tout $a>0$. Y-a-t'il CVN sur $\R_+$?
		\item Montrer que la somme $S$ de cette série est continue sur $\R_+$.
		\item Montrer que $S$ est $C^{1}$ sur $\R_+^*$ mais non dérivable à droite en $0$.		
		\item Montrer enfin que pour tout entier $k$ on a $\lim_{\infty
		}x^{k}S\left( x\right) =0$.
	\end{enumerate}
\end{Exo}

\begin{Exo}
	Vérifier que l'on peut définir une fonction $f$ sur $\R_+^*$ en posant:
	\[\forall x\geqslant 0,f(x)=\sum_{n=0}^{+\infty}\frac{(-1)^n}{n!(x+n)}\]
	\begin{enumerate}
		\item
		Montrer que $f$ est continue sur $\R_+^*$
		\item
		Montrer que pour tout $x>0:xf(x)-f(x+1)=\E^{-1}$, en déduire un équivalent de $f$ en $0^+$.
	\end{enumerate}
\end{Exo}

\begin{Exo}
	Vérifier que l'on peut définir une fonction $f$ sur $\R_+^*$ en posant: $\forall x> 0,f(x)=\sum_{n=1}^{+\infty}\frac{\E^{-nx}}{n}$. 
	Montrer que $f$ est $C^1$ sur $\R_+^*$, préciser $f'$ et en déduire $f$.
\end{Exo}
\newpage
\begin{Exo}[fonction zéta et zéta alternée de Riemann]\ 
	
	\begin{enumerate}
		\item
		Montrer que l'on définit une fonction sur $]1,+\infty[$ en posant $\zeta(x)=\sum_{n=1}^{+\infty}\frac{1}{n^x}$.
		\item
		Montrer que $\zeta$ est de classe $C^k$ sur $]1,+\infty$ pour tout entier naturel $k$ et préciser $\zeta'$.
		\item
		A l'aide d'une comparaison série-intégrale démontrer que $\zeta(x)\underset{1}{\sim}\frac{1}{x-1}$. \label{equivalent}
		\item Montrer que quand $x\to+\infty$ on a $\zeta(x)=1+2^{-x}+\underset{+\infty}{o}\left(2^{-x}\right)$.
		\item
		Montrer que pour tout $x>0$ la série $\sum_{n>0}\frac{(-1)^{n-1}}{n^x}$ converge. On note $\eta(x)$ sa somme.
		\item
		Montrer que $\eta$ est $C^{\infty}$ sur $\R_+^*$.
		\item Montrer que pour tout $x>1:\eta(x)=\left(1-2^{1-x}\right)\zeta(x)$.
		\item à l'aide de l'équivalent de $\zeta$ en $1$ trouvé en \ref{equivalent} retrouver la valeur de $\sum_{n=1}^{+\infty}\frac{(-1)^{n-1}}{n}$.
		\item Déterminer la limite en $0$ de $\eta$.
		\item A l'aide d'un DL en $1$ de $\eta$ donner la valeur de $\sum_{n=1}^{+\infty}\frac{(-1)^n\ln n}{n}$.
	\end{enumerate}
\end{Exo}

\begin{Exo}
	on note pour $n\geqslant 0:f_n(x)=\begin{cases}(-1)^{n+1}x^{2n+2}\ln(x)\text{ si }x\in]0,1]\\
		0\text{ si }x=0
	\end{cases}$.
	\begin{enumerate}
		\item
		CVS de $\sum_n f_n$ et fonction somme?
		\item Montrer qu'il y a CVU sur $[0,1]$.
		\item En déduire $\int_0^1\frac{\ln x}{1+x^2}\diff x=\sum_{n=0}^{+\infty}\frac{(-1)^{n+1}}{(2n+1)^2}$ et calculer sa valeur en admettant que $\sum_{n=1}^{+\infty}\frac{1}{n^2}=\frac{\pi^2}{6}$.
	\end{enumerate}
\end{Exo}


\begin{Exo}
	D\'{e}montrer que $\int_{0}^{1}\frac{dt}{t^{t}}=\underset{n=1}{\overset%
		{\infty }{\sum }}\frac{1}{n^{n}}$ et que $\int_{0}^{1}t^{t}dt=\underset{n=1}{%
		\overset{\infty }{\sum }}\frac{(-1)^{n-1}}{n^{n}}.$ Valeur approch\'{e}e ?
	%Calculer $\int_{0}^{1}\frac{\ln (1+t)}{t}dt$ (rappel: $\underset{n=1}{%\overset{\infty }{\sum }}\frac{1}{n^{2}}=\frac{\pi ^{2}}{6}$).
	
\end{Exo}
\begin{Exo}
	Dans tout le probl\'{e}me $\alpha $ d\'{e}signe un r\'{e}el. Soit $%
	\left( u_{n}\right) _{n\geq 1}$ la suite de fonctions d\'{e}finies sur $%
	\left[ 0,1\right] $ par: $u_{n}(0)=0$ et $u_{n}(x)=-n^{\alpha }x^{n}\ln
 x $ pour $x>0.$
	
	\begin{enumerate}
		\item Etudier suivant les valeurs de $\alpha $ la convergence et la
		convergence uniforme de la suite $\left( u_{n}\right) $ sur $[0,1]$. Pour
		quelles valeurs de $\alpha $ a-t'on $\lim_{n}\int_{0}^{1}u_{n}(x)dx=%
		\int_{0}^{1}\lim_{n}u_{n}(x)dx$ ?
		
		\item On étudie maintenant la série de fonctions $\Sigma u_{n}.$
		
		\begin{enumerate}
			\item Vérifier que pour tout $x\in \left[ 0,1\right] $ $
			\Sigma u_{n}(x)$ converge. On note $S(x)$ la somme de cette s\'{e}rie.
			
			\item Pour quelle valeurs de $\alpha $ la s\'{e}rie $\Sigma u_{n}$ est-elle
			normalement convergente sur $[0,1]$ ?
			
			\item Pour $\alpha =0$ calculer $S$ et \'{e}tudier sa continuit\'{e} sur $%
			[0,1].$
			
			\item Pour quelles valeurs de $\alpha $ $S$ est-elle continue \`{a} gauche
			en $1$?
			
			\item La fonction $S$ est-elle continue sur $[0,1[$?
		\end{enumerate}
	\end{enumerate}
\end{Exo}
%	
%	\item Montrer que la s\'{e}rie $\Sigma \frac{x^{2}}{1+x^{2}}\left( \frac{1}{%
	%		1+x^{2}}\right) ^{n}$ CVA sur $\left[ -1,1\right] mais$ pas uniform\'{e}%
%	ment. Sa somme est-elle continue sur $\left[ -1,1\right] ?$
%	
%	\item On consid\'{e}re la s\'{e}rie de fonctions $\Sigma f_{n}$ avec :$%
%	\forall n>0,\forall x\in \left[ 0,\pi \right] ,f_{n}\left( x\right) =\sin
%	x\cos ^{n}x.$
%	
%	\begin{enumerate}
	%		\item Montrez que la s\'{e}rie $\Sigma f_{n}$ est simplement convergente sur 
	%		$\left[ 0,\pi \right] $. Justifier rapidement pourquoi la convergence n'est
	%		pas uniforme sur $\left[ 0,\pi \right] $.
	%		
	%		\item Prouver qu'il y a convergence normale sur $\left[ a,\pi -a\right] $
	%		pour tout $0<a<\frac{\pi }{2}.$
	%		
	%		\item Calculez le reste d'indice $N$ de $\Sigma f_{n}$ et retrouvez la non $%
	%		CVU$ sur $\left[ 0,\pi \right] .$
	%		
	%		\item Montrez qu'il y a $CVU$ mais pas $CVN$ sur $\left[ a,\pi \right] $
	%		avec $0<a<\pi .$
	%	\end{enumerate}
%	
%	\item 
%	
%	\item On consid\'{e}re la s\'{e}rie de fonctions $\left[ f_{n}\right] $ avec 
%	$f_{n}\left( x\right) =\frac{e^{-nx}}{\left( n+x\right) ^{2}}.$
%	
%	\begin{enumerate}
	%		\item Etudiez la convergence de cette s\'{e}rie sur $\R_{+}.$
	%		
	%		\item Montrez que sa somme $S$ est continue, positive, d\'{e}croissante sur $%
	%		\R_{+}.$
	%		
	%		\item Pr\'{e}cisez $S\left( 0\right) $ et $\lim_{+\infty }S.$
	%		
	%		\item Montrez que $S$ est $C^{1}$ sur $\R_{+}^{*},$ convexe sur $%
	%		\R_{+}.$
	%		
	%		\item Montrez que $S$ n'est pas d\'{e}rivable en $0.$ On pourra montrer que $%
	%		\lim_{0}S^{\prime }=-\infty .$
	%	\end{enumerate}
%	
%	\item Soit $f(x)=\sum_{n=0}^{\infty }\dfrac{1}{x(x+1)\dots (x+n)}$.
%	
%	\begin{enumerate}
	%		\item \'{E}tablir l'existence et la continuit\'{e} de $f$ sur $\R%
	%		_{+}^{*}.$
	%		
	%		\item Calculer $f(x+1)$ en fonction de $f(x)$.Tracer la courbe de $f$.
	%	\end{enumerate}

%	\item Existence et calcul de $S\left( x\right) =\Sigma _{n=1}^{\infty }\frac{%
	%		1}{n!}\int_{1}^{x}\left( \ln t\right) ^{n}dt$
%	
%	\item On note $P_{n}\left( X\right) =1+X+\frac{X^{2}}{2!}+...+\frac{X^{n}}{n!%
	%	}$ et $r_{n}=\min \left\{ \left\vert z\right\vert ;z\in \mathbb{C}%
%	,P_{n}\left( z\right) =0\right\} .$ Montrez que $r_{n}\rightarrow +\infty .$
%	(on pourra proc\'{e}der par l'absurde).
%	
%\begin{Exo}
%	Etant donn\'{e} $\alpha \in ]0,2\pi \lbrack $ on pose pour $x\in \left[
%	0,1\right] $ et $n\in \mathbb{N}^{\ast }$ $u_{n}(x)=\frac{x^{n}\sin n\alpha 
%	}{n}.$
%	
%	\begin{enumerate}
%		\item Montrer que $\sum_n u_n$ converge uniform\'{e}ment sur $%
%		[0,1].$ On pose pour $x\in \left[ 0,1\right] ,$ $F(x)=\underset{n=1}{\overset%
%			{\infty }{\sum }}\frac{x^{n}\sin n\alpha }{n}.$
%		
%		\item Montrer que $F$ est d\'{e}rivable sur $[0,1[,$ d\'{e}terminer $%
%		F^{\prime }(x)$ et v\'{e}rifier que pour $x\in \left[ 0,1\right]$:
%		\[F(x)=\arctan \frac{x\sin \alpha }{1-x\cos \alpha }\]
%		\item En d\'{e}duire $F(1)$.
%		
%		\item Graphe de la fonction $S$ d\'{e}finie sur $\R$ par $S(\theta )=\underset{n=1}{\overset{\infty }{\sum }}\frac{\sin n\theta }{n}$.
%	\end{enumerate}
%\end{Exo}
%	
%	\item On pose $f\left( x\right) =\sum_{n=1}^{\infty }\arctan \frac{x}{%
	%		x^{2}+n^{2}}.$ Etudier la CV et la CVU. Pr\'{e}cisez la limite de $f$ en $%
%	+\infty .$
%	
%	\item Justifier: $\int_{0}^{1}\ln x\ln \left( 1-x\right)
%	dx=\sum_{n=1}^{\infty }\frac{1}{n\left( n+1\right) ^{2}}.$ Calculer une
%	valeur approch\'{e}e.
%	
%	\item Pour $a>0$ montrez que $\int_{0}^{1}\frac{dx}{1+x^{a}}%
%	=\sum_{n=0}^{\infty }\frac{\left( -1\right) ^{n}}{1+na}.$ En d\'{e}duire $%
%	\sum_{n=0}^{\infty }\frac{\left( -1\right) ^{n}}{2n+1}$ et $%
%	\sum_{n=0}^{\infty }\frac{\left( -1\right) ^{n}}{3n+1}.$
%	
%	\item On pose pour $n\geq 1$ et $x\geq 0,$ $u_{n}\left( x\right) =\frac{nx}{%
	%		\left( n^{2}+x^{2}\right) ^{2}}.$ Etudier les diverses convergences et
%	calculez $\lim_{+\infty }\sum_{n=1}^{\infty }u_{n}\left( x\right) .$
%	
%	\item On pose pour $x\in \R-\mathbb{Z}_{-}:$ $S(x)=\underset{n=0}{%
	%		\overset{\infty }{\sum }}\frac{(-1)^{n}}{n!(x+n)}.$
%	
%	\begin{enumerate}
	%		\item Montrer que $S$ est continue.
	%		
	%		\item Calculer $xS(x)-S(x+1).$
	%		
	%		\item Donner le d\'{e}veloppement asymptotique en $\underset{+\infty }{o}%
	%		\left( \frac{1}{x^{2}}\right) .$
	%		
	%		\item Montrer que pour $x>0,S(x)=\int_{0}^{1}t^{x-1}e^{-t}dt.$
	%	\end{enumerate}
%	
%	\item On pose pour $x$ r\'{e}el $S(x)=\underset{n=1}{\overset{\infty }{\sum }%
	%	}\frac{1}{n}\arctan \frac{x}{n}.$
%	
%	\begin{enumerate}
	%		\item Montrer que $S$ est d\'{e}finie et d\'{e}rivable sur $\R.$
	%		
	%		\item V\'{e}rifier que $S^{\prime }$ est d\'{e}croissante sur $\R%
	%		_{+}.$
	%		
	%		\item D\'{e}terminer $\lim_{+\infty }S(x)$ et $\lim_{+\infty }S^{\prime }(x)$
	%		.
	%		
	%		\item Tracer le graphe de $S.$
	%	\end{enumerate}
%\end{enumerate}
\newpage
\begin{Exo}
	Pour tout $n\in \mathbb{N}^*$, on pose\ $f_{n}\left( x\right) =\dfrac{e^{-x}}{%
		1+n^{2}x^{2}}\ \ $et $u_{n}=\displaystyle\int_{0}^{1}f_{n}\left( x\right) \mathrm{d}x$.
	
	\begin{enumerate}
		\item \'Etudier la convergence simple de la suite de fonctions $\left( f_{n}\right) $  sur $[0,1]$.
		\item
		Soit $a\in\left] 0,1 \right[$.  La suite de  fonctions $\left( f_{n}\right) $ converge-t-elle uniformément sur $\left[a,1 \right]$? 
		\item
		La suite de  fonctions $\left( f_{n}\right) $ converge-t-elle uniformément   sur $[0,1]$?
		\item Trouver la limite de la suite $\left( u_{n}\right) _{n\in \mathbb{N}^*}.$
	\end{enumerate}
\end{Exo}

\begin{Exo}
	Pour tout entier naturel $n$, on d\'efinit sur l'intervalle $J = [1,+\infty[$, la fonction $f_n$ d\'efinie par:
	\[
	f_n(x) = \frac{(-1)^n}{\sqrt{1+nx}}.
	\]
	\begin{enumerate}
		\item D\'emontrer que la s\'erie de fonctions $\displaystyle\sum_{n\geqslant 0} f_n$ converge simplement sur $J$.
		
		On note alors pour tout $x$ de $J$, $\varphi(x)$ sa somme.
		\item Montrer que cette s\'erie de fonctions ne converge pas normalement sur $J$.
		\item \'Etudier alors sa convergence uniforme sur $J$.
		\item D\'eterminer $\ell = \displaystyle\lim_{x\to +\infty} \sum_{n=0}^{+\infty} f_n(x)$.
		\item Pour $n \in \N^*$, on note $u_n = \dfrac{(-1)^n}{\sqrt{n}}$.
		\begin{enumerate}
			\item[5.1.] Justifier la convergence de la s\'erie de terme g\'en\'eral $u_n$. On note $a = \displaystyle\sum_{n=1}^{+\infty} u_n$ sa somme.
			\item[5.2.] Montrer que l'on a au voisinage de l'infini: $\varphi(x) = \ell + \dfrac{a}{\sqrt{x}}+O\left(\dfrac{1}{x^{3/2}}\right)$.
		\end{enumerate}
	\end{enumerate}
	
\end{Exo}

\end{document}
